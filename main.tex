\documentclass[12pt]{report}
\usepackage[utf8]{inputenc}
\usepackage[spanish]{babel}
\usepackage[T1]{fontenc}
\usepackage{geometry}
\geometry{a4paper, margin=3cm}
\usepackage{setspace}
\onehalfspacing\usepackage{graphicx}
\usepackage{float}
\usepackage{caption}
\usepackage{subcaption}
\usepackage{amsmath, amssymb}
\usepackage{hyperref}
\usepackage{natbib}
\usepackage{graphicx}
\usepackage{acronym}
\usepackage[absolute,overlay]{textpos}
\usepackage{babel}
\usepackage{lmodern}
\usepackage{iflang}
\usepackage{xspace}
\usepackage[table,xcdraw]{xcolor}
\usepackage{hyperref}
\usepackage{tabularx}
\usepackage{array}
\usepackage{titlesec}
\usepackage{eso-pic}


\usepackage[utf8]{inputenc}
\usepackage[spanish]{babel}
\usepackage[T1]{fontenc}
\usepackage{geometry}
\geometry{a4paper, margin=3cm}
\usepackage{setspace}
\onehalfspacing\usepackage{graphicx}
\usepackage{float}
\usepackage{caption}
\usepackage{subcaption}
\usepackage{amsmath, amssymb}
\usepackage{hyperref}
\usepackage{natbib}
\usepackage{graphicx}
\usepackage{acronym}
\newcommand{\mytype}{Tesis de pregrado}
\newcommand{\myname}{José Alejandro Arias Pinzón}
\newcommand{\matricle}{1002652342}
\newcommand{\mytitle}{Solución arquitectónica de tecnologías de virtualización basada en contenedores para el grupo de investigación en redes, información y distribución (GRID)}
\newcommand{\myinstitute}{Universidad del Quindío \\ Facultad de ingeniería \\ Programa de Ingeniería de Sistemas}
\newcommand{\reviewerone}{Dra. Diana Marcela Rivera Valencia}
\newcommand{\advisor}{Ph.D. Luis Eduardo Sepúlveda Rodríguez}
\newcommand{\timeend}{Julio 2025}
\newcommand{\submissiontime}{16.07.2025}
\newcommand{\mynameb}{Anubis Haxard Correa Urbano}
\newcommand{\matricleb}{1004871385} 

% Configuración de encabezados y pies de página
\pagestyle{fancy}
\fancyhf{} % Limpia todos los encabezados y pies
\fancyfoot[R]{\thepage} % Número de página abajo a la derecha
\renewcommand{\headrulewidth}{0pt} % Sin línea en el encabezado
\renewcommand{\footrulewidth}{0pt} % Sin línea en el pie

% Marcador para la sección de Siglas
\newcommand{\siglaref}{\hyperref[cap:siglas]{\nameref{cap:siglas}}}

% Ahora definimos cada sigla con enlace hacia la lista
\newcommand{\API}{\hyperref[cap:siglas]{API}\xspace}
\newcommand{\CMMI}{\hyperref[cap:siglas]{CMMI}\xspace}
\newcommand{\CPU}{\hyperref[cap:siglas]{CPU}\xspace}
\newcommand{\DAR}{\hyperref[cap:siglas]{DAR}\xspace}
\newcommand{\GRID}{\hyperref[cap:siglas]{GRID}\xspace}
\newcommand{\IT}{\hyperref[cap:siglas]{IT}\xspace}
\newcommand{\SMS}{\hyperref[cap:siglas]{SMS}\xspace}
\newcommand{\VBC}{\hyperref[cap:siglas]{VBC}\xspace}
\newcommand{\TI}{\hyperref[cap:siglas]{TI}\xspace}
\newcommand{\PMV}{\hyperref[cap:siglas]{PMV}\xspace}
\newcommand{\PMBOK}{\hyperref[cap:siglas]{PMBOK}\xspace}
\newcommand{\ISO}{\hyperref[cap:siglas]{ISO}\xspace}
\newcommand{\CNCF}{\hyperref[cap:siglas]{CNCF}\xspace}
\newcommand{\TOGAF}{\hyperref[cap:siglas]{TOGAF}\xspace}
\newcommand{\IEC}{\hyperref[cap:siglas]{IEC}\xspace}
\newcommand{\CS}{\hyperref[cap:siglas]{CS}\xspace}

% Comando para capítulos SIN numeración de página (páginas especiales)
\newcommand{\ChapterImageEmpty}[3][]{%
  \cleardoublepage\thispagestyle{empty}%
  \refstepcounter{chapter}%
  \begingroup
    \begin{tikzpicture}[remember picture,overlay]
      % Imagen de fondo al ancho de la página
      \node[inner sep=0pt, anchor=north west] at (current page.north west)
        {\includegraphics[width=\paperwidth]{#3}};
      % Caja blanca que se adapta al ancho del texto
      \node[
        anchor=north west,
        xshift=1.5cm, yshift=-1.0cm,
        text=black,
        fill=white, fill opacity=0.8,
        rounded corners=6pt,
        inner xsep=10pt, inner ysep=6pt
      ] at (current page.north west)
        {\Huge\bfseries \thechapter\quad #2\strut};
    \end{tikzpicture}
  \endgroup
}

% Comando para capítulos CON numeración de página (capítulos normales)
\newcommand{\ChapterImageStar}[3][]{%
  \cleardoublepage%
  \refstepcounter{chapter}%
  \begingroup
    \begin{tikzpicture}[remember picture,overlay]
      % Imagen de fondo al ancho de la página
      \node[inner sep=0pt, anchor=north west] at (current page.north west)
        {\includegraphics[width=\paperwidth]{#3}};
      % Caja blanca que se adapta al ancho del texto
      \node[
        anchor=north west,
        xshift=1.5cm, yshift=-1.0cm,
        text=black,
        fill=white, fill opacity=0.8,
        rounded corners=6pt,
        inner xsep=10pt, inner ysep=6pt
      ] at (current page.north west)
        {\Huge\bfseries \thechapter\quad #2\strut};
    \end{tikzpicture}
  \endgroup
}

\newcommand\BackgroundPic{%
    \put(-25,-2){%
        \parbox[b][\paperheight]{\paperwidth}{%
            \vfill
            \centering
            \includegraphics[scale=1,height=\paperheight]{./images/fondo-anteportada.png}%
            \vfill
}}}

\newcommand\PortadaPic{%
    \put(0,0){%
        \parbox[b][\paperheight]{\paperwidth}{%
            \vfill
            \centering
            \includegraphics[width=\paperwidth,height=\paperheight]{images/portada.png}%
            \vfill
}}}

\begin{document}


\input{anteportada.tex}

\begin{titlepage}
\centering
{\Large \textbf{Título de la Tesis}}\\[1cm]
Autor: Tu Nombre\\
Universidad del Ejemplo\\
Facultad de Ingeniería\\
Julio de 2025
\end{titlepage}


% Página en blanco antes de la dedicatoria
\cleardoublepage%
\thispagestyle{empty}
\vspace*{\fill}
\begin{center}
\small\textit{Página en blanco intencionalmente}
\end{center}
\vspace*{\fill}
\newpage

\input{capitulos/dedicatoria.tex}

% Página en blanco después de la dedicatoria
\cleardoublepage%
\thispagestyle{empty}
\vspace*{\fill}
\begin{center}
\small\textit{Página en blanco intencionalmente}
\end{center}
\vspace*{\fill}
\newpage

\input{indice.tex}
\ChapterImagePrelim[cap:resumen]{Resumen}{./images/fondo.png}\label{cap:resumen}
\mbox{}\\
El presente trabajo de grado desarrolla la especificación de una solución arquitectónica basada en tecnologías de virtualización por contenedores (\VBC) para el Grupo de Investigación en Redes, Información y Distribución (\GRID) de la Universidad del Quindío. El estudio surge como respuesta a la necesidad de ampliar el portafolio de servicios tecnológicos mediante instancias computacionales más ligeras que complementen la infraestructura existente basada en máquinas virtuales. 
Metodológicamente, la investigación comprende: un estudio de mapeo sistemático para identificar tecnologías de \VBC; un \textit{benchmarking} técnico de consumo de \CPU, memoria, tiempo de arranque, \textit{throughput} y latencia de E/S; y \@la aplicación de la metodología de Análisis de Decisiones y Resolución (\DAR) del modelo \CMMI. Los resultados señalaron a Containerd como la tecnología más adecuada, mientras que K3S se identificó como el motor de orquestación más viable. 
Finalmente, se propone un diseño arquitectónico modelado en Archimate que articula la infraestructura existente con la capa de virtualización (Containerd + K3S) y la capa de aplicación. La solución ofrece un servicio escalable y mantenible alineado con las necesidades académicas e investigativas del \GRID. El trabajo concluye con la implementación de un producto mínimo viable, que valida la pertinencia de la propuesta y establece un referente metodológico para decisiones tecnológicas en contextos académicos e institucionales. 

\ChapterImagePrelim[cap:abstract]{Abstract}{./images/fondo.png}\label{cap:abstract}
\mbox{}\\
This thesis develops the specification of an architectural solution based on container virtualization technologies (CVT) for the Research Group in Networks, Information, and Distribution (\GRID) at the University of Quindío. The study arises as a response to the need to expand the portfolio of technological services through lighter computational instances that complement the existing infrastructure based on virtual machines. Methodologically, the research comprises: a systematic mapping study to identify CVT technologies; a technical benchmarking of \CPU\ and memory consumption, startup time, throughput, and I/O latency; and the application of the Decision Analysis and Resolution (\DAR) methodology from the CMMI model. The results identified Containerd as the most suitable technology, while K3S was determined to be the most viable orchestration engine. Finally, an architectural design modeled in Archimate is proposed, articulating the existing infrastructure with the virtualization layer (Containerd + K3S) and the application layer. The solution provides a scalable and maintainable service aligned with the academic and research needs of \GRID. The work concludes with the implementation of a minimum viable product, which validates the relevance of the proposal and establishes a methodological reference for technological decision-making in academic and institutional contexts.

\input{capitulos/listaFiguras.tex}
\input{capitulos/listaTablas.tex}

\input{capitulos/introduccion.tex}
\input{capitulos/glosario.tex} 
\chapter*{Siglas y Abreviaturas}
\addcontentsline{toc}{chapter}{Siglas y Abreviaturas}
\label{cap:siglas}
\begin{acronym}[BASH]  % El parámetro aquí solo indica el ancho máximo para la alineación
  \acro{API}{Interfaz de Programación de Aplicaciones}
  \acro{CMMI}{Capability Maturity Model Integration}
  \acro{CPU}{Unidad Central de Procesamiento}
  \acro{DAR}{Decision Analysis and Resolution}
  \acro{GRID}{Grupo de Investigación en Redes, Información y Distribución}
  \acro{IT}{Tecnologías de la Información}
  \acro{SMS}{Systematic Mapping Study}
  \acro{VBC}{Virtualización Basada en Contenedores}
\end{acronym}

\clearpage  % <-- Esto asegura que el contenido siguiente empiece en una nueva página
\section*{1. Objetivos}

\subsection*{1.1 Objetivo general}
Especificar una arquitectura de tecnologías de virtualización basadas en contenedores (VBC), evaluando sus características a través de un benchmarking, seleccionando la que mejor se adapte a la necesidad, problema y oportunidad del GRID (Grupo de Investigación en Redes, Información y Distribución), haciendo un análisis DAR e implementando un producto mínimo viable (PMV).

\subsection*{1.2 Objetivos específicos}
\begin{itemize}
    \item Reconocer necesidades del GRID (Grupo de Investigación en Redes, Información y Distribución) con relación a las tecnologías de virtualización basadas en contenedores.
    \item Identificar las tecnologías de virtualización basadas en contenedores.
    \item Caracterizar tecnologías de virtualización basadas en contenedores.
    \item Seleccionar un conjunto de tecnologías de contenedores para realizar pruebas de concepto.
    \item Diseñar una especificación arquitectónica para las herramientas seleccionadas.
    \item Implementar el prototipo funcional.
    \item Validar casos con relación a la necesidad del cliente.
\end{itemize}
\chapter*{Justificación}
Actualmente el Grupo de Investigación en Redes, Información y Distribución (GRID) presenta diversas necesidades y oportunidades con relación a los servicios tecnológicos que ofrece a la Universidad del Quindío, en apoyo a sus objetivos misionales de docencia, investigación y extensión. En este contexto, el GRID busca identificar tecnologías emergentes que permitan potenciar su capacidad de brindar servicios tecnológicos avanzados para su propio beneficio y de la comunidad académica de su influencia. Con relación a lo anterior, la virtualización basada en procesos se presenta como una oportunidad para potenciar la gestión de recursos y servicios de tecnología informática (TI). Aunque el GRID cuenta con una infraestructura basada en máquinas virtuales y gestionadas mediante un hipervisor tipo I, aún se requiere de instancias computacionales más livianas para ser usadas en su oferta de servicios computacionales hacia la comunidad académica, representada por los estudiantes de Ingeniería de Sistemas y Computación de la Universidad del Quindío.
Como menciona Sepúlveda-Rodríguez, Chavarro-Porras, Sanabria-Ordoñez, Castro y Matthews (2022), las tecnologías de virtualización han proliferado en los últimos años y constituyen la base subyacente de infraestructuras modernas como el cloud computing. La VBC representan una opción de virtualización que requieren menos recursos computacionales para su operación (Xavier et al., 2013) y que en paralelo con las máquinas virtuales existentes en el GRID podrían constituir una oferta de servicios de TI con mayor diversificación, escalabilidad, flexibilidad y mantenibilidad para satisfacer los requerimientos del contexto académico del grupo de investigación. 
\chapter*{Metodología} Texto de la metodología.
\addcontentsline{toc}{chapter}{Metodología}\label{cap:metodologia}
\chapter*{Marco Conceptual}
\addcontentsline{toc}{chapter}{Marco Conceptual}\label{cap:marcoConceptual}
Texto del marco conceptual.
\input{capitulos/marcoTeorico.tex}
\chapter{Desarrollo Metodológico}

\section*{1. Entendimiento de la organización}
El primer paso en el desarrollo metodológico es comprender la organización y su contexto. Esto implica identificar

\section*{2. Definición de objetivos}
Los objetivos del proyecto deben ser claros y alcanzables. Esto incluye la identificación de las metas a corto y largo plazo, así como los indicadores de éxito que se utilizarán para medir el progreso.


\chapter*{1. Caracterización del GRID}
El Grupo de Investigación en Redes, Información y Distribución (GRID) de la Universidad del Quindío se dedica a la educación, 
investigación y extensión, siendo los objetivos misionales de la Universidad del Quindío. Desde el grupo de investigación se busca ofrecer servicios tecnológicos
a la comunidad académica, especialmente a los estudiantes de Ingeniería de Sistemas y Computación.

\section*{1.1 Análisis de stakeholders del grupo GRID}

Para comprender mejor las necesidades y expectativas del GRID, se realizó un análisis de los stakeholders involucrados. Este análisis incluyó a los miembros del grupo de investigación, estudiantes, 
docentes, entre otros, identificando sus roles, impacto y poder de influencia por una solución basada en las tecnologías de virtualización basadas en contenedores (VBC).

\begin{figure}[H]
    \centering
    \includegraphics[width=\textwidth] {tablas/cp1/definicionStakeholders.png}
    \caption{Análisis de stakeholders del proyecto}\label{fig:tabla-stakeholders}
\end{figure}

\section*{1.2 Priorización de stakeholders}
A partir del análisis de stakeholders, se priorizaron aquellos que tienen mayor impacto y poder de influencia en el proyecto. Esta priorización permite enfocar los esfuerzos de comunicación y gestión de expectativas hacia
los stakeholders más relevantes, asegurando que sus necesidades sean atendidas de manera especial.

\begin{figure}[H]
    \centering
    \includegraphics[width=\textwidth] {tablas/cp1/priorizacionStakeholders.png}
    \caption{Priorización de stakeholders del proyecto}\label{fig:tabla-priorizacion-stakeholders}
\end{figure}

\section*{1.3 Integrantes y áreas de trabajo del GRID}

El GRID está compuesto por un equipo multidisciplinario de investigadores y profesionales, cada uno con áreas de especialización diferentes. A continuación se presentan los diferentes integrantes y sus respectivas áreas de trabajo:

\begin{itemize}
  \item \href{https://scienti.minciencias.gov.co/cvlac/visualizador/generarCurriculoCv.do?cod_rh=0000210897}{\textbf{Christian Andrés Candela Uribe}}: Microservicios, desarrollo de software, minería de datos, infraestructura TI
  \item \href{https://scienti.minciencias.gov.co/cvlac/visualizador/generarCurriculoCv.do?cod_rh=0001383939}{\textbf{Luis Eduardo Sepúlveda Rodríguez}}: Infraestructura de TI, HPC, computación paralela
  \item \href{https://scienti.minciencias.gov.co/cvlac/visualizador/generarCurriculoCv.do?cod_rh=0001638854}{\textbf{Carlos Andrés Flórez Villarraga}}: Programación y algoritmia, Activa, inteligencia artificial
  \item \href{https://scienti.minciencias.gov.co/cvlac/visualizador/generarCurriculoCv.do?cod_rh=0001343801}{\textbf{Carlos Eduardo Gómez Montoya}}: Networking, ingeniería de software, cloud computing
  \item \href{https://scienti.minciencias.gov.co/cvlac/visualizador/generarCurriculoCv.do?cod_rh=0001398775}{\textbf{Sergio Augusto Cardona Torres}}: Big data y análisis de datos, ingeniería de software, educación asistida por computador - sistemas adaptativos, informática educativa
  \item \href{https://scienti.minciencias.gov.co/cvlac/visualizador/generarCurriculoCv.do?cod_rh=0000193550}{\textbf{Sonia Jaramillo Valbuena}}: Big data, electroquímica, inteligencia artificial
  \item \href{https://scienti.minciencias.gov.co/cvlac/visualizador/generarCurriculoCv.do?cod_rh=0000283495}{\textbf{Julián Esteban Gutiérrez Posada}}: Microservicios, desarrollo de software, minería de datos, infraestructura TI, HPC, computación paralela, networking, ingeniería de software
\end{itemize}

\section*{1.4 Misión del GRID}
\addcontentsline{toc}{section}{1.4 Misión del GRID}

La misión del GRID es heredada de la Universidad del Quindío. A continuación se presenta la misión del GRID:

\begin{quote}
La Universidad del Quindío contribuye a la transformación de la sociedad, mediante la formación integral desde el ser, el saber y el hacer, de líderes reflexivos y gestores del cambio; con estándares de calidad, a través de una oferta de formación en diferentes metodologías, que responda a una sociedad basada en el conocimiento; una investigación pertinente, que aporte a la solución de las problemáticas del desarrollo e integrada con la extensión y proyección social; educando en tiempos del posconflicto y de la consolidación de la paz, apoyada en una gestión creativa y con estándares de calidad.
\end{quote}

A partir de esta misión, se identifican los siguientes pilares fundamentales:

\begin{itemize}
    \item \textbf{Docencia:} La Universidad del Quindío contribuye a la transformación de la sociedad, mediante la formación integral desde el ser, el saber y el hacer, de líderes reflexivos y gestores del cambio; con estándares de calidad, a través de una oferta de formación en diferentes metodologías, que responda a una sociedad basada en el conocimiento.

    \item \textbf{Investigación:} Una investigación pertinente, que aporte a la solución de las problemáticas del desarrollo e integrada con la extensión y proyección social.

    \item \textbf{Extensión y Desarrollo Social:} Apoyada en una gestión creativa y con estándares de calidad.

    \item \textbf{Responsabilidad Social:} Educando en tiempos del posconflicto y de la consolidación de la paz.
\end{itemize}

\section*{1.5 Visión del GRID}
\addcontentsline{toc}{section}{1.5 Visión del GRID}

\begin{quote}
En el año 2025, la Universidad del Quindío estará consolidada como una institución \textit{Pertinente - Creativa - Integradora}, acreditada de alta calidad, con reconocimiento nacional e internacional en sus procesos de formación a través de diferentes metodologías, de investigación, extensión, proyección y responsabilidad social.
\end{quote}

A partir de esta visión, se destacan los siguientes enfoques estratégicos:

\begin{itemize}
    \item \textbf{Gestión:} La Universidad del Quindío estará consolidada como una institución \textit{Pertinente - Creativa - Integradora}.

    \item \textbf{Docencia:} Acreditada de alta calidad en sus procesos de formación a través de diferentes metodologías.

    \item \textbf{Investigación:} Consolidada como pertinente y de alta calidad en sus procesos de investigación.

    \item \textbf{Extensión y Desarrollo Social:} Procesos creativos e integradores en proyección social.

    \item \textbf{Responsabilidad Social:} Reconocimientos en sus procesos de responsabilidad social.
\end{itemize}

\section*{1.6 Impacto del proyecto en el GRID}
\addcontentsline{toc}{section}{1.6 Impacto del proyecto en el GRID}

El proyecto tiene como objetivo apoyar los procesos de \textbf{docencia}, \textbf{investigación} 
y \textbf{extensión} mediante la especificación de una arquitectura de tecnologías de 
virtualización basada en contenedores (VBC). 

Este trabajo se enfoca en la identificación de una tecnología de contenerización que 
\textbf{agregue valor a los procesos del GRID}, beneficiando a \textbf{docentes}, \textbf{estudiantes} 
y cualquier parte interesada que participe en los proyectos y actividades desarrolladas 
por este grupo de investigación.

\section*{1.7 Caracterización de la infraestructura tecnológica del GRID}
En el siguiente formato se van a especificar las características técnicas de la infraestructura tecnológica del GRID disponible para temas de virtualización. \href{https://docs.google.com/spreadsheets/d/14NBv72ucVTrLqGIldYdIsjdBGt3QlgwcblcVRis-DaQ/edit?usp=sharing}{Macro de la ficha técnica}

% Archivo de caracterización de infraestructura corregido

% Torre HP 1
\begin{table}[H]
\centering
\caption{Ficha técnica --- Torre 1}\label{tab:torre-hp-1}
\begin{tabular}{|p{0.6\textwidth}|p{0.3\textwidth}|}
\hline
\multicolumn{2}{|l|}{\textbf{DESCRIPCIÓN FÍSICA:} Servidor tipo torre} \\ \hline
\textbf{TIPO DE RECURSO:} Torre &
\multirow{5}{*}{\includegraphics[width=0.25\textwidth,height=4cm,keepaspectratio]{tablas-images/cp1/torres/torre-1.png}} \\ \cline{1-1}
\textbf{MODELO:} Desconocido & \\ \cline{1-1}
\textbf{MARCA:} HP & \\ \cline{1-1}
\textbf{CÓDIGO DE INVENTARIO:} 7 24390 49867 3 & \\ \cline{1-1}
\textbf{NÚMERO EN CPD:} 14 & \\ \hline
\multicolumn{2}{|l|}{\textbf{ESPECIFICACIONES TÉCNICAS}} \\ \hline
\multicolumn{2}{|p{0.95\textwidth}|}{
\footnotesize
- 8 entradas USB (4 al frente, 4 en la parte trasera)
- Entrada de audio y microfono
- Entrada HDMI
- Lector de DVDs
- 3 puertos Ethernet (Parte trasera)
- Entrada Displayport
- Puertos PS/2 (Teclado y Ratón)
} \\ \hline
\multicolumn{2}{|l|}{\textbf{PROPÓSITO:} Hipervisor de XCP-ng} \\ \hline
\multicolumn{2}{|l|}{\textbf{OPORTUNIDAD DE USO:} Proyectos del \GRID} \\ \hline
\multicolumn{2}{|p{0.9\textwidth}|}{\textbf{OBSERVACIONES:} El Equipo no tiene modelo. El equipo está diseñado para usuario final pero fue adaptado para entornos de virtualización.} \\ \hline
\end{tabular}
\end{table}


% Torre 2
\begin{table}[H]
\centering
\caption{Ficha técnica --- Torre 2}\label{tab:torre-2}
\begin{tabular}{|p{0.6\textwidth}|p{0.3\textwidth}|}
\hline
\multicolumn{2}{|l|}{\textbf{DESCRIPCIÓN FÍSICA:} Servidor tipo torre} \\ \hline
\textbf{TIPO DE RECURSO:} Torre & 
\multirow{5}{*}{\includegraphics[width=0.25\textwidth,height=4cm,keepaspectratio]{tablas-images/cp1/torres/torre-1.png}} \\ \cline{1-1}
\textbf{MODELO:} Desconocido & \\ \cline{1-1}
\textbf{MARCA:} HP & \\ \cline{1-1}
\textbf{CÓDIGO DE INVENTARIO:} 7 24390 49861 1 & \\ \cline{1-1}
\textbf{NÚMERO EN CPD:} 12 & \\ \hline
\multicolumn{2}{|l|}{\textbf{ESPECIFICACIONES TÉCNICAS}} \\ \hline
\multicolumn{2}{|p{0.95\textwidth}|}{
\footnotesize
- 8 entradas USB (4 al frente, 4 en la parte trasera)
- Entrada de audio y microfono
- Entrada HDMI
- Lector de DVDs
- 3 puertos Ethernet (Parte trasera)
- Entrada Displayport
- Puertos PS/2 (Teclado y Ratón)
} \\ \hline
\multicolumn{2}{|l|}{\textbf{PROPÓSITO:} Hipervisor de XCP-ng} \\ \hline
\multicolumn{2}{|l|}{\textbf{OPORTUNIDAD DE USO:} Proyectos del \GRID} \\ \hline
\multicolumn{2}{|p{0.9\textwidth}|}{\textbf{OBSERVACIONES:} El Equipo no tiene modelo. El equipo está diseñado para usuario final pero fue adaptado para entornos de virtualización.} \\ \hline
\end{tabular}
\end{table}

% Torre 3
\begin{table}[H]
\centering
\caption{Ficha técnica -- Torre 3}
\label{tab:torre-3}
\begin{tabular}{|p{0.6\textwidth}|p{0.3\textwidth}|}
\hline
\multicolumn{2}{|l|}{\textbf{DESCRIPCIÓN FÍSICA:} Servidor tipo torre} \\ \hline
\textbf{TIPO DE RECURSO:} Torre & 
\multirow{5}{*}{\includegraphics[width=0.25\textwidth,height=4cm,keepaspectratio]{tablas-images/cp1/torres/torre-1.png}} \\ \cline{1-1}
\textbf{MODELO:} Desconocido & \\ \cline{1-1}
\textbf{MARCA:} HP & \\ \cline{1-1}
\textbf{CÓDIGO DE INVENTARIO:} 7 24390 49969 4 & \\ \cline{1-1}
\textbf{NÚMERO EN CPD:} 13 & \\ \hline
\multicolumn{2}{|l|}{\textbf{ESPECIFICACIONES TÉCNICAS}} \\ \hline
\multicolumn{2}{|p{0.95\textwidth}|}{
\footnotesize
- 8 entradas USB (4 al frente, 4 en la parte trasera)
- Entrada de audio y microfono
- Entrada HDMI
- Lector de DVDs
- 3 puertos Ethernet (Parte trasera)
- Entrada Displayport
- Puertos PS/2 (Teclado y Ratón)
} \\ \hline
\multicolumn{2}{|l|}{\textbf{PROPÓSITO:} Hipervisor de XCP-ng} \\ \hline
\multicolumn{2}{|l|}{\textbf{OPORTUNIDAD DE USO:} Proyectos del \GRID} \\ \hline
\multicolumn{2}{|p{0.9\textwidth}|}{\textbf{OBSERVACIONES:} El Equipo no tiene modelo. El equipo está diseñado para usuario final pero fue adaptado para entornos de virtualización.} \\ \hline
\end{tabular}
\end{table}

% Torre 4
\begin{table}[H]
\centering
\caption{Ficha técnica --- Torre 4}
\label{tab:torre-4}
\begin{tabular}{|p{0.6\textwidth}|p{0.3\textwidth}|}
\hline
\multicolumn{2}{|l|}{\textbf{DESCRIPCIÓN FÍSICA:} Servidor tipo torre} \\ \hline
\textbf{TIPO DE RECURSO:} Torre & 
\multirow{5}{*}{\includegraphics[width=0.25\textwidth,height=4cm,keepaspectratio]{tablas-images/cp1/torres/torre-1.png}} \\ \cline{1-1}
\textbf{MODELO:} Desconocido & \\ \cline{1-1}
\textbf{MARCA:} HP & \\ \cline{1-1}
\textbf{CÓDIGO DE INVENTARIO:} 7 24390 49879 4 & \\ \cline{1-1}
\textbf{NÚMERO EN CPD:} 14 & \\ \hline
\multicolumn{2}{|l|}{\textbf{ESPECIFICACIONES TÉCNICAS}} \\ \hline
\multicolumn{2}{|p{0.95\textwidth}|}{
\footnotesize
- 8 entradas USB (4 al frente, 4 en la parte trasera)
- Entrada de audio y microfono
- Entrada HDMI
- Lector de DVDs
- 3 puertos Ethernet (Parte trasera)
- Entrada Displayport
- Puertos PS/2 (Teclado y Ratón)
} \\ \hline
\multicolumn{2}{|l|}{\textbf{PROPÓSITO:} Hipervisor de XCP-ng} \\ \hline
\multicolumn{2}{|l|}{\textbf{OPORTUNIDAD DE USO:} Proyectos del \GRID} \\ \hline
\multicolumn{2}{|p{0.9\textwidth}|}{\textbf{OBSERVACIONES:} El Equipo no tiene modelo. El equipo está diseñado para usuario final pero fue adaptado para entornos de virtualización.} \\ \hline
\end{tabular}
\end{table}

% Torre 5
\begin{table}[H]
\centering
\caption{Ficha técnica --- Torre 5}
\label{tab:torre-5}
\begin{tabular}{|p{0.6\textwidth}|p{0.3\textwidth}|}
\hline
\multicolumn{2}{|l|}{\textbf{DESCRIPCIÓN FÍSICA:} Servidor tipo torre} \\ \hline
\textbf{TIPO DE RECURSO:} Torre & 
\multirow{5}{*}{\includegraphics[width=0.25\textwidth,height=4cm,keepaspectratio]{tablas-images/cp1/torres/torre-2.png}} \\ \cline{1-1}
\textbf{MODELO:} G9 & \\ \cline{1-1}
\textbf{MARCA:} HP & \\ \cline{1-1}
\textbf{CÓDIGO DE INVENTARIO:} 72992 & \\ \cline{1-1}
\textbf{NÚMERO EN CPD:} 22 & \\ \hline
\multicolumn{2}{|l|}{\textbf{ESPECIFICACIONES TÉCNICAS}} \\ \hline
\multicolumn{2}{|p{0.95\textwidth}|}{
\footnotesize
- 9 entradas USB (4 al frente, 5 en la parte trasera)
- Entrada de audio y microfono
- Entrada HDMI
- Lector de DVDs
- 1 puerto Ethernet (Parte trasera)
- 2 Entrada Displayport
- Procesador Intel vPro i9
} \\ \hline
\multicolumn{2}{|l|}{\textbf{PROPÓSITO:} Hipervisor de XCP-ng} \\ \hline
\multicolumn{2}{|l|}{\textbf{OPORTUNIDAD DE USO:} Proyectos del \GRID} \\ \hline
\multicolumn{2}{|l|}{\textbf{OBSERVACIONES:} El equipo está diseñado para usuario final pero fue adaptado para entornos de virtualización.} \\ \hline
\end{tabular}
\end{table}

% Torre 6
\begin{table}[H]
\centering
\caption{Ficha técnica -- Torre 6}
\label{tab:torre-6}
\begin{tabular}{|p{0.6\textwidth}|p{0.3\textwidth}|}
\hline
\multicolumn{2}{|l|}{\textbf{DESCRIPCIÓN FÍSICA:} Servidor tipo torre} \\ \hline
\textbf{TIPO DE RECURSO:} Torre & 
\multirow{5}{*}{\includegraphics[width=0.25\textwidth,height=4cm,keepaspectratio]{tablas-images/cp1/torres/torre-6.png}} \\ \cline{1-1}
\textbf{MODELO:} Por definir & \\ \cline{1-1}
\textbf{MARCA:} Por definir & \\ \cline{1-1}
\textbf{CÓDIGO DE INVENTARIO:} Por definir & \\ \cline{1-1}
\textbf{FECHA DE ADQUISICIÓN (APROX.):} & \\ \hline
\multicolumn{2}{|l|}{\textbf{ESPECIFICACIONES TÉCNICAS}} \\ \hline
\multicolumn{2}{|p{0.95\textwidth}|}{
\footnotesize
Especificaciones por definir según imagen adjunta
} \\ \hline
\multicolumn{2}{|l|}{\textbf{PROPÓSITO:} Por definir} \\ \hline
\multicolumn{2}{|l|}{\textbf{IMPACTO:} Por evaluar} \\ \hline
\multicolumn{2}{|l|}{\textbf{OBSERVACIONES:} Ver imagen para detalles} \\ \hline
\end{tabular}
\end{table}

% Torre 7
\begin{table}[H]
\centering
\caption{Ficha técnica -- Torre 7}
\label{tab:torre-7}
\begin{tabular}{|p{0.6\textwidth}|p{0.3\textwidth}|}
\hline
\multicolumn{2}{|l|}{\textbf{DESCRIPCIÓN FÍSICA:} Servidor tipo torre} \\ \hline
\textbf{TIPO DE RECURSO:} Torre & 
\multirow{5}{*}{\includegraphics[width=0.25\textwidth,height=4cm,keepaspectratio]{tablas-images/cp1/torres/torre-7.png}} \\ \cline{1-1}
\textbf{MODELO:} Por definir & \\ \cline{1-1}
\textbf{MARCA:} Por definir & \\ \cline{1-1}
\textbf{CÓDIGO DE INVENTARIO:} Por definir & \\ \cline{1-1}
\textbf{FECHA DE ADQUISICIÓN (APROX.):} & \\ \hline
\multicolumn{2}{|l|}{\textbf{ESPECIFICACIONES TÉCNICAS}} \\ \hline
\multicolumn{2}{|p{0.95\textwidth}|}{
\footnotesize
Especificaciones por definir según imagen adjunta
} \\ \hline
\multicolumn{2}{|l|}{\textbf{PROPÓSITO:} Por definir} \\ \hline
\multicolumn{2}{|l|}{\textbf{IMPACTO:} Por evaluar} \\ \hline
\multicolumn{2}{|l|}{\textbf{OBSERVACIONES:} Ver imagen para detalles} \\ \hline
\end{tabular}
\end{table}

% Rack 1
\begin{table}[H]
\centering
\caption{Ficha técnica -- Rack 1}
\label{tab:rack-1}
\begin{tabular}{|p{0.6\textwidth}|p{0.3\textwidth}|}
\hline
\multicolumn{2}{|l|}{\textbf{DESCRIPCIÓN FÍSICA:} Servidor tipo rack} \\ \hline
\textbf{TIPO DE RECURSO:} Rack & 
\multirow{5}{*}{\includegraphics[width=0.25\textwidth,height=4cm,keepaspectratio]{tablas-images/cp1/racks/rack-1.png}} \\ \cline{1-1}
\textbf{MODELO:} Por definir & \\ \cline{1-1}
\textbf{MARCA:} Por definir & \\ \cline{1-1}
\textbf{CÓDIGO DE INVENTARIO:} Por definir & \\ \cline{1-1}
\textbf{FECHA DE ADQUISICIÓN (APROX.):} & \\ \hline
\multicolumn{2}{|l|}{\textbf{ESPECIFICACIONES TÉCNICAS}} \\ \hline
\multicolumn{2}{|p{0.95\textwidth}|}{
\footnotesize
Especificaciones por definir según imagen adjunta
} \\ \hline
\multicolumn{2}{|l|}{\textbf{PROPÓSITO:} Por definir} \\ \hline
\multicolumn{2}{|l|}{\textbf{IMPACTO:} Por evaluar} \\ \hline
\multicolumn{2}{|l|}{\textbf{OBSERVACIONES:} Ver imagen para detalles} \\ \hline
\end{tabular}
\end{table}

% Rack 2
\begin{table}[H]
\centering
\caption{Ficha técnica -- Rack 2}
\label{tab:rack-2}
\begin{tabular}{|p{0.6\textwidth}|p{0.3\textwidth}|}
\hline
\multicolumn{2}{|l|}{\textbf{DESCRIPCIÓN FÍSICA:} Servidor tipo rack} \\ \hline
\textbf{TIPO DE RECURSO:} Rack & 
\multirow{5}{*}{\includegraphics[width=0.25\textwidth,height=4cm,keepaspectratio]{tablas-images/cp1/racks/rack-2.png}} \\ \cline{1-1}
\textbf{MODELO:} Por definir & \\ \cline{1-1}
\textbf{MARCA:} Por definir & \\ \cline{1-1}
\textbf{CÓDIGO DE INVENTARIO:} Por definir & \\ \cline{1-1}
\textbf{FECHA DE ADQUISICIÓN (APROX.):} & \\ \hline
\multicolumn{2}{|l|}{\textbf{ESPECIFICACIONES TÉCNICAS}} \\ \hline
\multicolumn{2}{|p{0.95\textwidth}|}{
\footnotesize
Especificaciones por definir según imagen adjunta
} \\ \hline
\multicolumn{2}{|l|}{\textbf{PROPÓSITO:} Por definir} \\ \hline
\multicolumn{2}{|l|}{\textbf{IMPACTO:} Por evaluar} \\ \hline
\multicolumn{2}{|l|}{\textbf{OBSERVACIONES:} Ver imagen para detalles} \\ \hline
\end{tabular}
\end{table}

% Rack 3
\begin{table}[H]
\centering
\caption{Ficha técnica -- Rack 3}
\label{tab:rack-3}
\begin{tabular}{|p{0.6\textwidth}|p{0.3\textwidth}|}
\hline
\multicolumn{2}{|l|}{\textbf{DESCRIPCIÓN FÍSICA:} Servidor tipo rack} \\ \hline
\textbf{TIPO DE RECURSO:} Rack & 
\multirow{5}{*}{\includegraphics[width=0.25\textwidth,height=4cm,keepaspectratio]{tablas-images/cp1/racks/rack-3.png}} \\ \cline{1-1}
\textbf{MODELO:} Por definir & \\ \cline{1-1}
\textbf{MARCA:} Por definir & \\ \cline{1-1}
\textbf{CÓDIGO DE INVENTARIO:} Por definir & \\ \cline{1-1}
\textbf{FECHA DE ADQUISICIÓN (APROX.):} & \\ \hline
\multicolumn{2}{|l|}{\textbf{ESPECIFICACIONES TÉCNICAS}} \\ \hline
\multicolumn{2}{|p{0.95\textwidth}|}{
\footnotesize
Especificaciones por definir según imagen adjunta
} \\ \hline
\multicolumn{2}{|l|}{\textbf{PROPÓSITO:} Por definir} \\ \hline
\multicolumn{2}{|l|}{\textbf{IMPACTO:} Por evaluar} \\ \hline
\multicolumn{2}{|l|}{\textbf{OBSERVACIONES:} Ver imagen para detalles} \\ \hline
\end{tabular}
\end{table}

% Rack 4
\begin{table}[H]
\centering
\caption{Ficha técnica -- Rack 4}
\label{tab:rack-4}
\begin{tabular}{|p{0.6\textwidth}|p{0.3\textwidth}|}
\hline
\multicolumn{2}{|l|}{\textbf{DESCRIPCIÓN FÍSICA:} Servidor tipo rack} \\ \hline
\textbf{TIPO DE RECURSO:} Rack & 
\multirow{5}{*}{\includegraphics[width=0.25\textwidth,height=4cm,keepaspectratio]{tablas-images/cp1/racks/rack-4.png}} \\ \cline{1-1}
\textbf{MODELO:} Por definir & \\ \cline{1-1}
\textbf{MARCA:} Por definir & \\ \cline{1-1}
\textbf{CÓDIGO DE INVENTARIO:} Por definir & \\ \cline{1-1}
\textbf{FECHA DE ADQUISICIÓN (APROX.):} & \\ \hline
\multicolumn{2}{|l|}{\textbf{ESPECIFICACIONES TÉCNICAS}} \\ \hline
\multicolumn{2}{|p{0.95\textwidth}|}{
\footnotesize
Especificaciones por definir según imagen adjunta
} \\ \hline
\multicolumn{2}{|l|}{\textbf{PROPÓSITO:} Por definir} \\ \hline
\multicolumn{2}{|l|}{\textbf{IMPACTO:} Por evaluar} \\ \hline
\multicolumn{2}{|l|}{\textbf{OBSERVACIONES:} Ver imagen para detalles} \\ \hline
\end{tabular}
\end{table}

% Rack 5
\begin{table}[H]
\centering
\caption{Ficha técnica -- Rack 5}
\label{tab:rack-5}
\begin{tabular}{|p{0.6\textwidth}|p{0.3\textwidth}|}
\hline
\multicolumn{2}{|l|}{\textbf{DESCRIPCIÓN FÍSICA:} Servidor tipo rack} \\ \hline
\textbf{TIPO DE RECURSO:} Rack & 
\multirow{5}{*}{\includegraphics[width=0.25\textwidth,height=4cm,keepaspectratio]{tablas-images/cp1/racks/rack-5.png}} \\ \cline{1-1}
\textbf{MODELO:} Por definir & \\ \cline{1-1}
\textbf{MARCA:} Por definir & \\ \cline{1-1}
\textbf{CÓDIGO DE INVENTARIO:} Por definir & \\ \cline{1-1}
\textbf{FECHA DE ADQUISICIÓN (APROX.):} & \\ \hline
\multicolumn{2}{|l|}{\textbf{ESPECIFICACIONES TÉCNICAS}} \\ \hline
\multicolumn{2}{|p{0.95\textwidth}|}{
\footnotesize
Especificaciones por definir según imagen adjunta
} \\ \hline
\multicolumn{2}{|l|}{\textbf{PROPÓSITO:} Por definir} \\ \hline
\multicolumn{2}{|l|}{\textbf{IMPACTO:} Por evaluar} \\ \hline
\multicolumn{2}{|l|}{\textbf{OBSERVACIONES:} Ver imagen para detalles} \\ \hline
\end{tabular}
\end{table}

% NAS 1
\begin{table}[H]
\centering
\caption{Ficha técnica -- NAS 1}
\label{tab:nas-1}
\begin{tabular}{|p{0.6\textwidth}|p{0.3\textwidth}|}
\hline
\multicolumn{2}{|l|}{\textbf{DESCRIPCIÓN FÍSICA:} Sistema de almacenamiento conectado en red} \\ \hline
\textbf{TIPO DE RECURSO:} NAS (Network Attached Storage) & 
\multirow{5}{*}{\includegraphics[width=0.25\textwidth,height=4cm,keepaspectratio]{tablas-images/cp1/NAS/nas-1.png}} \\ \cline{1-1}
\textbf{MODELO:} Por definir & \\ \cline{1-1}
\textbf{MARCA:} Por definir & \\ \cline{1-1}
\textbf{CÓDIGO DE INVENTARIO:} Por definir & \\ \cline{1-1}
\textbf{FECHA DE ADQUISICIÓN (APROX.):} & \\ \hline
\multicolumn{2}{|l|}{\textbf{ESPECIFICACIONES TÉCNICAS}} \\ \hline
\multicolumn{2}{|p{0.95\textwidth}|}{
\footnotesize
Especificaciones por definir según imagen adjunta
} \\ \hline
\multicolumn{2}{|l|}{\textbf{PROPÓSITO:} Por definir} \\ \hline
\multicolumn{2}{|l|}{\textbf{IMPACTO:} Por evaluar} \\ \hline
\multicolumn{2}{|l|}{\textbf{OBSERVACIONES:} Ver imagen para detalles} \\ \hline
\end{tabular}
\end{table}

% Firewall 1
\begin{table}[H]
\centering
\caption{Ficha técnica -- Firewall 1}
\label{tab:firewall-1}
\begin{tabular}{|p{0.6\textwidth}|p{0.3\textwidth}|}
\hline
\multicolumn{2}{|l|}{\textbf{DESCRIPCIÓN FÍSICA:} Sistema de seguridad de red} \\ \hline
\textbf{TIPO DE RECURSO:} Firewall & 
\multirow{5}{*}{\includegraphics[width=0.25\textwidth,height=4cm,keepaspectratio]{tablas-images/cp1/firewall/firewall.png}} \\ \cline{1-1}
\textbf{MODELO:} Por definir & \\ \cline{1-1}
\textbf{MARCA:} Por definir & \\ \cline{1-1}
\textbf{CÓDIGO DE INVENTARIO:} Por definir & \\ \cline{1-1}
\textbf{FECHA DE ADQUISICIÓN (APROX.):} & \\ \hline
\multicolumn{2}{|l|}{\textbf{ESPECIFICACIONES TÉCNICAS}} \\ \hline
\multicolumn{2}{|p{0.95\textwidth}|}{
\footnotesize
Especificaciones por definir según imagen adjunta
} \\ \hline
\multicolumn{2}{|l|}{\textbf{PROPÓSITO:} Por definir} \\ \hline
\multicolumn{2}{|l|}{\textbf{IMPACTO:} Por evaluar} \\ \hline
\multicolumn{2}{|l|}{\textbf{OBSERVACIONES:} Ver imagen para detalles} \\ \hline
\end{tabular}
\end{table}

\ChapterImageStar[cap:revisionLiteratura]{Revisión sistemática de la literatura}{./images/fondo.png}\label{cap:revisionLiteratura}

\mbox{}\\
\section{Construcción de la bitácora}

En búsqueda de una base teórica para la elección de una tecnología de virtualización basada en contenedores, 
se realizó una revisión del estado del arte. Esta revisión se completó en diferentes etapas:

\subsection{Planeación}

Esta etapa consistió en establecer el propósito general que se buscaba alcanzar con el \SMS\ (\textit{Systematic Mapping Study}). 
A su vez, definió aspectos como objetivos, preguntas de investigación y métricas ver cuadros~\ref{tab:metricas},~\ref{tab:metas} y~\ref{tab:preguntas} respectivamente. Para ello, se siguió el modelo 
Objetivo-Pregunta-Métrica (\textit{Goal-Question-Metric}, GQM). A continuación, se definen los objetivos del \SMS\ aplicado 
a las tecnologías de virtualización basadas en contenedores en el cuadro.

\subsubsection{Definición de metas para el \SMS}

\begin{table}[H]
\centering
\renewcommand{\arraystretch}{1.2} % Espaciado reducido
\footnotesize % Texto más pequeño
\begin{tabular}{|c|p{13cm}|}  % Columna más ancha (12cm)
\hline
\textbf{Meta} & \textbf{Descripción} \\ \hline
G1 & Identificar trabajos relacionados de \VBC\ en docencia, investigación y extensión. \\ \hline
G2 & Clasificar trabajos relacionados de \VBC\ en dominios de \TI: desarrollo software, pensamiento computacional, computación paralela, análisis datos, IA, redes, infraestructura \TI, HPC, etc. \\ \hline
\end{tabular}
\caption{Definición de metas del SMS}
\label{tab:metas}
\end{table}

\subsubsection{Definición de preguntas de investigación}
\begin{table}[H]
\centering
\renewcommand{\arraystretch}{1.2} % Espaciado reducido
\scriptsize % Texto más pequeño
\begin{tabular}{|c|c|p{6cm}|p{6cm}|} % Columnas más estrechas
\hline
\textbf{Meta} & \textbf{Pregunta} & \textbf{Descripción} & \textbf{Motivación} \\ \hline
G1 & Q1 &
\textit{¿Cuáles trabajos de \VBC\ impactan positivamente en docencia, investigación y extensión?} &
La \VBC\ ofrece transversalidad y reproducibilidad, facilitando transporte de soluciones TI entre dominios. \\ \hline

G2 & Q2 &
\textit{¿Cuáles trabajos de \VBC\ contribuyen en dominios de \TI?} &
Proporcionar base sólida para comprender estado del arte de la \VBC\ sin análisis profundo. \\ \hline
\end{tabular}
\caption{Definición de preguntas de investigación del SMS}
\label{tab:preguntas}
\end{table}

\subsubsection{Definición de métricas}

\begin{table}[H]
\centering
\renewcommand{\arraystretch}{1.2} % Menor espaciado entre filas
\footnotesize % Texto más pequeño
\begin{tabular}{|c|p{9cm}|} % Columna de descripción más estrecha
\hline
\textbf{Métrica} & \textbf{Descripción} \\ \hline
M1 & Cantidad de trabajos identificados en cada dominio de \TI. \\ \hline
M2 & Cantidad de trabajos incluidos en educación. \\ \hline
M3 & Cantidad de trabajos incluidos en investigación. \\ \hline
M4 & Cantidad de trabajos incluidos en extensión. \\ \hline
\end{tabular}
\caption{Definición de métricas del SMS}
\label{tab:metricas}
\end{table}



\section{Búsqueda de estudios}

% Asegúrate de tener cargado el paquete enumitem en el preámbulo:
% \usepackage{enumitem}

Esta etapa comprendió las siguientes secciones: 
\begin{enumerate}[noitemsep, topsep=0pt, partopsep=0pt]
  \item Estrategia de búsqueda, ya sea independiente o combinada.
  \item Identificación general de estudios.
  \item Revisión de estudios.
  \item Selección de estudios para incluir en el SMS.\@
\end{enumerate}

\subsection{Estrategia de búsqueda}

Este trabajo combinó las estrategias de búsqueda en bases de datos y búsqueda en bola de nieve. 

\subsection{Búsqueda en bases de datos}\label{subsec:busquedaBasesDatos}
Se seleccionaron las siguientes bases de datos para este propósito: ACM, IEEE Xplore, Springer, Taylor \& Francis y Science Direct.

\subsubsection{Identificación de estudios mediante búsqueda en bases de datos}\label{subsubsec:identificacionEstudios}
En esta etapa del proceso fue necesario establecer las palabras clave que serían útiles en las cadenas de búsqueda para cada una de las bases de datos seleccionadas. 
Los términos consideran los elementos identificados en la etapa de planificación, para lo cual también se utilizó el modelo PICOC ( \textit{Population}, \textit{Intervention}, \textit{Comparator}, \textit{Outcome}, and \textit{Context} ) como guía metodológica. El modelo PICOC se describe en el cuadro~\ref{tab:picoc-model} y las palabras clave resultantes del modelo se desarrollan en el cuadro~\ref{tab:keywords-picoc}.

\begin{table}[H]
\centering
\renewcommand{\arraystretch}{1.2} % Espaciado reducido
\footnotesize % Texto más pequeño
\begin{tabularx}{\textwidth}{|p{0.18\textwidth}|X|} % Columna izquierda más estrecha
\hline
\textbf{Componente} & \textbf{Descripción} \\ \hline

Población & Trabajos sobre \VBC\ aplicadas en \TI, con énfasis en educación, investigación y extensión. \\ \hline

Intervención & Identificación y clasificación de trabajos \VBC\ en dominios de \TI. \\ \hline

Comparación & 
\textbf{1.} Comparación de proyectos \VBC\ por tasa de éxito en cada dominio \TI.\@        
\textbf{2.} Análisis de impacto de \VBC\ vs. otras soluciones en docencia, investigación y extensión. \\ \hline
Salida & Estructura de clasificación de trabajos \VBC\ que impactan en docencia, investigación y extensión. \\ \hline
Contexto & Docencia, investigación y extensión con apropiación de \VBC\ en \TI. \\ \hline
\end{tabularx}
\caption{Modelo PICOC}\label{tab:picoc-model}
\end{table}

\begin{table}[H]
\centering
\scriptsize
\setlength{\tabcolsep}{3pt}
\renewcommand{\arraystretch}{1.1}
\begin{tabular}{|p{3cm}|p{2.5cm}|p{2.5cm}|p{3cm}|p{3cm}|}
\hline
\textbf{Población} & \textbf{Intervención} & \textbf{Comparación} & \textbf{Salida} & \textbf{Contexto} \\
\hline
\VBC\ \newline Dominios de \TI\ Educación Investigación Extensión & Identificación \newline Clasificación & Tasa de éxito \newline Evidencia de uso & Clasificación de trabajos \newline relacionados con \VBC\ en cada dominio de \TI\ & Docencia Investigación Extensión \\
\hline
\end{tabular}
\caption{Palabras clave identificadas usando el modelo PICOC}
\label{tab:keywords-picoc}
\end{table}

Las palabras clave identificadas en el cuadro~\ref{tab:keywords-picoc} se complementaron con sinónimos y términos relacionados, los cuales se presentan en el cuadro~\ref{tab:keywords}. Estas keywords se utilizaron para construir las cadenas de búsqueda en cada base de datos.
\begin{table}[H]
\centering
\scriptsize
\setlength{\tabcolsep}{4pt}
\begin{tabular}{|p{5cm}|p{9.5cm}|}
\hline
\textbf{Palabras clave} & \textbf{Sinónimos} \\
\hline
Container-based virtualization & Application virtualization, Docker, Lightweight Virtualization \\
\hline
Education & Education System, Education Development, Higher Education \\
\hline
Research & Research Group, Research Proposal \\
\hline
Outreach & \IT\ Services, Technology Infrastructure, Cloud Computing \\
\hline
\end{tabular}
\caption{Palabras clave para la búsqueda en base de datos}
\label{tab:keywords}
\end{table}

Con el objetivo de filtrar los resultados y enfocarse en estudios relevantes, se definieron criterios de inclusión y exclusión, los cuales se presentan en el cuadro~\ref{tab:criterios-inclusion-exclusion}. Estos criterios ayudaron a seleccionar artículos que se alinean con los objetivos del \SMS\ y a descartar aquellos que no aportan valor al análisis.
\begin{table}[H]
\centering
\scriptsize
\setlength{\tabcolsep}{4pt}
\renewcommand{\arraystretch}{1.2}
\begin{tabular}{|p{4cm}|p{5cm}|p{5.5cm}|}
\hline
\textbf{Categoría} & \textbf{Inclusión} & \textbf{Exclusión} \\
\hline
Campos & Resumen & --- \\
\hline
Tipo de publicación & Artículos de revistas y conferencias & Tesis y capítulos de libros \\
\hline
Área/Disciplina & Management, \CS\, IT Management, engineering & Áreas no relacionadas con virtualización, \CS\ y \IT\ Management \\
\hline
Período & 2022 a 2024 & Antes de 2022 \\
\hline
Idioma & Inglés & --- \\
\hline
\end{tabular}
\caption{Criterios de Inclusión/Exclusión}\label{tab:criterios-inclusion-exclusion}
\end{table}

\subsubsection{Búsqueda en bases de datos}\label{par:busquedaBasesDatos}
Las cadenas de búsqueda específicas para cada base de datos se encuentran en la sección~\ref{sec:cadenas-busqueda} del apéndice.

\subsubsection{Métricas de la búsqueda sin criterios de inclusión/exclusión}\label{subsubsec:resumenBusqueda}
La Tabla~\ref{tab:bases-sin-criterio} presenta el número de publicaciones identificadas en las principales bases de datos consultadas durante la revisión inicial de literatura, antes de aplicar los criterios de inclusión y exclusión. En total se recuperaron 6.530 registros, distribuidos de la siguiente manera: ACM (189), IEEE (426), Springer (4.562), Science Direct (353) y Taylor \& Francis (1.000). Estos resultados se evidencian en el apéndice~\ref{sec:busqueda-sin-criterios}.

\begin{table}[H]
    \centering
    \includegraphics[width=\textwidth] {tablas-images/cp2/resumen-sin-criterios.png}
    \caption{Resumen de la búsqueda en bases de datos sin criterios de inclusión/exclusión}\label{tab:tabla-resumen}
\end{table}
\begin{figure}[H]
    \centering
    \includegraphics[width=\textwidth] {tablas-images/cp2/bases-sin-criterio.png}
    \caption{Resumen de la búsqueda en bases de datos sin criterios de inclusión/exclusión}\label{fig:tabla-resumen-busqueda}
\end{figure}

\subsubsection{Métricas de la búsqueda con criterios de inclusión/exclusión}\label{subsec:resumenBusquedaCriterios}
Esta búsqueda se realizó considerando los criterios de exclusión e inclusión definidos previamente. Las cadena de búsqueda son exactamente iguales que antes, este punto se diferencia por la aplicación de filtros. Para ver las capturas de pantalla veáse el apéndice~\ref{sec:busqueda-con-criterios}.
La Tabla~\ref{tab:bases-sin-criterio} muestra los resultados obtenidos tras aplicar los criterios de inclusión y exclusión previamente definidos. A diferencia de la búsqueda inicial, en esta fase se utilizaron filtros específicos que redujeron significativamente la cantidad de publicaciones relevantes. En total se identificaron 976 documentos, distribuidos en las bases de datos de la siguiente manera: ACM (48), IEEE (134), Springer (592), Science Direct (46) y Taylor \& Francis (156).

\begin{table}[H]
    \centering
    \includegraphics[width=\textwidth] {tablas-images/cp2/resumen-con-criterios.png}
    \caption{Resumen de la búsqueda en bases de datos con criterios de inclusión/exclusión}\label{tab:tabla-resumen}
\end{table}
\begin{figure}[H]
    \centering
    \includegraphics[scale=0.9] {tablas-images/cp2/bases-con-criterio.png}
    \caption{Diagrama de búsqueda en bases de datos}\label{fig:tabla-resumen-busqueda-con-criterio}
\end{figure}

\section{Eliminación de duplicados}\label{sec:eliminacionDuplicados}
La eliminación de duplicados se realizó haciendo uso de la herramienta de gestión de referencias Mendeley. Luego de obtener los artículos se agregaron a Mendeley y esta herramienta se encargó de eliminar duplicados. En este punto se eliminaron 274 artículos duplicados.

\section{Priorización de estudios}\label{sec:priorizacionEstudios}

Luego de la selección inicial de los artículos, se procedió a revisar el \textit{title}, \textit{abstract} y \textit{keywords} de cada uno. Como resultado de esta revisión, se generaron métricas de calidad para cada artículo, con el fin de priorizar aquellos más relevantes para la investigación. Las métricas utilizadas fueron las siguientes:

\begin{itemize}
    \item \textbf{SCI} (Science Citation Index)
    \item \textbf{CVI} (Core Value Index)
    \item \textbf{IRRQ} (Index Relation Research Question)
\end{itemize}

Este proceso inició con un total de 771 artículos, los cuales fueron evaluados según su alineamiento con los objetivos de la investigación. La evaluación temática permitió identificar un total de 110 artículos con una relación directa con el enfoque planteado.

\section{Estrategia de búsqueda usando bola de nieve}\label{sec:bolaDeNieve}

En esta etapa, se seleccionó el primer cuartil según el índice \textbf{IRRQ}, lo que resultó en un total de 24 artículos. Adicionalmente, se incluyeron dos artículos por criterio de inclusión directa, estableciendo así una línea base de \textbf{26 artículos}. 

Sobre esta base, se aplicó la estrategia de \textit{bola de nieve} en ambas direcciones: hacia adelante y hacia atrás. Como resultado, se obtuvieron \textbf{87 artículos} mediante la técnica hacia atrás y \textbf{495 artículos} mediante la técnica hacia adelante. 

Esto definió un nuevo conjunto de artículos para un proceso de selección adicional (\textit{screening}). En esta fase, se eliminaron \textbf{14 duplicados} y \textbf{452 artículos} fueron descartados por no estar alineados con la investigación. 

Finalmente, se obtuvo un total de \textbf{116 artículos} mediante esta estrategia de búsqueda ampliada.

\section{Diagrama de búsqueda}\label{sec:diagramaBusqueda}

\subsection{Usando cadenas de búsqueda}
En el diagrama~\ref{tab:tabla-diagrama-cadena-busqueda} se puede apreciar la estrategia de búsqueda de artículos por medio de base de datos, aplicando las cadenas de búsqueda, se consolidaron los resultados de distintas bases de datos para obtener un total de 6530 resultados, posteriormente y aplicando criterios de exclusión se redujo esta cantidad a menos de 1000 resultados. Adicional a los criterios de exclusión, también se hizo eliminación de artículos duplicados, 205 por parte del ~\textit{Reference Manager} (Mendeley), y 69 por parte del ~\textit{SMS-Builder} para un total de 274 artículos removidos. Finalmente, se realiza la etapa de screening, donde se leen las secciones claves de los artículos, como ~\textit{abstract}, ~\textit{keywords} e introducción, a través de esto se pudo descargar 671 artículos que no eran pertinentes para el estudio.
\begin{table}[H]
    \centering
    \includegraphics[width=\textwidth] {tablas-images/cp2/diagrama-cadena-busqueda.png}
    \caption{Diagrama de la cadena de búsqueda}\label{tab:tabla-diagrama-cadena-busqueda}
\end{table}\label{img:busqueda-bd}

\subsection{Usando bola de nieve}
Como segunda estrategia de búsqueda dentro del SMS, se aplicó la técnica de ~\textit{Snowball}, la cual consiste en extraer artículos adicionales a partir de las referencias citadas en los estudios obtenidos en la estrategia anterior y de los estudios que citan a estos. De los 110 estudios del paso anterior, se seleccionan aquellos que tengan un SCI más relevante y se agrega uno por inclusión directa, con esto se obtiene un total de 25 artículos de línea base. Aplicando snowball hacia adelante (artículos referenciados) se obtienen 87 nuevos estudios, aplicando snowball hacia atrás (artículos que referencian el artículo base) se obtienen 495, para un total de 582. Finalmente, aplicando criterios de exclusión de remoción de duplicados y aplicando la técnica de screening, se obtiene un resultado de 116 artículos incluidos por bola de nieve. Todo este proceso se puede apreciar en la gráfica~\ref{tab:tabla-diagrama-bola-nieve-busqueda}.
\begin{table}[H]
    \centering
    \includegraphics[width=\textwidth] {tablas-images/cp2/diagrama-bola-nieve-busqueda.png}
    \caption{Diagrama de la búsqueda en bola de nieve}\label{tab:tabla-diagrama-bola-nieve-busqueda}
\end{table}

\section{Identificación de estudios}

\subsection{Artículos por año y métricas}
A continuación se presentan las gráficas que resumen los resultados de las estrategias anteriores. En la figura~\ref{fig:diagrama-articulos-ano-metrica} se pueden visualizar las métricas de calidad, separadas por los últimos 3 años y mostrando el promedio en cada métrica de los artículos publicados en esos años. En la figura~\ref{fig:tipos-articulos} se muestra el conteo de artículos por su tipo específico, el cual puede ser uno de 3 opciones: 'Revista', 'Conferencia' o 'Genérico'. Vemos que la mayoría de artículos provienen de revistas. En la figura~\ref{fig:estrategia-busqueda-articulos} se detalla la cantidad de artículos que se extrajo de cada estrategia. Finalmente, en la figura~\ref{fig:diagrama-red-articulos} se puede apreciar un diagrama de red, que segrega por colores los tópicos más relacionados entre sí
\noindent
En la figura~\ref{fig:diagrama-articulos-ano-metrica} se pueden visualizar las métricas de calidad, separadas por los últimos 3 años y mostrando el promedio en cada métrica de los artículos publicados en esos años, así como un apartado donde se calcula la sumatoria de estas métricas por cada año.
\begin{figure}[H]
    \centering
    \includegraphics[scale=0.7]{tablas-images/cp2/diagrama-articulos-ano-metrica.png}
    \caption{Artículos por métricas y año}\label{fig:diagrama-articulos-ano-metrica}
\end{figure}
\noindent

\subsection{Tipo de artículos}
En la figura~\ref{fig:tipos-articulos} se muestra el conteo de artículos por su tipo específico, el cual puede ser uno de 3 opciones: ``Revista'', ``Conferencia'' o ``Genérico''. Vemos que la mayoría de artículos provienen de revistas.
\begin{figure}[H]
    \centering
    \includegraphics[scale=0.4]{tablas-images/cp2/tipos-articulos.png}
    \caption{Artículos por tipo}\label{fig:tipos-articulos}
\end{figure}
\noindent
\subsection{Estrategia de búsqueda de artículos}
En la figura~\ref{fig:estrategia-busqueda-articulos} se detalla la cantidad de artículos que se extrajeron de cada estrategia. Se puede observar que la estrategia que generó más artículos fue la técnica de~\textit{Snowball}. Además, es importante destacar que la inclusión directa se está contabilizando como una estrategia, ya que se consideraron artículos que no fueron encontrados en las bases de datos, pero que se conocían de antemano y cumplían con los criterios de inclusión, debido a que no fue hallado ni en las bases de datos ni en la técnica de~\textit{Snowball}.
\begin{figure}[H]
    \centering
    \includegraphics[scale=0.8]{tablas-images/cp2/estrategia-busqueda-articulos.png}
    \caption{Estrategia de búsqueda de artículos}\label{fig:estrategia-busqueda-articulos}
\end{figure}
\noindent
\subsection{Diagrama de red de artículos}
Finalmente, en la figura~\ref{fig:diagrama-red-articulos} se puede apreciar un diagrama de red, que segrega por colores los tópicos más relacionados entre sí, vemos 4 grandes grupos: IA, cloud computing, virtualización, desarrollo de software. Además permite observar qué tópicos están más relacionados entre sí, por ejemplo, se puede ver que los microservicios están muy relacionados con la computación en la nube y con Docker. La herramienta utilizada para generar este diagrama fue \textit{VOSviewer}~\citep{vosviewer_website} que permite crear mapas basados en la co-ocurrencia de términos en los artículos, y que es comúnmente utilizada para análisis bibliométricos. Basta con importar los artículos en formato \textit{RIS} o \textit{BibTeX} y la herramienta genera el diagrama automáticamente.
\begin{figure}[H]
    \centering
    \includegraphics[scale=0.8]{tablas-images/cp2/diagrama-red-busqueda.png}
    \caption{Diagrama de red de los artículos}\label{fig:diagrama-red-articulos}
\end{figure}

\section{Información de la herramienta}

\noindent
La herramienta utilizada para este proceso de revisión de la literatura fue \textbf{SMS-BUILDER}, la cual se encuentra disponible en \textit{Docker Hub}. El estudio realizado puede consultarse en el siguiente enlace:

\begin{center}
\href{https://sms-vbc.iti.grid.uniquindio.edu.co/}{\texttt{https://sms-vbc.iti.grid.uniquindio.edu.co/}}
\end{center}

\noindent
Adicionalmente, se implementaron procesos de respaldo como medida de seguridad. Estos \textit{backups} fueron almacenados en ubicaciones diferentes, siguiendo la estrategia de respaldo \textbf{3--2--1}.

\section{Nichos de mercado}
\noindent
La identificación de los nichos de mercado permite comprender el posicionamiento y la orientación estratégica de cada tecnología de \VBC. En este apartado se analizan las audiencias objetivo y los contextos en los que las diferentes tecnologías encuentran mayor adopción y aplicabilidad. Cada una de estas herramientas responde a necesidades particulares dentro del ecosistema de la computación y la orquestación de contenedores, lo que facilita distinguir sus ventajas competitivas y su relevancia en función de los requerimientos de los proveedores de servicios, organizaciones y desarrolladores.
\subsection{Containerd}
\noindent
Containerd está dirigido a proveedores de servicios en la nube y plataformas de orquestación como Kubernetes, donde se requiere una solución de gestión de contenedores ligera y compatible con OCI \citep{Vano2023}. Su arquitectura modular lo convierte en una opción preferida para grandes infraestructuras \citep{Zhou2021}.

\subsection{CRI-O}
\noindent
CRI-O está diseñado específicamente para su integración con Kubernetes, sirviendo como un motor de contenedores ligero y compatible con OCI para esta plataforma \citep{CNCF2019}. Es una solución ideal para proveedores de servicios en la nube y organizaciones que utilizan Kubernetes como su plataforma de orquestación principal \citep{151962df5f7e4b9faba0629540c11439}.

\subsection{Docker}
\noindent
Docker se posiciona principalmente en el nicho de mercado de desarrolladores de software, empresas tecnológicas y proveedores de servicios en la nube que buscan una solución para la creación, implementación y gestión de aplicaciones en contenedores \citep{Hill2025}. Su capacidad de automatizar despliegues y la portabilidad entre entornos lo convierte en una opción ideal para DevOps y desarrollo ágil \citep{Mag2025}.

\subsection{Firecracker}
\noindent
Firecracker está orientado a proveedores de servicios en la nube y plataformas de cómputo en la nube que requieren micro VMs eficientes y seguras \citep{Jain}. Es una solución ideal para plataformas \textit{serverless} y entornos multi-tenant \citep{246288}.

\subsection{Google gVisor}
\noindent
Google gVisor está dirigido a proveedores de servicios en la nube y organizaciones que priorizan la seguridad en sus entornos de contenedores \citep{LopezFalcon2024}. Su arquitectura de \textit{sandbox} proporciona un aislamiento fuerte, lo que lo convierte en una opción atractiva para aplicaciones sensibles \citep{gvisor2025}.

\subsection{Hyper-V Containers}
\noindent
Hyper-V Containers están orientados a empresas que utilizan infraestructuras basadas en Windows, ofreciendo una solución de contenedorización segura y eficiente para aplicaciones basadas en Windows \citep{Smith2016}. Su integración con el ecosistema de Microsoft lo hace ideal para empresas con infraestructuras híbridas \citep{Clark2024}.

\subsection{Kata Containers}
\noindent
Kata Containers se centra en entornos donde se requiere un alto nivel de seguridad y aislamiento, como proveedores de servicios en la nube y empresas que manejan información confidencial \citep{Viktorsson2020}. Su capacidad para combinar la eficiencia de los contenedores con el aislamiento de máquinas virtuales es su principal ventaja \citep{10.1145/1272996.1273025}.

\subsection{Linux VServer}
\noindent
Linux VServer está orientado a administradores de sistemas y proveedores de servicios que requieren una solución de virtualización ligera basada en contenedores para la administración de servidores seguros y eficientes \citep{10.1145/1272996.1273025}. Es una opción adecuada para entornos de servidor dedicados y alojamientos compartidos \citep{LinuxVirt2017}.

\subsection{LXC (Linux Containers)}
\noindent
LXC es popular en entornos de virtualización ligera y servidores, donde se requiere un control granular sobre los entornos de contenedores \citep{Silva2024}. Su uso está orientado a proveedores de alojamiento web, desarrolladores de software y administradores de sistemas que necesitan un control preciso del entorno del sistema operativo \citep{Simon2023}.

\subsection{LXD}
\noindent
LXD se enfoca en nichos de mercado que requieren entornos de virtualización basados en contenedores que imiten máquinas virtuales, como proveedores de servicios en la nube, plataformas de pruebas y entornos de desarrollo \citep{Silva2024}. Su capacidad para ofrecer entornos de sistema completo lo hace ideal para desarrolladores y administradores de sistemas \citep{Kaiser2022}.

\subsection{OpenVZ}
\noindent
OpenVZ se centra en proveedores de alojamiento web y servicios VPS, donde se requiere una solución de virtualización ligera basada en contenedores que permita un control granular sobre los recursos del sistema y la administración de múltiples instancias \citep{OpenVZ2015}.

\subsection{Podman}
\noindent
Podman está orientado a entornos empresariales y desarrolladores que requieren una solución de contenerización sin \textit{daemon}, compatible con OCI y con enfoque en la seguridad \citep{Surendhar2024}. Su naturaleza sin \textit{daemon} y su capacidad para ejecutar contenedores de forma aislada permiten su adopción en entornos donde la seguridad y la conformidad son prioridades \citep{Trevor2022}.

\subsection{Rkt}
\noindent
Rkt fue diseñado para satisfacer las necesidades de proveedores de servicios en la nube y organizaciones que buscan una alternativa a Docker con un enfoque en la seguridad y compatibilidad OCI \citep{Lingayat2018}. Aunque su desarrollo ha sido discontinuado, sigue siendo relevante en entornos donde la compatibilidad y la seguridad son críticas \citep{Watada2019}.

\subsection{runC}
\noindent
runC está orientado a proveedores de servicios en la nube, plataformas de orquestación como Kubernetes y desarrolladores de software que buscan una solución de contenedorización ligera y compatible con OCI \citep{Perez2005}. Su adopción en proyectos de gran escala se debe a su eficiencia y cumplimiento de estándares de contenedores \citep{151962df5f7e4b9faba0629540c11439}.

\subsection{Sarus}
\noindent
Sarus está dirigido a entornos de HPC y computación científica, donde los usuarios necesitan ejecutar contenedores de forma segura en sistemas de alto rendimiento \citep{Sarus2021}. Su compatibilidad con estándares de contenedores y su enfoque en la seguridad lo hacen ideal para centros de investigación y universidades \citep{B2020}.

\subsection{Singularity}
\noindent
Singularity se centra en entornos de computación científica y HPC, donde se requiere portabilidad de aplicaciones sin necesidad de privilegios de root \citep{10.1145/3332186.3332192}. Es ampliamente adoptado en universidades, centros de investigación y laboratorios que ejecutan aplicaciones de alto rendimiento \citep{Kurtzer2017}.

\subsection{Udocker}
\noindent
Udocker se especializa en nichos de mercado académicos y de investigación, donde los usuarios necesitan ejecutar contenedores sin privilegios en sistemas que no permiten la instalación de software de nivel de sistema \citep{Campos2017}. Su facilidad para funcionar en entornos HPC (Computación de Alto Rendimiento) sin requerir permisos de root lo hace adecuado para instituciones de investigación \citep{Gomes2018}.

\subsection{Wasm (WebAssembly)}
\noindent
Wasm se centra en el nicho de desarrollo web y aplicaciones de alto rendimiento en el navegador \citep{Haas2017}. Su capacidad para ejecutar código en múltiples plataformas, incluidas aplicaciones de escritorio y móviles, lo convierte en una opción atractiva para empresas de desarrollo de software que buscan optimización multiplataforma \citep{Jangda2019}.
\clearpage
\section{Comparativa de licencias}
\noindent
La tabla~\ref{tab:licencias-vbc} presenta un panorama comparativo de las tecnologías de \VBC\ presentadas antes, detallando aspectos clave como el tipo de licencia, los términos de uso y los costos asociados. Este análisis permite identificar no solo las diferencias en cuanto a modelos de distribución y sostenibilidad económica, sino también las implicaciones legales y técnicas que pueden influir en la selección de una u otra herramienta en contextos de investigación o implementación empresarial. Asimismo, se evidencia la coexistencia de soluciones de código abierto, ampliamente utilizadas en entornos académicos y científicos, junto con alternativas propietarias que implican costos más elevados, lo cual resalta la necesidad de evaluar cuidadosamente la relación entre funcionalidad, libertad de uso y viabilidad financiera.
\input{tablas-images/cp3/licencias.tex}

\section{Interfaz de uso}
\noindent
En la tabla~\ref{tab:interfaz-vbc} se describe la interfaz de uso por cada tecnología. Como se puede apreciar, la gran mayoría de tecnologías se utilizan a través de una \CLI. Esto, en muchos casos, puede implicar un aumento en la curva de aprendizaje, pero también facilita la gestión y automatización de las tecnologías una vez que se ha comprendido el uso de su interfaz.
\begin{table}[H]
    \centering
    \includegraphics[width=\textwidth] {tablas-images/cp3/interfaz-uso.png}
    \caption{Interfaz de uso de cada VBC}\label{tab:tabla-interfaz-uso}
\end{table}

\section{Integración con la nube}
\noindent
En el cuadro~\ref{tab:integracion-cloud-vbc} se describe la integración a las distintas plataformas \textit{cloud} que tiene cada tecnología. Se evidencia que muchas tecnologías tienen integración directa con 3 de los proveedores cloud más famosos: \AWS, \GCP\ y Azure. Algunas tecnologías, por otro lado, solo soportan implementación de nubes privadas, como LXD.\@
\begin{table}[H]
\centering
\scriptsize
\setlength{\tabcolsep}{3pt}
\renewcommand{\arraystretch}{1.1}
\begin{tabularx}{\textwidth}{|p{0.2\textwidth}|X|}
\hline
\textbf{Tecnología} & \textbf{Integración con Proveedores de Cloud} \\
\hline
Containerd & Integración fuerte con Kubernetes, que a su vez se integra con proveedores de nube como \AWS\ (EKS), \GCP\ (GKE), Azure (AKS) y otros. \\
\hline
CRI-O & Integración directa con Kubernetes, lo que le permite ser utilizado en proveedores de nube como \AWS\ (EKS), \GCP\ (GKE), Azure (AKS), y otros servicios de orquestación de contenedores. \\
\hline
Docker & Integración con \AWS\ (ECR, ECS), \GCP\ (GCR, GKE), Azure (ACR, AKS), y otros proveedores a través de herramientas como Docker Compose, Docker Swarm y Docker Desktop. \\
\hline
Google gVisor & Integración con Google Cloud, especialmente en Google Kubernetes Engine (GKE), para agregar una capa adicional de seguridad a los contenedores. \\
\hline
Hyper-V containers & Integración exclusiva con Microsoft Azure, especialmente con Azure Kubernetes Service (AKS) y otras soluciones basadas en Hyper-V. \\
\hline
Kata Containers & Soporta proveedores de nube pública como \AWS\, Google Cloud, y Azure a través de Kubernetes, proporcionando aislamiento similar a máquinas virtuales en entornos de contenedores. \\
\hline
Linux VServer & Utilizado principalmente en proveedores de hosting dedicados y servidores privados, sin integración directa con proveedores de nube pública como \AWS\, \GCP\ o Azure. \\
\hline
LXC & Se puede integrar en plataformas de nube privada y algunas soluciones híbridas. Se usa en servidores de nube como OpenStack, pero no tiene una integración directa con plataformas públicas principales. \\
\hline
LXD & Puede integrarse con plataformas de nube privada, como OpenStack, para ofrecer contenedores ligeros que emulan máquinas virtuales. No tiene integración directa con los proveedores de nube pública principales, pero puede ser utilizado en soluciones personalizadas. \\
\hline
OpenVZ & Tradicionalmente usado en proveedores de hosting como OVH, aunque su uso ha disminuido frente a soluciones más modernas. La integración con nubes públicas es limitada y generalmente personalizada. \\
\hline
Podman & Compatible con \AWS\ (ECR), \GCP\ (GCR), Azure (ACR), aunque su integración con orquestadores como Kubernetes es más reciente y menos prevalente que Docker. \\
\hline
Rkt & Aunque estaba integrado con Kubernetes y otras plataformas, su descontinuación limita la integración con proveedores de nube. En el pasado, soportaba plataformas como \AWS\ y \GCP\ (Google Cloud). \\
\hline
runC & Integración con Kubernetes, que se usa ampliamente en proveedores de nube como \AWS\ (EKS), \GCP\ (GKE), y Azure (AKS) para la orquestación de contenedores. \\
\hline
Singularity & Utilizado principalmente en entornos de computación científica y HPC. Puede integrarse con proveedores como \AWS\ (HPC, Batch) y \GCP\ (Compute Engine) para tareas específica de alto rendimiento. \\
\hline
Udocker & Generalmente se usa en entornos sin privilegios de root y en plataformas como \HPC. No tiene una integración directa con proveedores de nube a gran escala. \\
\hline
Wasm (WebAssembly) & Integración principalmente con servicios de computación en la nube como \AWS\ Lambda, \GCP\ (Cloud Functions), y Azure Functions, ya que permite la ejecución de código en la nube sin dependencia del sistema operativo subyacente. \\
\hline
\end{tabularx}
\caption{Integración cloud de cada VBC}\label{tab:integracion-cloud-vbc}
\end{table}

\clearpage
\section{Cuadrante Gartner}
\noindent
La tabla~\ref{tab:cuadrante-gartner} sintetiza la clasificación de tecnologías de \VBC\ en un cuadrante de Gartner adaptado, considerando los ejes de visión (X) y capacidad de ejecución (Y). Esta representación permite identificar el posicionamiento relativo de cada tecnología en el mercado, agrupándolas en cuatro cuadrantes estratégicos: líderes, retadores, visionarios y jugadores de nicho. Con ello se facilita el análisis comparativo entre soluciones consolidadas, alternativas emergentes con alto potencial, y herramientas especializadas con menor alcance, aportando una perspectiva integral para la toma de decisiones en proyectos de investigación y aplicaciones prácticas.
Los calculos tanto de visión como de ejecución se realizaron a partir de las métricas obtenidas en el análisis \DAR\ realizado. Se tuvieron en cuenta las características inherentes a cada tecnología, como su madurez, adopción en la industria, innovación, soporte comunitario y facilidad de integración, posteriormente se realizó un promedio ponderado de cada una de estas características para obtener un valor numérico que representara la visión y la ejecución de cada tecnología. Este promedio también se aproximó al entero superior más cercano para facilitar la interpretación en el cuadrante.
\begin{table}[H]
    \centering
    \includegraphics[width=\textwidth] {tablas-images/cp3/medicion-gartner.png}
    \caption{Tabla de medición para el cuadrante gartner}\label{tab:tabla-medicion-gartner}
\end{table}
\noindent
A partir de estas valoraciones se definieron los cuadrantes, estableciendo umbrales numéricos en cada eje para su clasificación. De este modo, se consideran Líderes aquellas tecnologías con Visión $\geq$ 7 y Ejecución $\geq$ 7, caracterizadas por estar consolidadas, combinar madurez técnica con una fuerte proyección estratégica, como es el caso de Docker y Containerd. Los Retadores corresponden a soluciones con Visión $<$ 7 y Ejecución $\geq$ 7, es decir, tecnologías robustas y estables en su uso, pero con menor capacidad de innovación o proyección a futuro, como Podman y CRI-O. En el cuadrante de Visionarios se ubican aquellas con Visión $\geq$ 7 y Ejecución $<$ 7, las cuales presentan un gran potencial disruptivo, aunque aún con limitaciones de adopción masiva o falta de consolidación, como Wasm, gVisor o Firecracker. Finalmente, los Jugadores de Nicho, definidos por Visión $<$ 7 y Ejecución $<$ 7, incluyen tecnologías con baja adopción, soporte reducido o enfoques muy específicos que restringen su aplicabilidad general, como LXC, Udocker y OpenVZ.
\begin{figure}[H]
    \centering
    \includegraphics[scale=0.1] {tablas-images/cp3/cuadrante-gartner.png}
    \caption{Cuadrante de Gartner de cada VBC}\label{fig:tabla-cuadrante-gartner}
\end{figure}

\section{Entornos de ejecución}
\noindent
A continuación, en la tabla~\ref{tab:entornos-ejecucion-vbc} se describen los ambientes de ejecución de diversas tecnologías de \VBC, resaltando los sistemas operativos compatibles y los contextos de uso más comunes. Este análisis evidencia la versatilidad de las herramientas, que abarcan desde entornos de desarrollo y producción en sistemas tradicionales (Linux, Windows y macOS), hasta aplicaciones en computación de alto rendimiento (\HPC), plataformas en la nube y navegadores web. Asimismo, se observa la coexistencia de soluciones ampliamente adoptadas, como Docker o Containerd, junto con propuestas especializadas en seguridad, portabilidad o rendimiento, lo que permite identificar sus ventajas diferenciales en función de los requerimientos técnicos y organizacionales.
\begin{table}[H]
    \centering
    \includegraphics[width=\textwidth] {tablas-images/cp3/entorno-ejecucion.png}
    \caption{Entornos de ejecución de cada VBC}\label{tab:tabla-entorno-ejecucion}
\end{table}

\section{Evaluación DOFA}
\noindent
La tabla~\ref{tab:matriz-dofa} presenta un análisis integral de las tecnologías de virtualización basadas en contenedores (\VBC), considerando tanto los factores internos —fortalezas y debilidades— como los externos —oportunidades y amenazas— que influyen en su adopción y desarrollo. Este enfoque permite identificar los beneficios clave de la contenerización, como la portabilidad, la eficiencia en el uso de recursos y la escalabilidad, al tiempo que visibiliza los retos asociados a la seguridad, la gestión de infraestructuras complejas y la dependencia de plataformas propietarias.
 
\begin{table}[H]
    \centering
    \includegraphics[width=\textwidth] {tablas-images/cp3/matriz-dofa.png}
    \caption{Tabla de matriz DOFA para el cuadrante gartner}\label{tab:tabla-matriz-dofa}
\end{table}
\clearpage
\section{Documentación y soporte}
\noindent
La tabla~\ref{tab:documentacion-tecnologias} presenta un compendio de enlaces a la documentación oficial de distintas tecnologías de \VBC\, lo cual constituye un recurso para investigadores, desarrolladores y administradores de sistemas. La disponibilidad de documentación confiable y actualizada resulta determinante para una implementación adecuada, resolver problemas técnicos y aprovechar al máximo las funcionalidades de cada herramienta. De esta forma, el cuadro permite acceder de manera centralizada a las fuentes oficiales de soporte, facilitando la consulta y el aprendizaje en entornos académicos y profesionales
\begin{table}[H]
\centering
\scriptsize
\setlength{\tabcolsep}{3pt}
\renewcommand{\arraystretch}{1.1}
\begin{tabular}{|>{\centering\arraybackslash}p{0.5cm}|>{\raggedright\arraybackslash}p{3.5cm}|>{\centering\arraybackslash}p{2.5cm}|}
\hline
\multicolumn{2}{|c|}{\textbf{Tecnología}} & \textbf{Documentación} \\
\hline
1 & Docker & \href{https://docs.docker.com/}{link} \\
\hline
2 & Podman & \href{https://podman.io/docs}{link} \\
\hline
3 & Udocker & \href{https://github.com/indigo-dc/udocker}{link} \\
\hline
4 & Wasm & \href{https://webassembly.org/docs/faq/}{link} \\
\hline
5 & LXC & \href{https://linuxcontainers.org/incus/docs/main/}{link} \\
\hline
6 & Containerd & \href{https://containerd.io/docs/}{link} \\
\hline
7 & LXD & \href{https://linuxcontainers.org/incus/docs/main/}{link} \\
\hline
8 & Rkt & \href{https://github.com/rkt/rkt}{link} \\
\hline
9 & Singularity & \href{https://docs.sylabs.io/guides/4.3/user-guide/}{link} \\
\hline
10 & runC & \href{https://github.com/opencontainers/runc}{link} \\
\hline
11 & CRI-O & \href{https://github.com/cri-o/cri-o}{link} \\
\hline
12 & Hyper-V containers & \href{https://docs.microsoft.com/en-us/virtualization/windowscontainers/}{link} \\
\hline
13 & OpenVZ & \href{https://openvz.org/}{link} \\
\hline
14 & Linux VServer & \href{http://linux-vserver.org/Documentation}{link} \\
\hline
15 & Google gVisor & \href{https://gvisor.dev/docs/}{link} \\
\hline
16 & Kata Containers & \href{https://katacontainers.io/docs/}{link} \\
\hline
17 & Firecracker & \href{https://firecracker-microvm.github.io/}{link} \\
\hline
18 & Sarus & \href{https://github.com/eth-cscs/sarus}{link} \\
\hline
\end{tabular}
\caption{Enlaces a la documentación de tecnologías de contenerización}\label{tab:documentacion-tecnologias}
\end{table}
\ChapterImageStar[cap:benchmarking]{Benchmarking}{./images/fondo.png}\label{cap:benchmarking}

\mbox{}\\
\section{Descripción del escenario de pruebas}
\noindent


\section{Definición de las pruebas}
\noindent
Para evaluar el rendimiento de distintas tecnologías de contenerización —específicamente Docker, Podman, LXC, LXD y Containerd— se diseñó un conjunto de pruebas orientadas a medir aspectos clave del desempeño en entornos controlados. Las pruebas incluyeron el consumo de CPU y memoria RAM, el tiempo de arranque de los contenedores, el \textit{throughput} de red y la latencia de acceso a disco. 
Con el objetivo de facilitar la repetibilidad y objetividad de los resultados, se desarrollaron scripts en \textit{Bash} que automatizan la ejecución de cada métrica en condiciones homogéneas. Estas pruebas permiten comparar las tecnologías evaluadas bajo criterios cuantificables y facilitar un análisis técnico de sus capacidades en escenarios reales de uso.

\section{Construcción de las pruebas}
\noindent
La construcción de las pruebas se llevó a cabo mediante el desarrollo de scripts automatizados en \textit{Bash}, diseñados para ejecutarse de forma uniforme sobre cada tecnología de contenerización evaluada. Cada script fue responsable de iniciar contenedores, ejecutar cargas de trabajo específicas y recolectar métricas de rendimiento relevantes.
Las características del entorno de prueba incluyeron el uso de una máquina con especificaciones técnicas homogéneas, propendiendo  porque las diferencias en rendimiento se debieran exclusivamente a las tecnologías evaluadas y no a variaciones en el hardware o la configuración del sistema operativo, las características del entorno de prueba pueden encontrarse en el apéndice~\ref{sec:env-benchmarking}.
Para medir el consumo de \CPU\ y memoria \RAM, se utilizó \texttt{pidstat}, una utilidad que permite la medición del consumo de recursos. El tiempo de arranque se determinó midiendo el intervalo entre la orden de inicio del contenedor y el momento en que estuvo completamente operativo. 
Para evaluar el \textit{throughput} de red se emplearon herramientas como \texttt{iperf}, mientras que la latencia de disco fue medida utilizando \texttt{fio}. Todas las pruebas fueron ejecutadas múltiples veces para reducir el impacto de variaciones puntuales y asegurar la confiabilidad de los resultados. Los scripts fueron programados para ejecutarse 10 veces; al final se extrae un promedio y este constituye el puntaje final de la tecnología de contenerización en cuestión.
En el repositorio \underline{\href{https://github.com/Anubis-1001/benchmark-tecnologias-de-contenerizacion} {\texttt{ GitHub benchmarking}}} se pueden encontrar los scripts resultantes de este proceso.

\section{Resultados de las pruebas}
\noindent
Los resultados obtenidos a partir de las pruebas evidencian diferencias significativas en el rendimiento entre las tecnologías de contenerización evaluadas. Estos se pueden consultar en el archivo de Excel \underline{\href{https://docs.google.com/spreadsheets/d/1Ce37Sm3Swyfa88Ur1yQbLarq_D86obUIAGGJocgQbUE/edit?usp=sharing} {\texttt{benchmarking\_tecnologias}}}.
En términos de consumo de CPU, Docker y Containerd presentan las mejores métricas; en consumo de memoria RAM, LXC y LXD mostraron un mejor uso de los recursos. En cuanto al tiempo de arranque, Containerd destacó por su velocidad, seguido de cerca por Docker, mientras que LXC presentó un arranque considerablemente más lento en comparación con las demás tecnologías.
Para el \textit{throughput} de red, todas las tecnologías mostraron un desempeño comparable, siendo LXC el más destacado; no obstante, Podman quedó muy por debajo en esta métrica. Finalmente, en la medición de latencia de disco, LXD y Containerd obtuvieron los mejores resultados, lo que sugiere una gestión de E/S más directa y liviana.
Estos resultados permiten establecer un panorama claro de fortalezas y debilidades de cada solución, según el tipo de carga o entorno de ejecución esperado.

\section{Métricas de rendimiento}
\noindent
De la ejecución de las pruebas se obtuvieron las siguientes métricas de rendimiento: \\

\noindent
La figura~\ref{fig:tabla-metricas-cpu} muestra una comparación del porcentaje de uso de CPU. Este análisis permite observar el nivel de utilización del procesador al ejecutar cargas de trabajo bajo entornos como Docker, Containerd, LXD, LXC y Podman. Los resultados tienen contrastes significativos entre las tecnologías, destacando aquellas que logran un mayor aprovechamiento del recurso de CPU frente a otras con un desempeño relativamente inferior.
\begin{figure}[H]
    \centering
    \includegraphics[width=\textwidth] {tablas-images/cp4/cpu.png}
    \caption{Métricas de uso de CPU}\label{fig:tabla-metricas-cpu}
\end{figure}

\noindent
La figura~\ref{fig:tabla-metricas-ram} presenta una comparación del porcentaje de uso de memoria RAM para el conjunto de tecnologías seleccionado. Este análisis permite evaluar el consumo de recursos de cada solución, considerando que un menor uso de memoria representa una ventaja en términos de escalabilidad y rendimiento en entornos de alta demanda. Los resultados muestran valores muy cercanos entre las tecnologías evaluadas, lo que sugiere un comportamiento homogéneo en este aspecto, aunque con ligeras variaciones que pueden influir en la elección según el contexto de uso.
\begin{figure}[H]
    \centering
    \includegraphics[scale=0.5] {tablas-images/cp4/ram.png}
    \caption{Métricas de uso de RAM}\label{fig:tabla-metricas-ram}
\end{figure}

\noindent
La figura~\ref{fig:tabla-metricas-io} muestra las métricas de latencia en operaciones de entrada/salida (E/S) para el conjunto de tecnologías seleccionado, expresadas en microsegundos. Este indicador es fundamental para evaluar el rendimiento de aplicaciones que dependen de un acceso rápido a datos y procesos intensivos en E/S. Los resultados reflejan diferencias significativas entre las tecnologías, evidenciando desde valores muy bajos, como en LXD, hasta tiempos de respuesta considerablemente altos, como en Podman. Este contraste permite identificar cuáles soluciones resultan más adecuadas en escenarios donde el manejo de datos es un factor crítico.
\begin{figure}[H]
    \centering
    \includegraphics[scale=0.5] {tablas-images/cp4/io.png}
    \caption{Métricas de entrada/salida}\label{fig:tabla-metricas-io}
\end{figure}

\noindent
La figura~\ref{fig:tabla-metricas-throughput} presenta una comparación de las métricas de throughput (rendimiento de transferencia de datos) para el conjunto de tecnologías seleccionado, expresadas en Gbits por segundo. Este parámetro resulta clave para evaluar la capacidad de cada tecnología en la transmisión de información a través de la red, lo cual impacta directamente en el desempeño de aplicaciones distribuidas y servicios en la nube. Los resultados muestran un rendimiento consistente en tecnologías como LXC, LXD, Docker y Containerd, mientras que Podman evidencia un throughput significativamente más bajo, lo que resalta contrastes importantes en términos de consumo de red
\begin{figure}[H]
    \centering
    \includegraphics[scale=0.5] {tablas-images/cp4/THROUGHTPUT.png}
    \caption{Métricas de throughput}\label{fig:tabla-metricas-throughput}
\end{figure}

\section{Análisis de los resultados}
\noindent
Los resultados obtenidos a partir de las pruebas permiten identificar comportamientos diferenciados entre las tecnologías de contenerización evaluadas. Containerd se posiciona como una de las soluciones más equilibradas, con excelente tiempo de arranque, bajo uso de CPU y buena latencia de disco, lo que lo hace ideal para entornos de bajos recursos.
LXC mostró consistentemente el menor consumo de recursos y alto rendimiento en red, lo que lo convierte en una opción adecuada para sistemas embebidos o despliegues que requieren un uso mínimo de \textit{overhead}. 
Por otro lado, Docker ofreció un rendimiento aceptable, pero con mayores niveles de consumo de CPU y latencia de disco, compensados por su madurez y ecosistema. 
LXD, al estar basado en LXC, heredó parte de sus beneficios, especialmente en uso de red, aunque con un ligero incremento en el tiempo de arranque.
En contraste, Podman, si bien ofrece un buen uso de CPU y memoria, presentó resultados considerablemente bajos en el \textit{throughput} de red y alta latencia de disco, lo cual podría limitar su aplicación en cargas sensibles a E/S o comunicación intensiva.

\vspace{0.5em}

\noindent En resumen, la elección de la tecnología de contenerización debe considerar el caso de uso específico: Containerd y LXC sobresalen por su bajo consumo de recursos; Docker y LXD ofrecen robustez y facilidad de integración; mientras que Podman puede ser más adecuado para entornos que prioricen la seguridad y compatibilidad con el modelo sin \textit{daemon}, siempre que el rendimiento de red o disco no sea crítico.

\ChapterImageStar[cap:dar]{Análisis de Decisiones y Resolución}{./images/fondo.png}\label{cap:dar}
\mbox{}\\
\section{Metodología de evaluación}
\noindent
La metodología de evaluación que se aplicó para la elección de la tecnología de \VBC fue \textit{Decision, analysis and resolution} (\DAR) de CMMI \citep{CMMIInstitute2010}. Esta metodología permitió evaluar las necesidades del grupo \GRID\ a través de un proceso estructurado que consideró múltiples alternativas, criterios de evaluación bien definidos y un análisis comparativo. En este caso, se analizaron las tecnologías \VBC\ encontradas en la revisión literaria, aplicando criterios como el tipo de licencia, la compatibilidad con herramientas de orquestación, el rendimiento entre otros. Así, el uso de \DAR\ no solo busca aportar transparencia al proceso, sino también trazabilidad y justificación técnica frente a una decisión para la arquitectura de infraestructura basada en contenedores.
El proceso de evaluación quedó registrado en un vídeo explicativo disponible en \href{https://youtu.be/xOmuQs2RX2c}{link}.

\section{Resultados de la evaluación}
\noindent
La tabla~\ref{tab:tabla-dar} presenta la aplicación de la metodología \DAR\ al proceso de selección de tecnologías de \VBC\. Este enfoque permitió evaluar de manera estructurada diversos criterios como el tipo de licencia, compatibilidad con orquestadores, soporte para redes y volúmenes, documentación disponible, consumo de recursos y costo de implementación en entornos productivos. Al asignar valores ponderados a cada aspecto, el análisis facilita la comparación objetiva entre múltiples alternativas, proporcionando un marco de referencia sólido para identificar la solución más adecuada según las necesidades técnicas, operativas y estratégicas de un proyecto.
\begin{figure}[H]
    \centering
    \includegraphics[width=\textwidth,height=0.85\textheight,keepaspectratio]{apendices/plantilla-DAR.png}
    \caption{Plantilla del análisis DAR}\label{fig:tabla-plantilla-dar}
\end{figure}

\section{Criterios de evaluación}

\subsection{VBC (¿Es una tecnología basada en contenedores?)}
\noindent
Este criterio define si la tecnología analizada entra dentro de la categoría de virtualización basada en contenedores, lo cual es el punto de partida para que pueda ser considerada en el análisis. Se evalúa como Sí (SI) o No (NO).

\subsection{Tipo de licencia}
\noindent
Se analiza el tipo de licencia bajo la cual se distribuye la tecnología, ya que esto afecta su adopción en proyectos académicos o comerciales. Las licencias permisivas (como Apache 2.0 o BSD) permiten mayor libertad de uso y modificación, mientras que licencias restrictivas (como AGPL o licencias propietarias) imponen ciertas limitaciones legales o técnicas.

\subsection{Posibilidad de orquestación}\label{sec:posibilidad-orquestacion}
\noindent
Se refiere a la capacidad de la tecnología para integrarse con herramientas de orquestación como Kubernetes, Docker Swarm o Apache mesos, lo cual es clave para la gestión automatizada de contenedores a gran escala. Una mayor puntuación indica mejor compatibilidad y soporte para estas herramientas.

\subsection{Compatibilidad con imágenes de Docker Hub}
\noindent
Evalúa si la tecnología puede ejecutar imágenes obtenidas directamente desde Docker Hub, el repositorio más utilizado para contenedores. Esto facilita la reutilización de contenedores existentes y la integración con flujos de trabajo ya establecidos.

\subsection{Soporte para redes personalizadas}
\noindent
Determina si la tecnología permite la creación y gestión de redes personalizadas entre contenedores. Este aspecto es fundamental en arquitecturas distribuidas, donde la comunicación entre servicios debe configurarse de forma segura.

\subsection{Persistencia de datos / volúmenes}
\noindent
Analiza si la solución permite la persistencia de datos, es decir, que los datos generados dentro de un contenedor puedan mantenerse incluso después de reiniciarlo o eliminarlo. Esto se logra mediante el uso de volúmenes o sistemas de almacenamiento externos.

\subsection{Documentación}
\noindent
Se valora la calidad, profundidad y accesibilidad de la documentación oficial. Una buena documentación facilita el aprendizaje, la resolución de problemas y la implementación de la tecnología.

\subsection{Soporte al proyecto}
\noindent
Considera el respaldo que tiene la tecnología por parte de la comunidad, empresas o fundaciones (como CNCF o Red Hat). Esto incluye mantenimiento activo, actualizaciones regulares, y foros o canales de ayuda disponibles.

\subsection{Popularidad}
\noindent
Este criterio mide la adopción y visibilidad de la tecnología, lo cual puede reflejar su madurez, confianza del mercado y disponibilidad de talento capacitado. Se puede estimar por métricas como el número de estrellas en GitHub.

\subsection{Consumo de recursos}
\noindent
Evalúa el nivel de consumo de recursos respecto al uso de CPU, memoria y almacenamiento. Se valora según lo que mencionan las organizaciones responsables en este aspecto.

\subsection{Compatibilidad de orquestación}
\noindent
Difiere levemente del punto~\ref{sec:posibilidad-orquestacion}, ya que aquí se mide qué tan bien se integra con los orquestadores, considerando estabilidad, plugins nativos y experiencia de uso. Un puntaje alto indica integración fluida y confiable.

\subsection{Costo de implementación y operación en ambientes productivos}
\noindent
Este criterio analiza los costos asociados a poner en marcha la tecnología en un entorno real. Incluye licencias, infraestructura, tiempo de configuración y mantenimiento. Una puntuación alta significa bajo costo o costo nulo, lo cual es ideal para instituciones académicas o proyectos con presupuesto limitado.
\section{Tecnología VBC ganadora}
\noindent
Del análisis comparativo realizado, \textbf{Containerd} se posiciona como la tecnología de virtualización basada en contenedores con mejor desempeño general. Destaca por su alta compatibilidad con Docker Hub, soporte para redes y volúmenes, excelente integración con orquestadores como Kubernetes, y una licencia permisiva que facilita su adopción. Además, cuenta con una sólida documentación y un respaldo activo de la comunidad. Estas características hacen de Containerd la opción adecuada, según el análisis \DAR, para ser implementada en ambientes productivos del grupo de investigación \GRID.
\clearpage
\section{Análisis \DAR\ del motor de Kubernetes}
\noindent
El análisis presentado en la figura~\ref{tab:tabla-dar-k8s} aplica la metodología \DAR\ al proceso de selección del motor de Kubernetes más adecuado para entornos de virtualización basados en contenedores. Este enfoque estructurado permitió evaluar diversos criterios técnicos y operativos. La importancia de la elección de un motor de Kubernetes radica en su impacto directo sobre la administración de contenedores. 
\input{tablas-images/cp5/dar-k8s.tex}

\ChapterImageStar[cap:cumplimiento-objetivos]{Cumplimiento de objetivos}{./images/fondo.png}\label{cap:cumplimiento-objetivos}
\mbox{}\\
\section{Cumplimiento de objetivos}
\noindent
En este capítulo se presenta un análisis del cumplimiento de los objetivos planteados al inicio del proyecto. Se evalúa cómo cada objetivo específico ha sido abordado y alcanzado a lo largo del desarrollo del trabajo, destacando los logros más significativos y las áreas que podrían beneficiarse de mejoras futuras.

\input{capitulos/referencias.tex}



\appendix

\chapter{Fichas técnicas y búsqueda en bases de datos}\label{apendice:fichas-y-busquedas}

\FloatBarrier\section{Ficha técnica del recurso tecnológico}
\begin{figure}[H]
    \centering
    \includegraphics[width=\textwidth,height=0.85\textheight,keepaspectratio]{apendices/caracterizacionInfraestructura.png}
    \caption{Ficha técnica del recurso tecnológico}\label{fig:tabla-ficha-tecnica}
\end{figure}

\FloatBarrier\section{Ficha técnica de servicios}
\begin{figure}[H]
    \centering
    \includegraphics[width=\textwidth,height=0.85\textheight,keepaspectratio]{apendices/caracterizacionServicios.png}
    \caption{Ficha técnica de servicios}\label{fig:tabla-ficha-servicios}
\end{figure}

%--------------------------------------------------------------------------------
\FloatBarrier\chapter{Búsquedas en bases de datos}

\section{Cadenas de búsqueda}\label{sec:cadenas-busqueda}

\begin{tcolorbox}[
  colback=gray!5, 
  colframe=black!60, 
  title=Cadena de búsqueda en ACM para educación, 
  fonttitle=\bfseries, 
  sharp corners=south
]
\scriptsize % o \footnotesize, \tiny según lo pequeño que lo quieras
\begin{verbatim}
(Title:("Container-based virtualization" OR "Application virtualization" OR "Docker" OR 
"Lightweight Virtualization") AND Title:("Education" OR "Education System" 
OR "Education Development" OR "Higher Education") ) 

OR

(Abstract:("Container-based virtualization" OR "Application virtualization" OR "Docker"
 OR "Lightweight Virtualization") AND Abstract:("Education" OR "Education System" 
 OR "Education Development" OR "Higher Education") )

OR

(Keyword:("Container-based virtualization" OR "Application virtualization" OR "Docker" OR 
"Lightweight Virtualization")
AND Keyword:("Education" OR "Education System" OR "Education Development" 
OR "Higher Education"))
\end{verbatim}
\end{tcolorbox}

\begin{tcolorbox}[
  colback=gray!5, 
  colframe=black!60, 
  title=Cadena de búsqueda en ACM para investigación, 
  fonttitle=\bfseries, 
  sharp corners=south
]
\scriptsize % puedes usar \tiny para hacerlo aún más pequeño
\begin{verbatim}
(Title:("Container-based virtualization" OR "Application virtualization" OR "Docker" OR 
"Lightweight Virtualization") AND Title:("Research" OR "Research Group" OR 
"Research Proposal"))

OR

(Abstract:("Container-based virtualization" OR "Application virtualization" OR "Docker" OR 
"Lightweight Virtualization") AND Abstract:("Research" OR "Research Group" OR 
"Research Proposal"))

OR

(Keyword:("Container-based virtualization" OR "Application virtualization" OR "Docker" OR 
"Lightweight Virtualization") AND Keyword:("Research" OR "Research Group" OR 
"Research Proposal"))
\end{verbatim}
\end{tcolorbox}

\begin{tcolorbox}[
  colback=gray!5, 
  colframe=black!60, 
  title=Cadena de búsqueda en ACM para extensión, 
  fonttitle=\bfseries, 
  sharp corners=south
]
\scriptsize % puedes usar \tiny para hacerlo aún más pequeño
\begin{verbatim}
(Title:("Container-based virtualization" OR "Application virtualization" OR "Docker" OR 
"Lightweight Virtualization") AND Title:("Industry" OR “IT Services” OR 
“Technology Infrastructure” OR “Cloud Computing”) ) 

OR

(Abstract:("Container-based virtualization" OR "Application virtualization" OR "Docker" 
OR "Lightweight Virtualization") AND Abstract:("Industry" OR “IT Services” OR 
“Technology Infrastructure” OR “Cloud Computing”) )

OR

(Keyword:("Container-based virtualization" OR "Application virtualization" OR "Docker" 
OR "Lightweight Virtualization")
AND Keyword:("Industry" OR “IT Services” OR “Technology Infrastructure” 
OR “Cloud Computing”))

\end{verbatim}
\end{tcolorbox}

\begin{tcolorbox}[
  colback=gray!5, 
  colframe=black!60, 
  title=Cadena de búsqueda en IEE para educación, 
  fonttitle=\bfseries, 
  sharp corners=south
]
\scriptsize % puedes usar \tiny para hacerlo aún más pequeño
\begin{verbatim}
(("Abstract":"Container-based virtualization" OR "Abstract":"Application virtualization" 
OR "Abstract":"Docker" OR "Abstract":"Lightweight Virtualization") AND ("Abstract":"Education" 
OR "Abstract":"Education System" OR "Abstract":"Education Development”  OR 
"Abstract":"Higher Education”)) 

OR (("Publication Title":"Container-based virtualization" OR "Publication 
Title":"Application virtualization" 
OR "Publication Title":"Docker" OR "Publication Title":"Lightweight Virtualization") 
AND ("Publication Title":"Education" 
OR "Publication Title":"Education System" OR "Publication Title":"Education Development”  
OR "Publication Title":"Higher Education” ))

OR (("Author Keywords":"Container-based virtualization" OR 
"Author Keywords":"Application virtualization" OR 
"Author Keywords":"Docker" OR "Author Keywords":"Lightweight Virtualization") AND 
("Author Keywords":"Education" 
OR "Author Keywords":"Education System" OR "Author Keywords":"Education Development”  
OR "Author Keywords":"Higher Education”))
\end{verbatim}
\end{tcolorbox}


\begin{tcolorbox}[
  colback=gray!5, 
  colframe=black!60, 
  title=Cadena de búsqueda en IEE para investigación, 
  fonttitle=\bfseries, 
  sharp corners=south
]
\scriptsize % puedes usar \tiny para hacerlo aún más pequeño
\begin{verbatim}
(("Abstract":"Container-based virtualization" OR "Abstract":"Application virtualization" 
OR "Abstract":"Docker" OR "Abstract":"Lightweight Virtualization") AND 
("Abstract":"Research Group" OR "Abstract":"Research Proposal")) 

OR (("Publication Title":"Container-based virtualization" OR 
"Publication Title":"Application virtualization" OR "Publication Title":"Docker" OR 
"Publication Title":"Lightweight Virtualization") AND 
("Publication Title":"Research Group" OR "Publication Title":"Research Proposal" ))

OR (("Author Keywords":"Container-based virtualization" OR 
"Author Keywords":"Application virtualization" OR "Author Keywords":"Docker" OR 
"Author Keywords":"Lightweight Virtualization") AND 
("Author Keywords":"Research Group" OR "Author Keywords":"Research Proposal"))
\end{verbatim}
\end{tcolorbox}

\begin{tcolorbox}[
  colback=gray!5, 
  colframe=black!60, 
  title=Cadena de búsqueda en IEE para extensión, 
  fonttitle=\bfseries, 
  sharp corners=south
]
\scriptsize % puedes usar \tiny para hacerlo aún más pequeño
\begin{verbatim}
(("Abstract":"Container-based virtualization" OR "Abstract":"Application virtualization" 
OR "Abstract":"Docker" OR "Abstract":"Lightweight Virtualization") AND 
("Abstract":"Industry" OR "Abstract":"IT Services" OR 
"Abstract":"Technology Infrastructure" OR "Abstract":"Cloud Computing")) 

OR (("Publication Title":"Container-based virtualization" OR 
"Publication Title":"Application virtualization" 
OR "Publication Title":"Docker" OR "Publication Title":"Lightweight Virtualization") AND 
("Publication Title":"Industry" OR "Publication Title":"IT Services" OR 
"Publication Title":"Technology Infrastructure" OR "Publication Title":"Cloud Computing"))

OR (("Author Keywords":"Container-based virtualization" OR 
"Author Keywords":"Application virtualization" OR "Author Keywords":"Docker" OR 
"Author Keywords":"Lightweight Virtualization") AND ("Author Keywords":"Industry" OR 
"Author Keywords":"IT Services" OR "Author Keywords":"Technology Infrastructure" OR 
"Author Keywords":"Cloud Computing"))
\end{verbatim}
\end{tcolorbox}

\begin{tcolorbox}[
  colback=gray!5, 
  colframe=black!60, 
  title=Cadena de búsqueda en Springer para educación, 
  fonttitle=\bfseries, 
  sharp corners=south
]
\scriptsize % puedes usar \tiny para hacerlo aún más pequeño
\begin{verbatim}
(title:("Container-based virtualization" OR "Application virtualization" OR 
"Docker" OR "Lightweight Virtualization") AND title:("Education" OR 
"Education System" OR "Education Development" OR "Higher Education"))

OR

(abstract:("Container-based virtualization" OR "Application virtualization" OR 
"Docker" OR "Lightweight Virtualization") AND abstract:("Education" OR 
"Education System" OR "Education Development" OR "Higher Education"))

OR 

(keyword:("Container-based virtualization" OR "Application virtualization" OR 
"Docker" OR "Lightweight Virtualization") AND keyword:("Education" OR 
"Education System" OR "Education Development" OR "Higher Education"))

\end{verbatim}
\end{tcolorbox}

\begin{tcolorbox}[
  colback=gray!5, 
  colframe=black!60, 
  title=Cadena de búsqueda en Springer para investigación, 
  fonttitle=\bfseries, 
  sharp corners=south
]
\scriptsize % puedes usar \tiny para hacerlo aún más pequeño
\begin{verbatim}
(title:("Container-based virtualization" OR "Application virtualization" OR 
"Docker" OR "Lightweight Virtualization") AND title:("research" OR 
"Research Group" OR "Research Proposal"))

OR

(abstract:("Container-based virtualization" OR "Application virtualization" 
OR "Docker" OR "Lightweight Virtualization") AND abstract:("research" 
OR "Research Group" OR "Research Proposal"))

OR 

(keyword:("Container-based virtualization" OR "Application virtualization"
 OR "Docker" OR "Lightweight Virtualization") AND keyword:("research" OR 
 "Research Group" OR "Research Proposal"))

\end{verbatim}
\end{tcolorbox}

\begin{tcolorbox}[
  colback=gray!5, 
  colframe=black!60, 
  title=Cadena de búsqueda en Springer para extensión, 
  fonttitle=\bfseries, 
  sharp corners=south
]
\scriptsize % puedes usar \tiny para hacerlo aún más pequeño
\begin{verbatim}
(title:("Container-based virtualization" OR "Application virtualization"
 OR "Docker" OR "Lightweight Virtualization") AND title:("Industry" OR 
 “IT Services” OR “Technology Infrastructure” OR “Cloud Computing”))

OR

(abstract:("Container-based virtualization" OR "Application virtualization" 
OR "Docker" OR "Lightweight Virtualization") AND abstract:("Industry" OR 
“IT Services” OR “Technology Infrastructure” OR “Cloud Computing”))

OR 

(keyword:("Container-based virtualization" OR "Application virtualization"
 OR "Docker" OR "Lightweight Virtualization") AND keyword:("Industry" 
 OR “IT Services” OR “Technology Infrastructure” OR “Cloud Computing”))

\end{verbatim}
\end{tcolorbox}

\begin{tcolorbox}[
  colback=gray!5, 
  colframe=black!60, 
  title=Cadena de búsqueda en Science Direct para educación, 
  fonttitle=\bfseries, 
  sharp corners=south
]
\scriptsize % puedes usar \tiny para hacerlo aún más pequeño
\begin{verbatim}
("Container-based virtualization" OR "Application virtualization" 
OR "Docker" OR "Lightweight Virtualization")  AND ("Education" OR 
"Education System" OR "Education Development" OR "Higher Education")
\end{verbatim}
\end{tcolorbox}


\begin{tcolorbox}[
  colback=gray!5, 
  colframe=black!60, 
  title=Cadena de búsqueda en Science Direct para investigación, 
  fonttitle=\bfseries, 
  sharp corners=south
]
\scriptsize % puedes usar \tiny para hacerlo aún más pequeño
\begin{verbatim}
("Container-based virtualization" OR "Application virtualization" OR 
"Docker" OR "Lightweight Virtualization")  AND ("Research" OR 
"Research Group" OR "Research Proposal")
\end{verbatim}
\end{tcolorbox}

\begin{tcolorbox}[
  colback=gray!5, 
  colframe=black!60, 
  title=Cadena de búsqueda en Science Direct para extensión, 
  fonttitle=\bfseries, 
  sharp corners=south
]
\scriptsize % puedes usar \tiny para hacerlo aún más pequeño
\begin{verbatim}
("Container-based virtualization" OR "Application virtualization" OR "Docker" OR 
"Lightweight Virtualization")  AND 
(“Industry” OR "IT Services" OR "Technology Infrastructure" OR "Cloud Computing")
\end{verbatim}
\end{tcolorbox}

\begin{tcolorbox}[
  colback=gray!5, 
  colframe=black!60, 
  title=Cadena de búsqueda en Taylor \& Francis para educación, 
  fonttitle=\bfseries, 
  sharp corners=south
]
\scriptsize % puedes usar \tiny para hacerlo aún más pequeño
\begin{verbatim}
("Application virtualization" OR "Docker" OR "Lightweight Virtualization" OR "Docker Container")   
AND   
("Education System" OR "Education Sector" OR "Education Development" OR "Higher Education")
\end{verbatim}
\end{tcolorbox}

\begin{tcolorbox}[
  colback=gray!5, 
  colframe=black!60, 
  title=Cadena de búsqueda en Taylor \& Francis para investigación, 
  fonttitle=\bfseries, 
  sharp corners=south
]
\scriptsize % puedes usar \tiny para hacerlo aún más pequeño
\begin{verbatim}
("Application virtualization" OR "Docker" OR "Lightweight Virtualization" OR "Docker Container")
AND   
("Specific Research Areas" OR "Research Group" OR "Research Proposal" OR "Research and Development")
\end{verbatim}
\end{tcolorbox}

\begin{tcolorbox}[
  colback=gray!5, 
  colframe=black!60, 
  title=Cadena de búsqueda en Taylor \& Francis para extensión, 
  fonttitle=\bfseries, 
  sharp corners=south
]
\scriptsize % puedes usar \tiny para hacerlo aún más pequeño
\begin{verbatim}
("Application virtualization" OR "Docker" OR "Lightweight Virtualization" OR "Docker Container")  
AND 
(“Industry” OR "IT Services" OR "Technology Infrastructure" OR "Cloud Computing")
\end{verbatim}
\end{tcolorbox}


\section{Búsqueda de artículos sin criterios de inclusión/exclusión}\label{sec:busqueda-sin-criterios}
\begin{figure}[H]
    \centering
    \includegraphics[width=\textwidth,keepaspectratio]{apendices/BD/sin-criterios/ACM-ed.png}
    \caption{Búsqueda de artículos de educación en ACM sin criterios de inclusión/exclusión \\
    Fecha de acceso: 12/03/25 9:13 pm
    }
\end{figure}
\FloatBarrier\begin{figure}[H]
    \centering
    \includegraphics[width=\textwidth,keepaspectratio]{apendices/BD/sin-criterios/ACM-inv.png}
    \caption{Búsqueda de artículos de investigación en ACM sin criterios de inclusión/exclusión \\
    Fecha de acceso: 12/03/25 8:23 pm
    }
\end{figure}
\FloatBarrier\begin{figure}[H]
    \centering
    \includegraphics[width=\textwidth,keepaspectratio]{apendices/BD/sin-criterios/ACM-ind.png}
    \caption{Búsqueda de artículos de extensión en ACM sin criterios de inclusión/exclusión \\
    Fecha de acceso: 12/03/25 9:20 pm
    }
\end{figure}
\FloatBarrier\begin{figure}[H]
    \centering
    \includegraphics[width=\textwidth,keepaspectratio]{apendices/BD/sin-criterios/IEEE-ed.png}
    \caption{Búsqueda de artículos de educación en IEEE sin criterios de inclusión/exclusión
    Fecha de acceso: 7/03/25 8:50 pm
    }
\end{figure}
\FloatBarrier\begin{figure}[H]
    \centering
    \includegraphics[width=\textwidth,keepaspectratio]{apendices/BD/sin-criterios/IEEE-inv.png}
    \caption{Búsqueda de artículos de investigación en IEEE sin criterios de inclusión/exclusión
    Fecha de acceso: 7/03/25 8:46 pm
    }
\end{figure}
\FloatBarrier\begin{figure}[H]
    \centering
    \includegraphics[width=\textwidth,keepaspectratio]{apendices/BD/sin-criterios/IEEE-ind.png}
    \caption{Búsqueda de artículos de extensión en IEEE sin criterios de inclusión/exclusión
    Fecha de acceso: 12/03/25 8:54 pm
    }
\end{figure}
\FloatBarrier\begin{figure}[H]
    \centering
    \includegraphics[width=\textwidth,keepaspectratio]{apendices/BD/sin-criterios/Springer-ed.png}
    \caption{Búsqueda de artículos de educación en Springer sin criterios de inclusión/exclusión
    Fecha de acceso: 12/03/25 9:58 pm
    }
\end{figure}
\FloatBarrier\begin{figure}[H]
    \centering
    \includegraphics[width=\textwidth,keepaspectratio]{apendices/BD/sin-criterios/Springer-inv.png}
    \caption{Búsqueda de artículos de investigación en Springer sin criterios de inclusión/exclusión
    Fecha de acceso: 13/03/25 12:40 pm
    }
\end{figure}
\FloatBarrier\begin{figure}[H]
    \centering
    \includegraphics[width=\textwidth,keepaspectratio]{apendices/BD/sin-criterios/Springer-ind.png}
    \caption{Búsqueda de artículos de extensión en Springer sin criterios de inclusión/exclusión
    Fecha de acceso: 13/03/25 12:48 pm}
\end{figure}
\FloatBarrier\begin{figure}[H]
    \centering
    \includegraphics[width=\textwidth,keepaspectratio]{apendices/BD/sin-criterios/SD-ed.png}
    \caption{Búsqueda de artículos de educación en Science Direct sin criterios de inclusión/exclusión
    Fecha de acceso: 13/03/25 1:03 pm}
\end{figure}
\FloatBarrier\begin{figure}[H]
    \centering
    \includegraphics[width=\textwidth,keepaspectratio]{apendices/BD/sin-criterios/SD-inv.png}
    \caption{Búsqueda de artículos de investigación en Science Direct sin criterios de inclusión/exclusión
    Fecha de acceso: 13/03/25 1:43 pm}
\end{figure}
\FloatBarrier\begin{figure}[H]
    \centering
    \includegraphics[width=\textwidth,keepaspectratio]{apendices/BD/sin-criterios/SD-ind.png}
    \caption{Búsqueda de artículos de extensión en Science Direct sin criterios de inclusión/exclusión
    Fecha de acceso: 13/03/25 1:48 pm}
\end{figure}
\FloatBarrier\begin{figure}[H]
    \centering
    \includegraphics[width=\textwidth,keepaspectratio]{apendices/BD/sin-criterios/TF-ed.png}
    \caption{Búsqueda de artículos de educación en Taylor \& Francis sin criterios de inclusión/exclusión
    Fecha de acceso: 16/03/25 9:21 pm}
\end{figure}
\FloatBarrier\begin{figure}[H]
    \centering
    \includegraphics[width=\textwidth,keepaspectratio]{apendices/BD/sin-criterios/TF-inv.png}
    \caption{Búsqueda de artículos de investigación en Taylor \& Francis sin criterios de inclusión/exclusión
    Fecha de acceso: 16/03/25 9:31 pm
    }
\end{figure}
\FloatBarrier\begin{figure}[H]
    \centering
    \includegraphics[width=\textwidth,keepaspectratio]{apendices/BD/sin-criterios/TF-ind.png}
    \caption{Búsqueda de artículos de extensión en Taylor \& Francis sin criterios de inclusión/exclusión
    Fecha de acceso: 16/03/25 9:34 pm}
\end{figure}

\FloatBarrier\section{Búsqueda de artículos usando criterios de inclusión/exclusión}\label{sec:busqueda-con-criterios}
\begin{figure}[H]
    \centering
    \includegraphics[width=\textwidth,keepaspectratio]{apendices/BD/criterios/ACM-ed.png}
    \caption{Búsqueda de artículos de educación en ACM con criterios de inclusión/exclusión
    Fecha de acceso: 20/03/25 1:15 pm
    }
\end{figure}
\FloatBarrier\begin{figure}[H]
    \centering
    \includegraphics[width=\textwidth,keepaspectratio]{apendices/BD/criterios/ACM-inv.png}
    \caption{Búsqueda de artículos de investigación en ACM con criterios de inclusión/exclusión
    Fecha de acceso: 20/03/25 1:19 pm
    }
\end{figure}
\FloatBarrier\begin{figure}[H]
    \centering
    \includegraphics[width=\textwidth,keepaspectratio]{apendices/BD/criterios/ACM-ind.png}
    \caption{Búsqueda de artículos de extensión en ACM con criterios de inclusión/exclusión
    Fecha de acceso: 20/03/25 1:20 pm
    }
\end{figure}
\FloatBarrier\begin{figure}[H]
    \centering
    \includegraphics[width=\textwidth,keepaspectratio]{apendices/BD/criterios/IEEE-ed.png}
    \caption{Búsqueda de artículos de educación en IEEE con criterios de inclusión/exclusión
    Fecha de acceso: 20/03/25 1:27 pm
    }
\end{figure}
\FloatBarrier\begin{figure}[H]
    \centering
    \includegraphics[width=\textwidth,keepaspectratio]{apendices/BD/criterios/IEEE-ind.png}
    \caption{Búsqueda de artículos de extensión en IEEE con criterios de inclusión/exclusión
    Fecha de acceso: 20/03/25 1:37 pm
    }
\end{figure}
\FloatBarrier\begin{figure}[H]
    \centering
    \includegraphics[width=\textwidth,keepaspectratio]{apendices/BD/criterios/Springer-ed.png}
    \caption{Búsqueda de artículos de educación en Springer con criterios de inclusión/exclusión
    Fecha de acceso: 20/03/25 2:29 pm
    }
\end{figure}
\FloatBarrier\begin{figure}[H]
    \centering
    \includegraphics[width=\textwidth,keepaspectratio]{apendices/BD/criterios/Springer-inv.png}
    \caption{Búsqueda de artículos de investigación en Springer con criterios de inclusión/exclusión
    Fecha de acceso: 16/03/25 11:05 am
    }
\end{figure}
\FloatBarrier\begin{figure}[H]
    \centering
    \includegraphics[width=\textwidth,keepaspectratio]{apendices/BD/criterios/Springer-ind.png}
    \caption{Búsqueda de artículos de extensión en Springer con criterios de inclusión/exclusión
    Fecha de acceso: 16/03/25 11:07 am
    }
\end{figure}
\FloatBarrier\begin{figure}[H]
    \centering
    \includegraphics[width=\textwidth,keepaspectratio]{apendices/BD/criterios/SD-ed.png}
    \caption{Búsqueda de artículos de educación en Science Direct con criterios de inclusión/exclusión
    Fecha de acceso: 20/03/25 2:59 am
    }
\end{figure}
\FloatBarrier\begin{figure}[H]
    \centering
    \includegraphics[width=\textwidth,keepaspectratio]{apendices/BD/criterios/SD-inv.png}
    \caption{Búsqueda de artículos de investigación en Science Direct con criterios de inclusión/exclusión
    Fecha de acceso: 20/03/25 3:01 am
    }
\end{figure}
\FloatBarrier\begin{figure}[H]
    \centering
    \includegraphics[width=\textwidth,keepaspectratio]{apendices/BD/criterios/SD-ind.png}
    \caption{Búsqueda de artículos de extensión en Science Direct con criterios de inclusión/exclusión
    Fecha de acceso: 20/03/25 3:07 am
    }
\end{figure}
\FloatBarrier\begin{figure}[H]
    \centering
    \includegraphics[width=\textwidth,keepaspectratio]{apendices/BD/criterios/TF-ed.png}
    \caption{Búsqueda de artículos de educación en Taylor \& Francis con criterios de inclusión/exclusión
    Fecha de acceso: 20/03/25 4:46 am
    }
\end{figure}
\FloatBarrier\begin{figure}[H]
    \centering
    \includegraphics[width=\textwidth,keepaspectratio]{apendices/BD/criterios/TF-inv.png}
    \caption{Búsqueda de artículos de investigación en Taylor \& Francis con criterios de inclusión/exclusión
    Fecha de acceso: 20/03/25 4:49 am
    }
\end{figure}
\FloatBarrier\begin{figure}[H]
    \centering
    \includegraphics[width=\textwidth,keepaspectratio]{apendices/BD/criterios/TF-ind.png}
    \caption{Búsqueda de artículos de extensión en Taylor \& Francis con criterios de inclusión/exclusión
    Fecha de acceso: 20/03/25 4:50 am
    }
\end{figure}



%--------------------------------------------------------------------------------
\FloatBarrier\chapter{Plantilla del análisis DAR}
\begin{figure}[H]
    \centering
    \includegraphics[width=\textwidth,height=0.85\textheight,keepaspectratio]{apendices/plantilla-DAR.png}
    \caption{Plantilla del análisis DAR}\label{fig:tabla-plantilla-dar}
\end{figure}


%--------------------------------------------------------------------------------
\FloatBarrier\chapter{elementos en ArchiMate}
Esta sección del apéndice fue tomado del trabajo de grado de Michael Stiven Honores Quishpillo y Juan David Naranjo Sánchez titulado "Propuesta de Transformación Digital para la Gestión de la Información en el Proceso de Producción Establecido en la Organización Prefabricados JAMAR Validada a través de la Construcción de un Prototipo Funcional". 

\subsection{elementos de negocio}

\begin{longtable}{|c|p{8cm}|}
\caption{Elementos de negocio en ArchiMate} \label{tab:elementos-negocio-archimate} \\
\hline
\textbf{Icono} & \textbf{Descripción} \\
\hline
\endfirsthead

\caption[]{Elementos de negocio en ArchiMate (continuación)} \\
\hline
\textbf{Icono} & \textbf{Descripción} \\
\hline
\endhead

\hline
\endfoot

\endlastfoot
\includegraphics{apendices/ARCHI/business/actor.png} & 
\textbf{Business Actor:} Una entidad que puede realizar un comportamiento dentro de la empresa, como una persona o una organización. \\
\hline
\includegraphics{apendices/ARCHI/business/rol.png} & 
\textbf{Business Role:} Define el conjunto de responsabilidades y comportamientos que un actor de negocio puede desempeñar. \\
\hline
\includegraphics{apendices/ARCHI/business/colaboration.png} & 
\textbf{Business Collaboration:} Una agregación de dos o más roles de negocio que trabajan juntos para alcanzar un objetivo común. \\
\hline
\includegraphics{apendices/ARCHI/business/interface.png} & 
\textbf{Business Interface:} Un punto de acceso donde un servicio de negocio es provisto a los actores de negocio. \\
\hline
\includegraphics{apendices/ARCHI/business/process.png} & 
\textbf{Business Process:} Un conjunto de actividades que logran un resultado específico para un cliente interno o externo. \\
\hline
\includegraphics{apendices/ARCHI/business/function.png} & 
\textbf{Business Function:} Una agrupación de comportamientos con una base similar de recursos. \\
\hline
\includegraphics{apendices/ARCHI/business/interaction.png} & 
\textbf{Business Interaction:} Un comportamiento que describe la interacción entre dos o más roles de negocio. \\
\hline
\includegraphics{apendices/ARCHI/business/service.png} & 
\textbf{Business Service:} Un servicio que satisface una necesidad de un cliente, interno o externo, de la empresa. \\
\hline
\includegraphics{apendices/ARCHI/business/event.png} & 
\textbf{Business Event:} Algo que sucede y que afecta la continuidad de un proceso de negocio. \\
\hline
\includegraphics{apendices/ARCHI/business/object.png} & 
\textbf{Business Object:} Un concepto usado dentro de un dominio de negocio particular. \\
\hline
\includegraphics{apendices/ARCHI/business/contract.png} & 
\textbf{Contract:} Un acuerdo formal o informal que especifica los derechos y obligaciones asociados con un producto. \\
\hline
\includegraphics{apendices/ARCHI/business/representation.png} & 
\textbf{Representation:} Una forma perceptible de la información llevada por un objeto de negocio. \\
\hline
\includegraphics{apendices/ARCHI/business/product.png} & 
\textbf{Product:} Una colección coherente de servicios y/o objetos pasivos, acompañada de un contrato/conjunto de acuerdos. \\
\hline
\includegraphics{apendices/ARCHI/business/grouping.png} &
\textbf{Grouping:} Un mecanismo para agrupar elementos relacionados en un modelo de arquitectura. \\
\end{longtable}

\subsection{elementos de aplicación}




%--------------------------------------------------------------------------------
\FloatBarrier\chapter{Solicitud de recursos en formato YAML}
\begin{figure}[H]
    \centering
    \includegraphics[scale=0.3]{apendices/ex-yml/yml-ex.png}
    \caption{Ejemplo de solicitud de recursos tecnológicos \.yml}\label{fig:ap-ejemplo-yml}
\end{figure}

%--------------------------------------------------------------------------------
\FloatBarrier\chapter{Eventos de difusión}

\FloatBarrier\section{ACOFI 2024}
\noindent
En esta sección se presenta la ponencia presentada en el congreso ACOFI 2025 como resultado de la investigación desarrollada en este trabajo.

% Página 1
\begin{figure}[H]
    \centering
    \begin{tcolorbox}[
        colback=white,
        colframe=gray!50,
        boxrule=1pt,
        arc=2pt,
        boxsep=5pt,
        left=3pt,
        right=3pt,
        top=3pt,
        bottom=3pt,
        drop shadow
    ]
        \includegraphics[width=0.95\textwidth,keepaspectratio]{apendices/ACOFI/pagina_1.png}
    \end{tcolorbox}
    \caption{Ponencia ACOFI --- Página 1}\label{fig:acofi-pagina-1}
\end{figure}
\FloatBarrier% Página 2
\begin{figure}[H]
    \centering
    \begin{tcolorbox}[
        colback=white,
        colframe=gray!50,
        boxrule=1pt,
        arc=2pt,
        boxsep=5pt,
        left=3pt,
        right=3pt,
        top=3pt,
        bottom=3pt,
        drop shadow
    ]
        \includegraphics[width=0.95\textwidth,keepaspectratio]{apendices/ACOFI/pagina_2.png}
    \end{tcolorbox}
    \caption{Ponencia ACOFI --- Página 2}\label{fig:acofi-pagina-2}
\end{figure}
\FloatBarrier% Página 3
\begin{figure}[H]
    \centering
    \begin{tcolorbox}[
        colback=white,
        colframe=gray!50,
        boxrule=1pt,
        arc=2pt,
        boxsep=5pt,
        left=3pt,
        right=3pt,
        top=3pt,
        bottom=3pt,
        drop shadow
    ]
        \includegraphics[width=0.95\textwidth,keepaspectratio]{apendices/ACOFI/pagina_3.png}
    \end{tcolorbox}
    \caption{Ponencia ACOFI --- Página 3}\label{fig:acofi-pagina-3}
\end{figure}
\FloatBarrier% Página 4
\begin{figure}[H]
    \centering
    \begin{tcolorbox}[
        colback=white,
        colframe=gray!50,
        boxrule=1pt,
        arc=2pt,
        boxsep=5pt,
        left=3pt,
        right=3pt,
        top=3pt,
        bottom=3pt,
        drop shadow
    ]
        \includegraphics[width=0.95\textwidth,keepaspectratio]{apendices/ACOFI/pagina_4.png}
    \end{tcolorbox}
    \caption{Ponencia ACOFI --- Página 4}\label{fig:acofi-pagina-4}
\end{figure}
\FloatBarrier% Página 5
\begin{figure}[H]
    \centering
    \begin{tcolorbox}[
        colback=white,
        colframe=gray!50,
        boxrule=1pt,
        arc=2pt,
        boxsep=5pt,
        left=3pt,
        right=3pt,
        top=3pt,
        bottom=3pt,
        drop shadow
    ]
        \includegraphics[width=0.95\textwidth,keepaspectratio]{apendices/ACOFI/pagina_5.png}
    \end{tcolorbox}
    \caption{Ponencia ACOFI --- Página 5}\label{fig:acofi-pagina-5}
\end{figure}
\FloatBarrier% Página 6
\begin{figure}[H]
    \centering
    \begin{tcolorbox}[
        colback=white,
        colframe=gray!50,
        boxrule=1pt,
        arc=2pt,
        boxsep=5pt,
        left=3pt,
        right=3pt,
        top=3pt,
        bottom=3pt,
        drop shadow
    ]
        \includegraphics[width=0.95\textwidth,keepaspectratio]{apendices/ACOFI/pagina_6.png}
    \end{tcolorbox}
    \caption{Ponencia ACOFI --- Página 6}\label{fig:acofi-pagina-6}
\end{figure}
\FloatBarrier% Página 7
\begin{figure}[H]
    \centering
    \begin{tcolorbox}[
        colback=white,
        colframe=gray!50,
        boxrule=1pt,
        arc=2pt,
        boxsep=5pt,
        left=3pt,
        right=3pt,
        top=3pt,
        bottom=3pt,
        drop shadow
    ]
        \includegraphics[width=0.95\textwidth,keepaspectratio]{apendices/ACOFI/pagina_7.png}
    \end{tcolorbox}
    \caption{Ponencia ACOFI --- Página 7}\label{fig:acofi-pagina-7}
\end{figure}
\FloatBarrier% Página 8
\begin{figure}[H]
    \centering
    \begin{tcolorbox}[
        colback=white,
        colframe=gray!50,
        boxrule=1pt,
        arc=2pt,
        boxsep=5pt,
        left=3pt,
        right=3pt,
        top=3pt,
        bottom=3pt,
        drop shadow
    ]
        \includegraphics[width=0.95\textwidth,keepaspectratio]{apendices/ACOFI/pagina_8.png}
    \end{tcolorbox}
    \caption{Ponencia ACOFI --- Página 8}\label{fig:acofi-pagina-8}
\end{figure}
\FloatBarrier% Página 9
\begin{figure}[H]
    \centering
    \begin{tcolorbox}[
        colback=white,
        colframe=gray!50,
        boxrule=1pt,
        arc=2pt,
        boxsep=5pt,
        left=3pt,
        right=3pt,
        top=3pt,
        bottom=3pt,
        drop shadow
    ]
        \includegraphics[width=0.95\textwidth,keepaspectratio]{apendices/ACOFI/pagina_9.png}
    \end{tcolorbox}
    \caption{Ponencia ACOFI --- Página 9}\label{fig:acofi-pagina-9}
\end{figure}
\FloatBarrier% Página 10
\begin{figure}[H]
    \centering
    \begin{tcolorbox}[
        colback=white,
        colframe=gray!50,
        boxrule=1pt,
        arc=2pt,
        boxsep=5pt,
        left=3pt,
        right=3pt,
        top=3pt,
        bottom=3pt,
        drop shadow
    ]
        \includegraphics[width=0.95\textwidth,keepaspectratio]{apendices/ACOFI/pagina_10.png}
    \end{tcolorbox}
    \caption{Ponencia ACOFI --- Página 10}\label{fig:acofi-pagina-10}
\end{figure}


\FloatBarrier\section{JISA 2025}
En esta sección se presenta el artículo de revista publicado en Journal of Information Systems and Applications (JISA) como resultado de la investigación desarrollada en este trabajo.

% Portada del artículo JISA
\begin{figure}[H]
    \centering
    \begin{tcolorbox}[
        colback=white,
        colframe=gray!50,
        boxrule=1pt,
        arc=2pt,
        boxsep=5pt,
        left=3pt,
        right=3pt,
        top=3pt,
        bottom=3pt,
        drop shadow
    ]
        \includegraphics[width=0.95\textwidth,keepaspectratio]{apendices/PORTADA-ART/JISA-PORTADA.png}
    \end{tcolorbox}
    \caption{Portada del artículo JISA}\label{fig:jisa-portada}
\end{figure}
\FloatBarrier

% Página 1
\begin{figure}[H]
    \centering
    \begin{tcolorbox}[
        colback=white,
        colframe=gray!50,
        boxrule=1pt,
        arc=2pt,
        boxsep=5pt,
        left=3pt,
        right=3pt,
        top=3pt,
        bottom=3pt,
        drop shadow
    ]
        \includegraphics[width=0.95\textwidth,keepaspectratio]{apendices/JISA/pagina_1.png}
    \end{tcolorbox}
    \caption{Artículo JISA --- Página 1}\label{fig:jisa-pagina-1}
\end{figure}
\FloatBarrier% Página 2
\begin{figure}[H]
    \centering
    \begin{tcolorbox}[
        colback=white,
        colframe=gray!50,
        boxrule=1pt,
        arc=2pt,
        boxsep=5pt,
        left=3pt,
        right=3pt,
        top=3pt,
        bottom=3pt,
        drop shadow
    ]
        \includegraphics[width=0.95\textwidth,keepaspectratio]{apendices/JISA/pagina_2.png}
    \end{tcolorbox}
    \caption{Artículo JISA --- Página 2}\label{fig:jisa-pagina-2}
\end{figure}
\FloatBarrier% Página 3
\begin{figure}[H]
    \centering
    \begin{tcolorbox}[
        colback=white,
        colframe=gray!50,
        boxrule=1pt,
        arc=2pt,
        boxsep=5pt,
        left=3pt,
        right=3pt,
        top=3pt,
        bottom=3pt,
        drop shadow
    ]
        \includegraphics[width=0.95\textwidth,keepaspectratio]{apendices/JISA/pagina_3.png}
    \end{tcolorbox}
    \caption{Artículo JISA --- Página 3}\label{fig:jisa-pagina-3}
\end{figure}
\FloatBarrier% Página 4
\begin{figure}[H]
    \centering
    \begin{tcolorbox}[
        colback=white,
        colframe=gray!50,
        boxrule=1pt,
        arc=2pt,
        boxsep=5pt,
        left=3pt,
        right=3pt,
        top=3pt,
        bottom=3pt,
        drop shadow
    ]
        \includegraphics[width=0.95\textwidth,keepaspectratio]{apendices/JISA/pagina_4.png}
    \end{tcolorbox}
    \caption{Artículo JISA --- Página 4}\label{fig:jisa-pagina-4}
\end{figure}
\FloatBarrier% Página 5
\begin{figure}[H]
    \centering
    \begin{tcolorbox}[
        colback=white,
        colframe=gray!50,
        boxrule=1pt,
        arc=2pt,
        boxsep=5pt,
        left=3pt,
        right=3pt,
        top=3pt,
        bottom=3pt,
        drop shadow
    ]
        \includegraphics[width=0.95\textwidth,keepaspectratio]{apendices/JISA/pagina_5.png}
    \end{tcolorbox}
    \caption{Artículo JISA --- Página 5}\label{fig:jisa-pagina-5}
\end{figure}
\FloatBarrier% Macro para crear páginas del artículo de forma automática
% Páginas 6-34
\newcounter{jisampage}
\setcounter{jisampage}{6}
\loop%
    \begin{figure}[H]
        \centering
        \begin{tcolorbox}[
            colback=white,
            colframe=gray!50,
            boxrule=1pt,
            arc=2pt,
            boxsep=5pt,
            left=3pt,
            right=3pt,
            top=3pt,
            bottom=3pt,
            drop shadow
        ]
            \includegraphics[width=0.95\textwidth,keepaspectratio]{apendices/JISA/pagina_\thejisampage.png}
        \end{tcolorbox}
        \caption{Artículo JISA --- Página \thejisampage}\label{fig:jisa-pagina-\thejisampage}
    \end{figure}
    \FloatBarrier\stepcounter{jisampage}
    \ifnum\value{jisampage}<35
\repeat%


\FloatBarrier\section{CEIFI 2025}
\foreach \n in {1,...,16} {
    \begin{figure}[H]
        \centering
        \fbox{\includegraphics[width=0.9\textwidth]{apendices/CEIFI/\n.png}}
        \caption{Slide \n de CEIFI}
    \end{figure}
}

\FloatBarrier\section{GRID 2025-I }
\foreach \n in {1,...,13} {
    \begin{figure}[H]
        \centering
        \fbox{\includegraphics[width=0.9\textwidth]{apendices/GRID-2025/\n.png}}
        \caption{Slide \n de GRID 2025-I}
    \end{figure}
}

\FloatBarrier\section{Poster GRID 2024-II}
\begin{figure}[H]
    \centering
    \fbox{\includegraphics[width=0.9\textwidth]{apendices/GRID-2024/image.png}}
    \caption{Póster de GRID 2024-II}
\end{figure}

\end{document}