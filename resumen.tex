\ChapterImageStar[cap:resumen]{Resumen}{./images/fondo.png}\label{cap:resumen}
\mbox{}\\
La presente tesis aborda la especificación de una solución arquitectónica basada en tecnologías de virtualización por contenedores (VBC) para el Grupo de Investigación en Redes, Información y Distribución (GRID) de la Universidad del Quindío. El trabajo se desarrolla en respuesta a la necesidad del GRID de ampliar su portafolio de servicios tecnológicos mediante el uso de instancias computacionales más ligeras y eficientes, complementando su infraestructura existente basada en máquinas virtuales gestionadas con XCP-ng. El estudio inicia con una caracterización exhaustiva del GRID, que incluye un análisis de sus stakeholders, misión, visión, infraestructura tecnológica y servicios actuales y esperados. Se identifica una oportunidad clara para incorporar VBC como medio para fortalecer los pilares misionales de docencia, investigación y extensión, beneficiando especialmente a estudiantes y docentes del programa de Ingeniería de Sistemas y Computación. Metodológicamente, la investigación se estructura en varias fases: primero, se realiza un estudio de mapeo sistemático (SMS) para identificar y clasificar tecnologías VBC relevantes en dominios de TI afines al GRID. Este proceso incluyó búsquedas en bases de datos académicas y la aplicación de criterios de inclusión/exclusión, resultando en la selección de múltiples herramientas como Docker, Podman, LXC, Containerd, entre otras. Posteriormente, se lleva a cabo un benchmarking técnico para evaluar el rendimiento de estas tecnologías en términos de consumo de CPU, memoria, tiempo de arranque, throughput de red y latencia de E/S. Con base en los resultados del benchmarking y mediante la aplicación de la metodología de Análisis de Decisiones y Resolución (DAR) del modelo CMMI, se selecciona Containerd como la tecnología VBC más adecuada para el contexto del GRID, destacando por su eficiencia, integración con Kubernetes, licencia permisiva y amplia documentación. Adicionalmente, se evalúan distintas opciones de motores de orquestación, donde K3S emerge como la alternativa más viable. Finalmente, se propone un diseño arquitectónico modelado en Archimate, que incluye vistas de negocio, aplicación y tecnología, así como un modelo por capas que articula la infraestructura existente con la capa de virtualización (Containerd + K3S) y la capa de aplicación. La solución busca proporcionar un servicio escalable y mantenible para la provisión de entornos computacionales basados en contenedores, alineado con las necesidades académicas y de investigación del GRID. El trabajo concluye con la implementación de un producto mínimo viable (PMV) y su validación, sentando las bases para la adopción efectiva de VBC en el grupo de investigación y ofreciendo un referente metodológico para decisiones tecnológicas similares en contextos académicos e institucionales.

\ChapterImageStar[cap:abstract]{Abstract}{./images/fondo.png}\label{cap:abstract}
\mbox{}\\
This thesis addresses the specification of an architectural solution based on container virtualization technologies (VBC) for the Research Group on Networks, Information, and Distribution (GRID) at the University of Quindío. The work is developed in response to GRID's need to expand its portfolio of technological services by using lighter and more efficient computational instances, complementing its existing infrastructure based on virtual machines managed with XCP-ng. The study begins with a comprehensive characterization of GRID, including an analysis of its stakeholders, mission, vision, technological infrastructure, and current and expected services. A clear opportunity is identified to incorporate VBC as a means to strengthen the mission pillars of teaching, research, and outreach, especially benefiting students and faculty in the Systems and Computing Engineering program. Methodologically, the research is structured in several phases: first, a systematic mapping study (SMS) is conducted to identify and classify relevant VBC technologies in IT domains related to GRID. This process included searches in academic databases and the application of inclusion/exclusion criteria, resulting in the selection of multiple tools such as Docker, Podman, LXC, Containerd, among others. Subsequently, a technical benchmarking is carried out to evaluate the performance of these technologies in terms of CPU consumption, memory, boot time, network throughput, and I/O latency. Based on the benchmarking results and through the application of the Decision Analysis and Resolution (DAR) methodology from the CMMI model, Containerd is selected as the most suitable VBC technology for the GRID context, standing out for its efficiency, integration with Kubernetes, permissive license, and extensive documentation. Additionally, various orchestration engine options are evaluated, with K3S emerging as the most viable alternative. Finally, an architectural design modeled in Archimate is proposed, which includes business, application, and technology views, as well as a layered model that articulates the existing infrastructure with the virtualization layer (Containerd + K3S) and the application layer. The solution aims to provide a scalable and maintainable service for the provision of container-based computational environments, aligned with the academic and research needs of GRID.