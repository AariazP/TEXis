\ChapterImageStar[cap:resumen]{Resumen}{./images/fondo.png}\label{cap:resumen}
\mbox{}\\
La presente tesis desarrolla la especificación de una solución arquitectónica basada en tecnologías de virtualización por contenedores (\VBC) para el Grupo de Investigación en Redes, Información y Distribución (\GRID) de la Universidad del Quindío. El estudio surge como respuesta a la necesidad de ampliar el portafolio de servicios tecnológicos mediante instancias computacionales más ligeras que complementen la infraestructura existente basada en máquinas virtuales. 
Metodológicamente, la investigación comprende: (a) un estudio de mapeo sistemático (\SMS) para identificar tecnologías de \VBC; (b) un \textit{benchmarking} técnico de consumo de \CPU, memoria, tiempo de arranque, \textit{throughput} y latencia de E/S; y (c) la aplicación de la metodología de Análisis de Decisiones y Resolución (\DAR) del modelo \CMMI. Los resultados señalan a Containerd como la tecnología más adecuada, mientras que K3S se identificó como el motor de orquestación más viable. 
Finalmente, se propone un diseño arquitectónico modelado en Archimate que articula la infraestructura existente con la capa de virtualización (Containerd + K3S) y la capa de aplicación. La solución busca ofrecer un servicio escalable y mantenible alineado con las necesidades académicas e investigativas del \GRID. El trabajo concluye con la implementación de un producto mínimo viable (\PMV), que valida la pertinencia de la propuesta y establece un referente metodológico para decisiones tecnológicas en contextos académicos e institucionales. 

\ChapterImageStar[cap:abstract]{Abstract}{./images/fondo.png}\label{cap:abstract}
\mbox{}\\
This thesis develops the specification of an architectural solution based on container virtualization technologies (CVT) for the Research Group on Networks, Information, and Distribution (\GRID) at the University of Quindío. The study arises in response to the need to expand the technological service portfolio through lighter computational instances that complement the existing infrastructure, which is based on virtual machines.
Methodologically, the research includes: (a) a systematic mapping study (\SMS) to identify CVT technologies; (b) a technical benchmarking of CPU consumption, memory, boot time, throughput, and I/O latency; and (c) the application of the Decision Analysis and Resolution (\DAR) methodology from the CMMI model. The results indicate Containerd as the most suitable technology, while K3S was identified as the most viable orchestration engine.
Finally, an architectural design modeled in Archimate is proposed, which integrates the existing infrastructure with the virtualization layer (Containerd + K3S) and the application layer. The solution aims to offer a scalable and maintainable service aligned with \GRID’s academic and research needs. The work concludes with the implementation of a minimum viable product (MVP), which validates the relevance of the proposal and establishes a methodological reference for technological decisions in academic and institutional contexts.