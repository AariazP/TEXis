\begin{minted}[frame=lines,
    fontsize=\scriptsize,
    breaklines]{bash}
#!/bin/bash

# ---------------------------------------------------------------
# Script: cluster-bootup.sh
# Descripción:
#   Este script crea e inicializa un cluster K3s en XCP-ng, 
#   desplegando servidores master, workers, y opcionalmente 
#   un load balancer y un servidor NAT de backup. Configura 
#   hostnames, IPs, HAProxy (si hay load balancer) y une 
#   todos los nodos al cluster.
# ---------------------------------------------------------------

# Valores por defecto
NUM_WORKERS=2
NUM_MASTERS=1
BACKUP=0
HAS_LB=0

# Procesar opciones:
# -w NUM_WORKERS  → Número de nodos worker
# -m NUM_MASTERS  → Número de nodos master
# -b              → Crear NAT server backup
while getopts "w:m:b" opt; do
  case $opt in
    w) NUM_WORKERS=$OPTARG ;;
    m) NUM_MASTERS=$OPTARG ;;
    b) BACKUP=1 ;;
    *) echo "Usage: $0 [-w NUM_WORKERS] [-m NUM_MASTERS] [-b NAT server backup?]" >&2
       exit 1 ;;
  esac
done

# Si hay más de un master, se necesitará un load balancer
[[ $NUM_MASTERS > 1 ]] && HAS_LB=1 

# Crear servidores NAT (Master y opcionalmente Backup)
[[ $BACKUP > 0 ]] &&  boot-servers -b ||  boot-servers 

# Esperar 3 segundos para que arranquen
sleep 3

# Definir nombres y cantidad de nodos a crear
NAMES=( master worker )
VALUES=( $NUM_MASTERS $NUM_WORKERS )

# Si hay load balancer, agregarlo al arreglo
if [[ $HAS_LB > 0 ]]; then
    NAMES=( load-balancer master worker )
    VALUES=( 1 $NUM_MASTERS $NUM_WORKERS )
fi

# Crear VMs según tipo y cantidad
for idx in $( seq 0 ${#NAMES[@]} | head -n -1 ); do
    name=${NAMES[$idx]}
    value=${VALUES[$idx]} 

    # Crear cada VM del tipo correspondiente
    for i in $( seq 1 $value ); do
        bootvm $name$i
    done

    sleep 2

    # Configurar hostname y /etc/hosts en cada VM
    for i in $( seq 1 $value ); do
        echo "=== setting up $name$i hostname ==="
        
        # Esperar a que la VM tenga IP
        while [ -z $VM_IP ]; do 
            VM_IP=$( get-ip $name$i )
            sleep 1
        done

        # Configurar hostname remoto
        ssh -i ~/.ssh/id_internal_vm root@$VM_IP \
            "hostnamectl set-hostname $name$i; hostname; \
             echo '127.0.0.1 $name$i' >> /etc/hosts"

        VM_IP=""
    done
done

# Obtener IP del master1
MASTER_IP=$( get-ip master1 )

# Obtener IP del load balancer si existe
[ $HAS_LB -gt 0 ] && LB_IP=$( get-ip load-balancer1 )

# Instalar K3s en master1
if [ $HAS_LB -gt 0 ]; then
    ssh -i ~/.ssh/id_internal_vm root@$MASTER_IP \
        "curl -sfL https://get.k3s.io | sh -s - server --cluster-init --tls-san ${LB_IP}"
else
    ssh -i ~/.ssh/id_internal_vm root@$MASTER_IP \
        "curl -sfL https://get.k3s.io | sh -"
fi

# Obtener token de cluster K3s para unir nodos
while [ -z $TOKEN ]; do
    TOKEN=$( ssh -i ~/.ssh/id_internal_vm root@$MASTER_IP \
             "sudo cat /var/lib/rancher/k3s/server/node-token" )
done

# Configurar HAProxy si hay load balancer
if [[ $HAS_LB > 0 ]]; then

    # Instalar HAProxy
    ssh -i ~/.ssh/id_internal_vm root@$LB_IP "apt install haproxy -y"

    # Configuración básica de HAProxy
    config="frontend k3s-api
        bind *:6443
        default_backend k3s-masters

    backend k3s-masters
        balance roundrobin
    "

    # Agregar cada master al backend de HAProxy
    for i in $( seq 1 $NUM_MASTERS ); do
        ip=$( get-ip master$i )
        config+="    server master${i} ${ip}:6443 check
"
    done

    # Mostrar configuración
    echo "_____________________"
    echo "$config"
    echo "_____________________"

    # Subir configuración al LB y reiniciar servicio
    ssh -i ~/.ssh/id_internal_vm root@$LB_IP \
        "cat > /etc/haproxy/haproxy.cfg" <<< "$config"
    ssh -i ~/.ssh/id_internal_vm root@$LB_IP "sudo systemctl restart haproxy"

fi

# Arreglos de nombres, cantidad y modos para unir nodos
NAMES=( master worker )
VALUES=( $NUM_MASTERS $NUM_WORKERS )
MODES=( "-s - server" "-" )

# Determinar IP a la que los nodos se unirán
JOIN_IP=${LB_IP}
[ $HAS_LB -eq 0 ] && JOIN_IP=$MASTER_IP

# Unir nodos al cluster K3s
for idx in {0..1}; do
    name=${NAMES[$idx]}
    value=${VALUES[$idx]}
    mode=${MODES[$idx]}

    for i in $( seq 1 $value ); do
        # Saltar master1 que ya está inicializado
        [[ ${name}${i} == "master1" ]] && continue

        echo "=== joining host $name$i to cluster ==="

        # Esperar a que la VM tenga IP
        while [ -z $VM_IP ]; do 
            VM_IP=$( get-ip $name$i )
            sleep 1
        done

        # Ejecutar script K3s para unir nodo al cluster
        ssh -i ~/.ssh/id_internal_vm root@$VM_IP \
            "curl -sfL https://get.k3s.io | K3S_URL=https://${JOIN_IP}:6443 K3S_TOKEN=${TOKEN} sh ${mode}"

        VM_IP=""
    done
done
\end{minted}