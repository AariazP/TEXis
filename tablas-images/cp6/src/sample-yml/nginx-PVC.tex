El recurso \texttt{PersistentVolumeClaim} (PVC) en Kubernetes actúa como una solicitud de almacenamiento por parte de una aplicación dentro del clúster. En este caso, el PVC denominado \texttt{pvc-sample} solicita un volumen con capacidad de \texttt{1Gi}, en modo de acceso \texttt{ReadWriteOnce}, lo cual permite que solo un nodo lo monte con permisos de lectura y escritura al mismo tiempo. Este reclamo hace uso de la clase de almacenamiento \texttt{manual}, lo que asegura su vinculación con el \texttt{PersistentVolume} previamente definido que comparte las mismas características.

\begin{minted}[frame=lines,
    fontsize=\scriptsize,
    breaklines]{yaml}
apiVersion: v1
kind: PersistentVolumeClaim
metadata:
  name: pvc-sample
spec:
  accessModes:
    - ReadWriteOnce
  storageClassName: manual
  resources:
    requests:
      storage: 1Gi
\end{minted}