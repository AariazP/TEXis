El recurso \texttt{Service} en Kubernetes permite exponer aplicaciones desplegadas dentro del clúster, asegurando una comunicación estable entre pods y otros componentes. En este caso, el servicio denominado \texttt{nginx-service} está configurado con el tipo \texttt{ClusterIP}, lo que significa que será accesible únicamente desde dentro del clúster. Su selector utiliza la etiqueta \texttt{app: nginx} para dirigir el tráfico hacia los pods gestionados por el \texttt{Deployment} de Nginx. El puerto expuesto es el \texttt{80}, el cual redirige al mismo \texttt{targetPort} dentro de los contenedores.

\begin{minted}[frame=lines,
    fontsize=\scriptsize,
    breaklines]{yaml}
apiVersion: v1
kind: Service
metadata:
  name: nginx-service
spec:
  selector:
    app: nginx
  ports:
    - protocol: TCP
      port: 80
      targetPort: 80
  type: ClusterIP
\end{minted}