\begin{minted}[frame=lines,
    fontsize=\scriptsize,
    breaklines]{bash}
#!/bin/bash

# ---------------------------------------------------------------
# Script: bootup.sh
# Descripción:
#   Este script fue creado inicialmente para desplegar y configurar 
#   máquinas virtuales en XCP-ng de manera manual antes de contar 
#   con plantillas. Actualmente no se usa, pero se conserva como 
#   alternativa en caso de perder las plantillas predefinidas.
# ---------------------------------------------------------------

# Listar hosts disponibles
xe host-list

# Obtener UUID del host principal y exportarlo como variable de entorno
export host_uuid="$( xe host-list | grep -oP "[\d\w-]+" | grep -oP ".{20,}" | head -n 1 )"
echo "host_uuid is this : _____________ $host_uuid ______"

# Listar todos los storage repositories (SR)
xe sr-list
xe sr-list name-label="Local storage"

# Listar plantillas disponibles y filtrar Debian
xe template-list | grep Debian

# Definir nombre de la VM (por defecto "xeclivm" si no se pasa argumento)
export vm_name="${1:-xeclivm}"

# Crear nueva VM a partir de la plantilla "Debian Bookworm 12"
xe vm-install template="Debian Bookworm 12" new-name-label="${vm_name}"

# Obtener el UUID de la VM recién creada
export vm_uuid=$( xe vm-list | grep -B 1 "$vm_name" | grep -oP "[\d\w-]{20,}" )
echo "vm_uuid is this : _______________ $vm_uuid __________"

# Preparar directorio para ISO locales
mkdir -p /var/opt/xen/ISO_Store
cd /var/opt/xen/ISO_Store

# Montar un SR de tipo ISO
xe-mount-iso-sr /run/sr-mount/bedd1ebf-c12c-875d-960c-257de20dec45/

# Crear un nuevo SR para almacenar ISOs locales
xe sr-create host-uuid=$host_uuid name-label=LOCAL_ISO type=iso \
  device-config:location=/var/opt/xen/ISO_Store \
  device-config:legacy_mode=true content-type=iso

# Listar SR de tipo ISO
xe sr-list type=iso

# Obtener UUID del SR de ISOs recién creado
export sr_iso_uuid=$(xe sr-list type=iso | grep -B 1 LOCAL_ISO | head -n 1 | grep -oP "[\d\w-]+-[\w\d]+")

# Forzar escaneo del SR de ISOs
xe sr-scan uuid=$sr_iso_uuid

# Definir nombre del ISO de instalación
export cd_iso="debian-12.11.0-amd64-netinst.iso"

# Obtener UUID del disco de la VM recién creada
export vm_disk_uuid=$( xe vm-disk-list uuid=$vm_uuid | grep -B 2 "Local storage" | head -n 1 | grep -oP "[\d\w-]+-[\d\w]+" )

# Obtener UUID de la red principal (xenbr0)
export network_uuid=$( xe network-list | grep -B 3 xenbr0 | head -n 1 | grep -oP "[\d\w-]+-[\d\w]+" )

# Insertar ISO de instalación en la VM
xe vm-cd-add uuid=$vm_uuid cd-name="$cd_iso" device=1

# Configurar política de arranque de la VM
xe vm-param-set HVM-boot-policy="BIOS order" uuid=$vm_uuid

# Crear interfaz de red para la VM
xe vif-create vm-uuid=$vm_uuid network-uuid=$network_uuid device=0

# Configurar memoria de la VM (mínimo, máximo y dinámico)
xe vm-memory-limits-set dynamic-max=2048MiB dynamic-min=2048MiB \
  static-max=2048MiB static-min=512MiB uuid=$vm_uuid

# Redimensionar el disco de la VM a 20 GiB
xe vdi-resize uuid=$vm_disk_uuid disk-size=20GiB

# Configurar orden de arranque: disco y CD
xe vm-param-set uuid=$vm_uuid HVM-boot-params:order=dc

# Crear un nuevo dispositivo virtual de CD (ejemplo con UUID fijo de VDI)
xe vbd-create \
  vm-uuid=$vm_uuid \
  device=3 \
  type=CD \
  vdi-uuid=8dba583e-57df-44eb-a954-c1db24d22cc0 \
  mode=RO \
  bootable=true

# Desactivar Secure Boot y forzar emulación tradicional de QEMU
xe vm-param-set uuid=$vm_uuid platform:secureboot=false
xe vm-param-set uuid=$vm_uuid platform:device-model=qemu-traditional

# Ajustar parámetros de arranque a BIOS
xe vm-param-set uuid=$vm_uuid HVM-boot-policy="BIOS order"
xe vm-param-set uuid=$vm_uuid HVM-boot-params:firmware=bios

# Expulsar cualquier CD-ROM previo
xe vm-cd-eject uuid=$vm_uuid

# Insertar ISO de Guest Tools
xe vm-cd-insert uuid=$vm_uuid cd-name=guest-tools.iso

# Insertar ISO de Debian en la VM (usando nombre pasado como argumento $1)
xe vm-cd-insert vm=$1 cd-name=debian-12.11.0-amd64-netinst.iso

# Iniciar la VM
xe vm-start uuid=$vm_uuid
\end{minted}