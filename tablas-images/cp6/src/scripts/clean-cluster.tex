\begin{minted}[frame=lines,
    fontsize=\scriptsize,
    breaklines]{bash}
#!/bin/bash

# ---------------------------------------------------------------
# Script: clean-cluster.sh
# Descripción:
#   Este script elimina todas las máquinas virtuales (VMs) de un 
#   cluster en XCP-ng, excluyendo dominios del sistema. Puede 
#   ejecutarse de manera interactiva o automática usando la opción -y.
# ---------------------------------------------------------------

# Variable que indica si se debe asumir "sí" automáticamente
ASSUME_YES=false

# ---------------------------------------------------------------
# Procesar opción:
#   -y  → Asumir "sí" automáticamente para borrar las VMs
# ---------------------------------------------------------------
while getopts "y" opt; do
  case $opt in
    y)
      ASSUME_YES=true  # Activar borrado automático
      ;;
    *)
      echo "Usage: $0 [-y]"
      exit 1
      ;;
  esac
done

# Si no se asumió "sí", pedir confirmación al usuario
if [ "$ASSUME_YES" = false ]; then
  read -p "Are you really sure you want to clean cluster. (y/N): " resp
  [[ ! $resp =~ (y|Y|yes|Yes|YES) ]] && exit 0
fi

# Obtener lista de VMs a eliminar, excluyendo dominios del sistema
vms=$(xe vm-list | grep name-label | grep -v "domain" | grep -oP "(?<=: ).+")

# Iterar sobre cada VM encontrada
for vm in $vms; do
  # Obtener UUID de la VM
  UUID=$(xe vm-list name-label="$vm" --minimal | tr ',' ' ')
  
  # Apagar la VM de forma forzada en segundo plano
  xe vm-shutdown uuid=$UUID force=true &
  wait $!  # Esperar a que termine el apagado
  
  # Eliminar la VM del sistema
  xe vm-destroy uuid=$UUID
done
\end{minted}