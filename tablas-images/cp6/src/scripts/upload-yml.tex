El siguiente script recibe como argumento el nombre de un nodo y una lista de archivos YAML locales. Primero obtiene la dirección IP del nodo especificado mediante el script auxiliar \texttt{get-ip}. Posteriormente, transfiere los archivos al nodo usando \texttt{scp} y ejecuta en la máquina remota el comando \texttt{kubectl apply -f} para aplicar los manifiestos en el clúster de Kubernetes. Finalmente, elimina los archivos temporales en el nodo para mantener la limpieza del sistema.

\begin{minted}[frame=lines,
    fontsize=\scriptsize,
    breaklines]{bash}
#!/bin/bash

# ---------------------------------------------------------------
# Script: upload_yaml.sh
# Descripción:
#   Este script transfiere archivos YAML a un nodo específico 
#   en un clúster y aplica los manifiestos con kubectl.
#   Pasos:
#     1. Obtiene la IP del nodo usando `get-ip`.
#     2. Copia cada archivo YAML al nodo vía `scp`.
#     3. Ejecuta `kubectl apply -f` en el nodo para aplicar
#        los cambios y luego elimina el archivo remoto.
# ---------------------------------------------------------------

# Redirigir errores a /dev/null para evitar mensajes innecesarios.
exec 2> /dev/null

# Guardar el primer argumento como nombre del nodo
NODE="$1"

# Eliminar el primer argumento (nombre del nodo) y dejar solo los archivos
shift

# Obtener la dirección IP del nodo a partir de su nombre
IP=$( get-ip "$NODE" )

# Recorrer cada archivo recibido como argumento
for FILE in "$@"; do
    # Obtener solo el nombre del archivo (sin ruta)
    FILENAME=$( basename "$FILE" )

    # Copiar el archivo al nodo remoto en el directorio actual (~)
    scp -i ~/.ssh/id_internal_vm "$FILE" root@$IP:./

    # Conectarse al nodo y aplicar el manifiesto con kubectl
    # Luego eliminar el archivo del nodo para mantener limpieza
    ssh -i ~/.ssh/id_internal_vm root@$IP \
        "kubectl apply -f $FILENAME && rm $FILENAME"
done
\end{minted}