\begin{minted}[frame=lines,
    fontsize=\scriptsize,
    breaklines]{bash}
#!/bin/bash

# ---------------------------------------------------------------
# Script: boot-servers.sh
# Descripción:
#   Este script crea e inicia las máquinas virtuales principales 
#   de red NAT en XCP-ng. Puede crear un servidor NAT principal 
#   (Master) y, opcionalmente, un servidor de respaldo (Backup) 
#   usando plantillas predefinidas.
# ---------------------------------------------------------------

# Variable que indica si se debe crear el servidor de backup
BACKUP=0

# Procesar opciones:
#   -b  → crea el servidor NAT de backup adicional
while getopts "b" opt; do
  case $opt in
    b) BACKUP=1 ;;  # Activar creación del backup
    *) echo "Usage: $0 [-w NUM_WORKERS] [-m NUM_MASTERS] [-b NAT server backup?]" >&2
       exit 1 ;;
  esac
done

# Crear el servidor NAT principal (Master) a partir de la plantilla "nat-server"
MASTER_UUID=$(xe vm-install template="nat-server" new-name-label="NAT-Master")

# Si se indicó la opción -b, crear también el servidor de backup
[[ $BACKUP > 0 ]] && BACKUP_UUID=$(xe vm-install \
     template="nat-server" new-name-label="NAT-Backup")

# Iniciar la VM principal (Master)
xe vm-start uuid=$MASTER_UUID

# Esperar 2 segundos para dar tiempo a que arranque la Master
sleep 2

# Si se creó la VM de backup, iniciarla también
[[ $BACKUP > 0 ]] && xe vm-start uuid=$BACKUP_UUID
\end{minted}