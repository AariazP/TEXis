El script \texttt{get-ip.sh} tiene como finalidad obtener de manera automática la dirección IP de una \VM\, partiendo únicamente de su nombre. Para ello, se conecta al servidor \NAT\ a través de \texttt{SSH} utilizando una llave privada, consulta la tabla ARP mediante el comando \texttt{ip n}, y filtra las entradas que coinciden con las direcciones MAC asociadas a la \VM. De esta forma, el script permite resolver dinámicamente la IP de la \VM.

\begin{minted}[frame=lines,
    fontsize=\scriptsize,
    breaklines]{bash}
#!/bin/bash

# ---------------------------------------------------------------
# Script: get-ip.sh
# Descripción:
#   Este script obtiene la dirección IP de una máquina virtual 
#   (VM) en XCP-ng a partir de su nombre.
# ---------------------------------------------------------------

# Verificación de argumentos:
# Si no se pasa al menos un argumento (nombre de la VM), se muestra 
# el uso correcto del script y se termina.
if [ $# -lt 1 ]
then
   echo "usage: get-ip <vm-name>"
   exit 0
fi

# Caso especial: 
# Si el nombre de la VM es "NAT-Master", se devuelve una IP fija 
# sin necesidad de consultar nada en XCP-ng.
if [[ "$1" == "NAT-Master" ]]
then
   echo "172.30.29.2"
   exit 0
fi

# Obtiene el UUID de la VM en XCP-ng a partir del nombre de la VM.
VM_UUID=$( xe vm-list name-label="$1" --minimal )

# Obtiene las direcciones MAC de las interfaces de red (VIFs) 
# asociadas a la VM, separadas por coma.
VM_MAC=$( xe vif-list vm-uuid="$VM_UUID" params=MAC --minimal )

# Convierte la lista separada por comas en un arreglo de MACs. 
# Se asumen máximo dos interfaces de red.
MACS=( $( echo $VM_MAC | cut -d, -f1 ) $( echo $VM_MAC | cut -d, -f2 ) )

# Redirige los errores a /dev/null para evitar mostrar mensajes 
# de advertencia de ssh u otros.
exec 2> /dev/null

# Se conecta al servidor NAT (172.30.29.2) vía SSH usando una llave 
# privada, y ejecuta el comando `ip n` (vecinos en ARP/NDP).
# Luego filtra las líneas que contengan alguna de las MACs de la VM 
# y extrae la primera columna (la IP asociada).
VM_IP=$( ssh -i ~/.ssh/id_nat_server debian@172.30.29.2 \
        "ip n | fgrep -e ${MACS[0]} -e ${MACS[1]}" | cut -d" " -f1 )

# Muestra la IP obtenida sin salto de línea adicional al final.
echo -n $VM_IP
\end{minted}
