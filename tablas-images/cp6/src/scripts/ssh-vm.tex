\begin{minted}[frame=lines,
    fontsize=\scriptsize,
    breaklines]{bash}
#!/bin/bash

# ---------------------------------------------------------------
# Script: ssh-vm.sh
# Descripción:
#   Este script permite conectarse vía SSH a una máquina virtual 
#   en XCP-ng usando su nombre como referencia. Se puede especificar:
#   - Si la conexión debe hacerse como root.
#   - Un comando a ejecutar de forma remota en la VM.
# ---------------------------------------------------------------

# Variables iniciales
ROOT=0     # Indica si se usará el usuario root (1 = sí, 0 = no)
CMD=""     # Comando opcional a ejecutar en la VM
VM=""      # Nombre de la VM a la que se conectará

# Procesar opciones iniciales: 
#   -r  → usar root como usuario.
#   -c  → comando a ejecutar.
while getopts "rc:" opt; do
  case "$opt" in
    r) ROOT=1 ;;                # Activar conexión como root
    c) CMD="$OPTARG" ;;          # Guardar comando remoto
    *) echo "Usage: $0 [-r] <vm-name> [-c command]" >&2; exit 1 ;;
  esac
done
shift $((OPTIND - 1))            # Avanzar en la lista de argumentos

# Validar que al menos se haya pasado el nombre de la VM
if [ $# -lt 1 ]; then
  echo "Usage: $0 [-r] <vm-name> [-c command]" >&2
  exit 1
fi

# Guardar nombre de la VM y avanzar
VM="$1"
shift 1

# Reiniciar índice de opciones para procesar posibles parámetros -c después del nombre de la VM
OPTIND=1
while getopts "c:" opt; do
  case "$opt" in
    c) CMD="$OPTARG" ;;
  esac
done
shift $((OPTIND - 1))

# Obtener la dirección IP de la VM usando el script auxiliar `get-ip`
# (debe estar en el PATH del sistema).
IP=$( get-ip "$VM" 2>/dev/null )

# Conexión SSH
if [ "$ROOT" -eq 1 ]; then
  # Si se especifica conexión como root
  if [ -n "$CMD" ]; then
    # Ejecutar comando remoto como root
    ssh -i ~/.ssh/id_internal_vm root@"$IP" "$CMD" 2>/dev/null
  else
    # Abrir sesión interactiva como root
    ssh -i ~/.ssh/id_internal_vm root@"$IP" 2>/dev/null
  fi
else
  # Conexión como usuario por defecto (depende de la configuración SSH de la VM)
  if [ -n "$CMD" ]; then
    # Ejecutar comando remoto sin root
    ssh "$IP" "$CMD" 2>/dev/null
  else
    # Abrir sesión interactiva sin root
    ssh "$IP" 2>/dev/null
  fi
fi
\end{minted}