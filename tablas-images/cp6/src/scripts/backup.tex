El siguiente script implementa un mecanismo automatizado para realizar respaldos de diferentes tipos de recursos en un clúster de Kubernetes gestionado desde XCP-ng. En particular, soporta tres modos de operación: (i) respaldo de un \texttt{PersistentVolumeClaim (PVC)}, (ii) respaldo de rutas arbitrarias dentro de un \texttt{Pod}, y (iii) respaldo de carpetas en el \texttt{nodo host}. Para ello, el script crea de manera dinámica un \texttt{Pod} temporal (\textit{exporter}) que monta el recurso a respaldar, copia los datos hacia el nodo maestro del clúster y los empaqueta en un archivo \texttt{tar} identificado con la fecha y hora del proceso. Finalmente, el archivo resultante se transfiere de forma segura mediante \texttt{scp} al host que ejecuta el script, permitiendo así una copia local de los datos.

\begin{minted}[frame=lines,
    fontsize=\scriptsize,
    breaklines]{bash}
#!/bin/bash

# ---------------------------------------------------------------
# Script: backup-resource.sh
# Descripción:
#   Este script permite realizar respaldos de recursos en un 
#   clúster de Kubernetes desde un entorno XCP-ng. Soporta tres
#   modos de respaldo:
#     - PVC    : copia el contenido de un PersistentVolumeClaim.
#     - Pod    : copia una ruta arbitraria desde un Pod.
#     - Folder : copia una carpeta directamente desde el nodo host.
#
#   El resultado se empaqueta en un archivo .tar con sello temporal
#   y se transfiere mediante scp al sistema local.
# ---------------------------------------------------------------

# Limpieza automática: elimina el archivo temporal si se interrumpe el script (Ctrl+C)
trap "rm -f exporter_file.yaml" SIGINT

# Modo por defecto: pvc
MODE="pvc"
TARGET=""
POD=""
FOLDER=""

# Procesar argumentos con getopts:
#   -p <pod>     → respaldo desde un pod:ruta
#   -f <folder>  → respaldo desde una carpeta del host
while getopts "p:f:" opt; do
  case "$opt" in
    p) MODE="pod"; POD="$OPTARG" ;;          # backup desde un pod específico
    f) MODE="folder"; FOLDER="$OPTARG" ;;    # backup desde carpeta en nodo
  esac
done
shift $((OPTIND-1))

# Seleccionar nodo maestro (usando helper get-vm)
NODE="$( get-vm -a -n 1 master )"

# Primer argumento: nombre del PVC a respaldar
PVC="$1"

# Sello temporal para identificar los respaldos
DATE=$( date +%Y-%m-%d-%H-%M-%S-%s )

# ---------------------------------------------------------------
# Modo PVC: se crea un pod temporal que monta el PVC y exporta su contenido
# ---------------------------------------------------------------
if [[ "$MODE" == "pvc" ]]; then
  # Crear definición YAML para pod temporal "exporter"
  cat <<EOF > exporter_file.yaml
apiVersion: v1
kind: Pod
metadata:
  name: exporter
spec:
  containers:
  - name: exporter
    image: alpine
    command: ["sleep", "3600"]
    volumeMounts:
    - mountPath: /data
      name: export-pvc
  volumes:
  - name: export-pvc
    persistentVolumeClaim:
      claimName: ${PVC}
EOF

  # Subir definición YAML al nodo maestro
  upload_yaml ${NODE} exporter_file.yaml

  # Esperar hasta que el pod esté en estado Running
  while [[ $(ssh-vm -r ${NODE} -c "kubectl get pod/exporter -o jsonpath='{.status.phase}'") != "Running" ]]; do
    sleep 1
  done

  # Copiar datos del pod y empaquetar en archivo tar
  FILENAME="/tmp/backup-${DATE}-pvc.tar"
  ssh-vm -r ${NODE} -c "kubectl cp exporter:/data ./backup-${DATE}"
  ssh-vm -r ${NODE} -c "kubectl delete pod/exporter"
  ssh-vm -r ${NODE} -c "tar -cvf ${FILENAME} ./backup-${DATE}"

# ---------------------------------------------------------------
# Modo POD: copia una ruta arbitraria desde un Pod específico
# ---------------------------------------------------------------
elif [[ "$MODE" == "pod" ]]; then
  FILENAME="/tmp/backup-${DATE}-pod.tar"
  ssh-vm -r ${NODE} -c "kubectl cp ${POD} ./backup-${DATE}"
  ssh-vm -r ${NODE} -c "tar -cvf ${FILENAME} ./backup-${DATE}"

# ---------------------------------------------------------------
# Modo FOLDER: copia directamente una carpeta de un nodo
# ---------------------------------------------------------------
elif [[ "$MODE" == "folder" ]]; then
  FILENAME="/tmp/backup-${DATE}-node.tar"
  # Separar nodo y ruta (ejemplo: nodo:/path)
  NODE=$( echo $FOLDER | cut -d":" -f1 )
  FOLDER=$( echo $FOLDER | cut -d":" -f2 ) 
  echo "++++++ $NODE ++++ $FOLDER +++"
  ssh-vm -r ${NODE} -c "tar -cvf ${FILENAME} ${FOLDER}"
fi

# ---------------------------------------------------------------
# Transferir archivo tar al host local mediante scp
# ---------------------------------------------------------------
exec 2>/dev/null
scp -i ~/.ssh/id_internal_vm root@$(get-ip ${NODE}):"${FILENAME}" ./

# Eliminar YAML temporal en caso de existir
rm -f exporter_file.yaml

\end{minted}