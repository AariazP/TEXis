El siguiente script implementa un comando principal denominado \texttt{vbc}, que funciona como una interfaz centralizada para la gestión de \VM, clústeres y operaciones auxiliares. Su diseño modular permite agrupar diferentes dominios de comandos: \texttt{vm} (máquinas virtuales), \texttt{cluster} (operaciones sobre el clúster), \texttt{kubectl} (comandos pasados directamente al gestor de Kubernetes) y \texttt{console} (operaciones de consola). De esta forma, el script facilita la administración de recursos a través de una única herramienta, estandarizando la interacción con las distintas funciones ya implementadas en scripts auxiliares.

\begin{minted}[frame=lines,
    fontsize=\scriptsize,
    breaklines]{bash}
#!/bin/bash

set -euo pipefail  # Manejo estricto de errores, variables indefinidas y pipes

# ---------------------------------------------------------------
# Función: usage
# Descripción:
#   Muestra el modo de uso del script y las opciones disponibles.
# ---------------------------------------------------------------
usage() {
    echo "Usage: vbc <domain> <subcommand> [args...]"
    echo
    echo "Domains:"
    echo "  vm        create | ping | list | ssh | get-ip"
    echo "  cluster   create | clean | upload | backup"
    echo "  kubectl   (no subcommands yet, passes directly)"
    echo "  console   (no subcommands yet, passes directly)"
    exit 1
}

# ---------------------------------------------------------------
# MAIN
# ---------------------------------------------------------------

# Validar número mínimo de argumentos
if [[ $# -lt 1 ]]; then
    usage
fi

DOMAIN=$1          # Primer argumento: dominio de operación
SUBCMD=${2:-}      # Segundo argumento: subcomando (opcional)

shift              # Eliminar el dominio de la lista de argumentos
[[ -n "$SUBCMD" ]] && shift # Eliminar también el subcomando si está presente

# Evaluar dominio principal
case "$DOMAIN" in
  vm)
    # Subcomandos para la gestión de VMs
    case "$SUBCMD" in
      create)   bootvm "$@" ;;           # Crear VM
      ping)     get-vm -p "$@" ;;        # Verificar disponibilidad de VM
      list)     get-vm ;;                # Listar VMs
      ssh)      ssh-vm "$@" ;;           # Conexión SSH a VM
      get-ip)   get-ip "$@" ;;           # Obtener IP de VM
      *) echo "Unknown vm subcommand: $SUBCMD"; usage ;;
    esac
    ;;

  cluster)
    # Subcomandos para gestión del clúster
    case "$SUBCMD" in
      create)   cluster-bootup "$@" ;;   # Inicializar clúster
      clean)    clean-cluster "$@" ;;    # Limpiar clúster
      upload)   upload_yaml $( get-vm -a -n 1 master ) "$@" ;; # Subir manifiestos
      backup)   backup "$@" ;;           # Crear copia de respaldo
      *) echo "Unknown cluster subcommand: $SUBCMD"; usage ;;
    esac
    ;;

  kubectl)
    # Pasar directamente el comando kubectl a un nodo master vía SSH
    ssh-vm -r $( get-vm -a -n 1 master ) -c "kubectl $SUBCMD $@"
    ;;

  console)
    # Pasar comandos directamente al gestor de consola
    console "$@"
    ;;

  *)
    # En caso de dominio no reconocido
    echo "Unknown domain: $DOMAIN"
    usage
    ;;
esac
\end{minted}
