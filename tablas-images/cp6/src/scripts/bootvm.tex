El script \texttt{bootvm.sh} permite automatizar la creación e inicio de una \VM\. Su funcionamiento parte de una plantilla predefinida, identificada por un \texttt{UUID}, desde la cual se instancia una nueva máquina virtual. El usuario debe proporcionar un nombre como argumento de entrada, el cual se asigna directamente a la \VM\ creada. Finalmente, el script arranca la máquina virtual recién generada de forma automática. 

\begin{minted}[frame=lines,
    fontsize=\scriptsize,
    breaklines]{bash}
#!/bin/bash

# ---------------------------------------------------------------
# Script: bootvm.sh
# Descripción:
#   Este script crea e inicia una máquina virtual (VM) en XCP-ng 
#   a partir de una plantilla predefinida, usando el nombre que 
#   se le pase como argumento.
# ---------------------------------------------------------------

# Validar que se haya pasado al menos un argumento (nombre de la VM)
if [ $# -lt 1 ]
then
   echo "usage: bootvm <vm-name>"
   exit 0
fi

# Crear la VM a partir de una plantilla específica (UUID de la plantilla)
# y asignarle un nombre nuevo igual al argumento pasado
VM_UUID=$(xe vm-install template=0c838875-cb93-dfe3-8c36-a2f42183b434 \
           new-name-label="$1")

# Iniciar la VM recién creada usando su UUID
xe vm-start uuid=$VM_UUID
\end{minted}