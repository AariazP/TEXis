Este script en \texttt{bash} realiza una prueba funcional básica sobre un clúster recién creado. Primero levanta la infraestructura mínima con un nodo maestro y un nodo trabajador. Luego, transfiere y aplica en el maestro un archivo \texttt{YAML} que define un despliegue de prueba (\texttt{test-deployment.yaml}). Finalmente, el script comprueba si el despliegue denominado \texttt{nginx-deployment} se encuentra en ejecución mediante el comando \texttt{kubectl get pods}. Si se detecta correctamente, se imprime un mensaje de éxito; en caso contrario, se muestra un mensaje de fallo.  

\begin{minted}[frame=lines,
    fontsize=\scriptsize,
    breaklines]{bash}

#!/bin/bash

# Levanta un clúster con:
# - 1 nodo master
# - 1 nodo worker
cluster-bootup -m 1 -w 1

# Sube y aplica el archivo YAML al nodo master1
upload_yaml master1 test-deployment.yaml

# Verifica que el pod asociado al despliegue "nginx-deployment" esté corriendo.
# Si aparece en la lista de pods, la prueba es exitosa, de lo contrario falla.
ssh-vm -r master1 -c "kubectl get pods" | grep -q nginx-deployment \
    && echo "Test succedeed" \
    || echo "Test failed"

\end{minted}