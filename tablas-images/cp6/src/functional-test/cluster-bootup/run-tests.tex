Este script en \texttt{bash} implementa un conjunto de pruebas funcionales automatizadas para verificar la correcta creación de clústeres virtuales mediante el uso de la herramienta \texttt{cluster-bootup}. Se evalúan tres configuraciones distintas: un clúster con un maestro y un trabajador, un clúster con dos maestros y un trabajador, y finalmente una variante que además incluye un servidor \texttt{NAT-Backup}. El script compara los resultados esperados con los obtenidos tras el despliegue de las máquinas virtuales, contabilizando las pruebas superadas y limpiando el clúster al finalizar cada verificación. De esta forma, se garantiza la reproducibilidad y consistencia en los entornos de prueba.  

\begin{minted}[frame=lines,
    fontsize=\scriptsize,
    breaklines]{bash}
#!/bin/bash

# Contador de pruebas exitosas
SUCCEDEED_TESTS=0

# Función de verificación: compara valores esperados con resultados
check(){
     echo +++++++++++++++++strings+++++++++++++++++++++++++
     echo $2
     echo $3
     if [[ "$2" == "$3" ]]
     then
        echo "Test for $1 succedeed"
        SUCCEDEED_TESTS=$(( $SUCCEDEED_TESTS + 1 ))
        return 0
     fi

     echo "Test for $1 failed"
     return 1
}

# ===== Primera prueba: 1 master, 1 worker =====
export EXPECTED_RESULT=" master1 NAT-Master worker1"
cluster-bootup -m 1 -w 1
export RESULT=$( get-vm | sort | tr -d '\n')
check  "1 master 1 worker" "$EXPECTED_RESULT" "$RESULT"
clean-cluster -y

# ===== Segunda prueba: 2 masters, 1 worker =====
export EXPECTED_RESULT=" load-balancer1 master1 master2 NAT-Master worker1"
cluster-bootup -m 2 -w 1
export RESULT=$( get-vm | sort | tr -d '\n')
check  "2 master 1 worker" "$EXPECTED_RESULT" "$RESULT"
clean-cluster -y

# ===== Tercera prueba: 2 masters, 1 worker, 1 NAT Backup =====
export EXPECTED_RESULT=" load-balancer1 master1 master2 NAT-Backup NAT-Master worker1"
cluster-bootup -m 2 -w 1 -b
export RESULT=$( get-vm | sort | tr -d '\n')
check  "2 master 1 worker 1 NAT Backup" "$EXPECTED_RESULT" "$RESULT"
clean-cluster -y

# Reporte final de pruebas
echo "${SUCCEDEED_TESTS}/3 tests succeeded"
\end{minted}