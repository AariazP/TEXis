Este script en \texttt{bash} realiza una serie de pruebas para validar la correcta creación de un clúster con dos nodos maestros, un nodo trabajador, un balanceador de carga y un servidor \texttt{NAT-Backup}. Tras ejecutar el despliegue mediante la función \texttt{cluster-bootup}, el script verifica que cada máquina virtual reciba la dirección IP correspondiente a su rol: los maestros, el trabajador y el balanceador de carga dentro de la red \texttt{192.x.x.x}, mientras que los servidores NAT deben ubicarse en la subred \texttt{172.x.x.x}. Cada prueba exitosa incrementa un contador, y al final se presenta un resumen del total de pruebas superadas.  

\begin{minted}[frame=lines,
    fontsize=\scriptsize,
    breaklines]{bash}
#!/bin/bash

# Inicia el clúster con:
# - 1 worker
# - 2 masters
# - servidor NAT de respaldo (-b)
cluster-bootup -w 1 -m 2 -b

# Breve pausa para dar tiempo a que arranquen las VMs
sleep 2

# Contador de pruebas exitosas
SUCCESS_TEST_COUNT=0

# Verificación de IPs según el rol de cada VM
get-ip master1 | grep 192 && echo "Test for master succedeed" \
  && SUCCESS_TEST_COUNT=$(( $SUCCESS_TEST_COUNT + 1 ))

get-ip worker1 | grep 192 && echo "Test for worker succedeed" \
  && SUCCESS_TEST_COUNT=$(( $SUCCESS_TEST_COUNT + 1 ))

get-ip load-balancer | grep 192 && echo "Test for load-balancer succedeed" \
  && SUCCESS_TEST_COUNT=$(( $SUCCESS_TEST_COUNT + 1 ))

get-ip NAT-Master | grep 172 && echo "Test for NAT-Master succedeed" \
  && SUCCESS_TEST_COUNT=$(( $SUCCESS_TEST_COUNT + 1 ))

get-ip NAT-Backup | grep 172 && echo "Test for NAT-Backup succedeed" \
  && SUCCESS_TEST_COUNT=$(( $SUCCESS_TEST_COUNT + 1 ))

# Resultado final de las pruebas
echo "$SUCCESS_TEST_COUNT/5 tests succedeed"
\end{minted}