\begin{minted}[frame=lines,
    fontsize=\scriptsize,
    breaklines]{bash}
#!/bin/bash

# ---------------------------------------------------------------
# Script: red.sh
# Descripción:
#   Este script crea una red interna en XCP-ng, la configura con
#   NAT para permitir que las máquinas virtuales conectadas a ella
#   tengan salida a Internet, y habilita el reenvío de paquetes 
#   en el host.
# ---------------------------------------------------------------

# Parámetros de configuración de la red
NETWORK_NAME="vm-net-30"         # Nombre asignado a la red en XCP-ng
SUBNET="172.30.30.0/24"          # Rango de direcciones IP para la red
GATEWAY="172.30.30.1"            # Dirección IP del gateway de la red
HOST_IF_NAME="host-vif-vmnet30"  # Nombre de la interfaz virtual del host
OUT_INTERFACE="eth0"             # Interfaz del host que tiene salida a Internet

# Crear red interna sin bridge en XCP-ng
echo "Creando red virtual sin bridge..."
xe network-create name-label="$NETWORK_NAME"

# Obtener el UUID de la red recién creada
NET_UUID=$(xe network-list name-label="$NETWORK_NAME" --minimal)

# Validar que la red haya sido creada correctamente
if [ -z "$NET_UUID" ]; then
    echo "Error: no se pudo crear o encontrar la red."
    exit 1
fi

echo "Red creada con UUID: $NET_UUID"

# Crear interfaz virtual para el host en la red interna
# Esto permite que el host actúe como gateway para las VMs
HOST_UUID=$(xe host-list --minimal)

echo "Agregando interfaz virtual al host..."
xe vif-create network-uuid=$NET_UUID device=0 vm-uuid=$HOST_UUID MAC=random

# Configurar dirección IP en el bridge creado automáticamente por Xen
BRIDGE_IF=$(xe network-param-get param-name=bridge uuid=$NET_UUID)
echo "Asignando IP $GATEWAY al bridge $BRIDGE_IF..."
ip addr add $GATEWAY/24 dev $BRIDGE_IF
ip link set dev $BRIDGE_IF up

# Habilitar reenvío de paquetes IPv4 en el host
echo "Habilitando IP forwarding..."
sysctl -w net.ipv4.ip_forward=1
sed -i '/^#net.ipv4.ip_forward=1/c\net.ipv4.ip_forward=1' /etc/sysctl.conf

# Configurar reglas de NAT con iptables para salida a Internet
echo "Configurando NAT con iptables..."
iptables -t nat -A POSTROUTING -s $SUBNET -o $OUT_INTERFACE -j MASQUERADE
iptables -A FORWARD -s $SUBNET -j ACCEPT

# Guardar reglas iptables si la herramienta está disponible
if command -v netfilter-persistent &> /dev/null; then
    echo "Guardando reglas iptables con netfilter-persistent..."
    netfilter-persistent save
elif command -v iptables-save &> /dev/null; then
    echo "Puedes guardar las reglas con: sudo iptables-save > /etc/iptables/rules.v4"
fi

echo "Configuración completada. Las VMs conectadas a '$NETWORK_NAME' con IPs en $SUBNET podrán salir a Internet vía $OUT_INTERFACE."
\end{minted}