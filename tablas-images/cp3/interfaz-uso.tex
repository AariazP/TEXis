\begin{table}[H]
\centering
\scriptsize
\setlength{\tabcolsep}{3pt}
\renewcommand{\arraystretch}{1.1}
\begin{tabularx}{\textwidth}{|p{0.18\textwidth}|>{\raggedright\arraybackslash}X|}
\hline
\textbf{Tecnología} & \textbf{Interfaz de Uso} \\
\hline
Docker & \CLI\ principalmente, con Docker Desktop para interfaz gráfica. \\
\hline
Podman & \CLI\ similar a Docker, sin daemon. Opcional Podman Desktop. \\
\hline
Udocker & \CLI\ específica para ejecutar contenedores sin privilegios root. \\
\hline
Wasm  & Ejecución través de navegadores web, \API\ de JavaScript. \\
\hline
LXC & \CLI\ mediante comando lxc, sin interfaz gráfica oficial. \\
\hline
Containerd & \CLI\ con herramientas como ctr, backend para otras herramientas. \\
\hline
LXD & \CLI\ mediante lxd/lxc, con interfaz web LXD Web \UI. \\
\hline
Rkt & \CLI\ mediante comandos como rkt run (descontinuado). \\
\hline
Singularity & \CLI\ mediante comandos singularity para gestión de contenedores. \\
\hline
runC & \CLI\ mediante comandos runc, runtime bajo Docker y Kubernetes. \\
\hline
CRI-O & \CLI, interactúa con Kubernetes, sin interfaz gráfica dedicada. \\
\hline
Hyper-V containers & \CLI\ (PowerShell) o Hyper-V Manager para VMs. \\
\hline
OpenVZ & \CLI\ mediante comandos vzctl, con interfaces gráficas de terceros. \\
\hline
Linux VServer & \CLI\ mediante comandos vserver para gestión. \\
\hline
Google gVisor & \CLI\ mediante comandos estándar de Docker con seguridad adicional. \\
\hline
Kata Containers & \CLI\ mediante kata-runtime, integración con Kubernetes. \\
\hline
Firecracker & \CLI\ mediante \API\ RESTful y herramientas firecracker. \\
\hline
Sarus & \CLI\ mediante comando sarus para entornos \HPC. \\
\hline
\end{tabularx}
\caption{Interfaz de uso de cada VBC}\label{tab:interfaz-vbc}
\end{table}