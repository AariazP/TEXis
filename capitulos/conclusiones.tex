\ChapterImageStar[cap:conclusiones]{Conclusiones}{./images/fondo.png}\label{cap:conclusiones}
\mbox{}\\
\noindent
El desarrollo de este trabajo permitió especificar una solución arquitectónica basada en tecnologías de virtualización por contenedores para el Grupo de Investigación en Redes, Información y Distribución de la Universidad del Quindío. A lo largo del proyecto, se consolidó un proceso metodológico que integró la caracterización del contexto institucional, la revisión sistemática del estado del arte, la evaluación técnica de tecnologías y el diseño arquitectónico, culminando con la implementación de un producto mínimo viable que materializó la propuesta.\\ \\
\noindent
Desde la perspectiva técnica, la revisión sistemática de la literatura identificó un ecosistema amplio y diverso de tecnologías de \VBC, con diferencias notorias en licencias, modelos de uso, integración con plataformas \textit{cloud} y enfoques de seguridad. El análisis comparativo mediante el cuadrante Gartner adaptado permitió visualizar la posición relativa de cada tecnología, situando a Docker y Containerd como referentes por su madurez y adopción, mientras que otras herramientas se orientan a nichos específicos como la computación de alto rendimiento (\HPC) o entornos con restricciones de privilegios. El \textit{benchmarking} técnico evidenció comportamientos diferenciados en cuanto al consumo de recursos, tiempo de arranque y rendimiento de red y almacenamiento, proporcionando datos cuantitativos que respaldaron la selección final.\\ \\
\noindent
La aplicación de la metodología de Análisis de Decisiones y Resolución facilitó una evaluación estructurada y reproducible, considerando criterios técnicos, operativos y organizacionales. Containerd surgió como la tecnología más adecuada para el contexto del \GRID, debido a su compatibilidad con el ecosistema Docker, su integración nativa con Kubernetes, su licencia permisiva y su bajo consumo de recursos. De igual forma, K3S se identificó como el motor de orquestación más viable, dada su ligereza, facilidad de instalación y adaptación a entornos con recursos limitados.\\ \\
\noindent
El diseño arquitectónico modelado en ArchiMate articuló la infraestructura existente del \GRID\ —basada en hipervisores Xcp-ng— con una capa de virtualización ligera mediante contenedores, estableciendo un modelo por capas que separa los niveles de infraestructura, virtualización y aplicación. Este enfoque facilitó la representación de las vistas de negocio, aplicación y tecnología, mostrando cómo los procesos misionales se soportan en servicios tecnológicos escalables y mantenibles.\\ \\
\noindent
La implementación del \PMV\ demostró la viabilidad de la solución propuesta, mediante scripts de automatización que permiten el despliegue y gestión de un clúster K3S sobre máquinas virtuales. La validación en un entorno controlado confirmó que la arquitectura es funcional y se ajusta a las necesidades de usuarios finales como estudiantes e investigadores, sin requerir modificaciones mayores en la infraestructura preexistente.\\ \\
\noindent
Desde el punto de vista metodológico, el proyecto estableció un marco de referencia para la toma de decisiones tecnológicas en contextos académicos, combinando revisión sistemática, evaluación técnica y diseño arquitectónico. Este enfoque puede ser replicado en otras instituciones o grupos de investigación que enfrenten desafíos similares de selección e implantación de tecnologías emergentes.\\ \\
\noindent
En términos institucionales, el trabajo contribuye al fortalecimiento de las capacidades del \GRID, ofreciendo una alternativa complementaria a la virtualización tradicional que amplía su portafolio de servicios y se alinea con sus objetivos de docencia, investigación y extensión. La documentación generada —incluyendo modelos arquitectónicos, scripts de automatización y resultados de pruebas— constituye un acervo técnico que puede ser utilizado como base para futuras extensiones o adaptaciones.\\ \\
\noindent
Finalmente, el proyecto refleja la importancia de alinear las decisiones tecnológicas con las particularidades del contexto organizacional, considerando no solo aspectos técnicos, sino también restricciones presupuestarias, capacidades institucionales y necesidades reales de los usuarios. La solución propuesta sienta las bases para una evolución gradual de la infraestructura del \GRID\ hacia modelos más flexibles y adaptativos, en línea con las tendencias actuales en computación en la nube y aplicaciones nativas del entorno \textit{cloud}.