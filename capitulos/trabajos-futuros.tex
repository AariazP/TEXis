\ChapterImageStar[cap:trabajos-futuros]{Trabajos Futuros}{./images/fondo.png}\label{cap:trabajos-futuros}
\mbox{}\\
El desarrollo de esta solución arquitectónica basada en contenedores para el \GRID\ establece una base sobre la cual se pueden proyectar diversas líneas de trabajo posteriores. Estas propuestas se orientan a expandir las capacidades técnicas, explorar nuevos escenarios de uso y profundizar en aspectos específicos que, por alcance o tiempo, no fueron cubiertos en el presente proyecto. \\ \\
\noindent
Una línea natural de trabajo consiste en la integración de mecanismos de gestión de identidad y acceso más avanzados dentro del clúster K3S. Esto incluiría la implementación de roles y permisos granulares (RBAC) adaptados a los distintos perfiles de usuarios del \GRID\ (estudiantes, administrativos, docentes e investigadores). Adicionalmente, se podría explorar la incorporación de herramientas de auditoría y trazabilidad que registren el uso de los recursos computacionales por parte de los usuarios, facilitando la generación de reportes de consumo y la planificación de capacidad. \\ \\
\noindent
Otro ámbito de desarrollo futuro radica en el fortalecimiento de las estrategias de resiliencia y disponibilidad del clúster. Si bien el diseño actual considera la redundancia de servidores NAT y la posibilidad de múltiples nodos maestros, se podría implementar un sistema de respaldo y recuperación externo y automatizado para la configuración del clúster (etcd) y los volúmenes de datos persistentes. La exploración de soluciones de almacenamiento distribuido compatible con Kubernetes, como Rook o Longhorn, permitiría ofrecer volúmenes persistentes con replicación de datos entre los nodos del clúster, aumentando la tolerancia a fallos y siguiendo con práctias de recuperación como lo es el \RPO\ y \RTO\. \\ \\
\noindent
La automatización del ciclo de vida de las aplicaciones desplegadas en el clúster representa otra área de interés. Sería valioso desarrollar flujos de trabajo basados en GitOps para la implementación continua de aplicaciones, utilizando herramientas como ArgoCD o Flux. Esto permitiría que los cambios en los repositorios de código o configuración se reflejen automáticamente en el entorno de ejecución, mejorando la consistencia y la velocidad de despliegue para los proyectos académicos.\\ \\
\noindent
Extender la solución hacia un modelo de nube híbrida constituye una dirección estratégica. Esto implicaría estudiar la viabilidad de integrar el clúster local del GRID con servicios de nube pública, como AWS EKS, Google GKE o Azure AKS, permitiendo el desplazamiento de cargas de trabajo (bursting) cuando la demanda local exceda la capacidad disponible. Este enfoque facilitaría la ejecución de proyectos que requieran recursos computacionales puntuales muy elevados, sin necesidad de una inversión permanente en hardware.
\noindent
Desde una perspectiva pedagógica, se podría elaborar material de formación y documentación técnica más detallada dirigida a los usuarios finales. Esto incluiría guías prácticas para la definición de descriptores YAML, buenas prácticas para la contenerización de aplicaciones comunes en el ámbito académico (por ejemplo, entornos de desarrollo para Python o Java). La creación de un catálogo interno de plantillas de aplicaciones preconfiguradas agilizaría la adopción de la plataforma. \\ \\
\noindent
Finalmente, la metodología empleada en este trabajo —que combina caracterización, revisión sistemática, evaluación técnica y diseño arquitectónico— podría ser formalizada y adaptada como un marco de referencia para la adopción de tecnologías emergentes en otros grupos de investigación o dependencias universitarias. Documentar este proceso en forma de guía metodológica permitiría su replicación y adaptación a diferentes contextos institucionales, contribuyendo a la toma de decisiones tecnológicas informadas en el ámbito académico.\\ \\
\noindent
Estas líneas de trabajo futuro representan una evolución natural del proyecto actual, apuntando a una consolidación progresiva de la plataforma de contenedores como un servicio robusto, flexible y alineado con las necesidades a largo plazo del \GRID\ y de la Universidad del Quindío.