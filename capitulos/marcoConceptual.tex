\ChapterImageStar[cap:marcoConceptual]{Marco Conceptual}{./images/fondo.png}\label{cap:marcoConceptual}
\mbox{}\\
Los conceptos a continuación no solo delimitan el ámbito de estudio, sino que también proporcionan las bases terminológicas y estructurales necesarias para la evaluación, comparación e implementación de las tecnologías consideradas. 

\section{Virtualización}
\noindent
La virtualización constituye el punto de partida de este marco y según~\citep{AmazonWebServicesInc2023} se entiende como una tecnología que permite la creación de representaciones virtuales de recursos físicos de computación, como servidores, almacenamiento o redes, facilitando la ejecución de múltiples entornos aislados sobre una misma infraestructura física. Tradicionalmente, la virtualización se ha implementado mediante hipervisores, software que posibilita la creación y gestión de máquinas virtuales (\VM), las cuales proyectan un isomorfismo del hardware completo y ejecutan sistemas operativos independientes (\OS)~\citep{KLEIDERMACHER201225}. Si bien esta aproximación ofrece un alto grado de aislamiento, conlleva una sobrecarga significativa en términos de consumo de recursos, debido a la duplicación de sistemas operativos y la necesidad de reservar recursos de forma estática para cada instancia~\citep{bauman2015survey}. Actualmente, en el grupo de investigación se aprovisionan servicios mediante máquinas virtuales.

\section{Virtualización basada en contenedores}
\noindent
Como alternativa al modelo anterior, emerge la \VBC. A diferencia de las máquinas virtuales, los contenedores no virtualizan el hardware subyacente, sino que comparten el kernel del sistema operativo anfitrión~\citep{eder2016hypervisor}. Esto se logra mediante mecanismos de aislamiento a nivel de sistema operativo, como \textit{namespaces} y \textit{cgroups} en Linux, que permiten aislar procesos, sistemas de archivos, redes y recursos computacionales para cada contenedor~\citep{jain2020linux}. El resultado son instancias ligeras, con tiempos de arranque reducidos y un consumo de recursos inferior en el uso de \CPU\;, memoria y almacenamiento, en comparación con las \VM\ tradicionales~\citep{6903537}. Se puede identificar como un complemento a la virtualización completa ya disponible en el grupo.

\section{Registro de contenedores}
\noindent
Un concepto central en el ecosistema de contenedores es el de la imagen de contenedor. Una imagen es un paquete inmutable que incluye todo lo necesario para ejecutar una aplicación: código, bibliotecas, herramientas del sistema y configuraciones~\citep{straesser2023empirical}. Las imágenes se construyen en capas a partir de un archivo de instrucciones (Dockerfile, por ejemplo), lo que favorece la reutilización y el control de versiones~\citep{dahlmanns2023secrets}. Estas imágenes se almacenan en un \textit{registry} (como Docker Hub), desde donde pueden ser descargadas y ejecutadas en cualquier entorno compatible~\citep{anwar2018improving}. El uso de registros populares permite al grupo de investigación aprovisionar software estándar de manera rápida y sencilla. Agregando a esto, se pueden crear imágenes personalizadas que encapsulen aplicaciones específicas del grupo o sus dependencias, facilitando su despliegue y distribución en repositorios ya existentes o privados.

\section{Orquestación de contenedores}
\noindent
La orquestación de contenedores representa otro pilar conceptual indispensable. Gestionar manualmente unos pocos contenedores es factible, pero a medida que el número de contenedores crece, se requiere de herramientas automatizadas para desplegar, escalar, monitorizar y mantener la salud de cientos o miles de contenedores~\citep{al2019container}. Aquí, Kubernetes se erige como el estándar \textit{de facto}~\citep{zhou2021container}. Kubernetes es una plataforma \textit{open-source} que automatiza la gestión de aplicaciones contenerizadas, proporcionando mecanismos para el descubrimiento de servicios, balanceo de carga, auto-escalamiento y despliegues automatizados~\citep{carrion2022kubernetes}. Para el caso del \GRID, Kubernetes no solo facilita la administración de contenedores, sino que también permite afinar el uso de recursos en un entorno compartido, propendiendo por la alta disponibilidad y resiliencia frente a fallos.

\section{Runtime de contenedores}
\noindent
Kubernetes no interactúa directamente con los contenedores, sino a través de un \textit{runtime} de contenedores que cumple con la \textit{Open Container Initiative} (\OCI), un estándar industrial que especifica formatos y rutinas para contenedores~\citep{girma2018evaluation}. Es en este nivel donde se sitúan las tecnologías de \VBC\ evaluadas en este trabajo, como Docker~\citep{docker_website}, Podman~\citep{podman_website}, containerd~\citep{containerd_website} y LXC~\citep{lxc_website}. \@Cada una ofrece una implementación distinta para crear y ejecutar contenedores \OCI. Docker, por ejemplo, popularizó el uso de contenedores al ofrecer una herramienta todo-en-uno que incluye un \textit{daemon}, un \CLI\ y herramientas de construcción de imágenes~\citep{Buchanan2020}. Containerd, por su parte, es un \textit{runtime} más minimalista y enfocado, que se diseñó para ser embebido en sistemas más grandes como Kubernetes, manejando el ciclo de vida de los contenedores a bajo nivel~\citep{protogeros2024cargosync}.

\section{Cloud native}
\noindent
La transición hacia infraestructuras definidas por \textit{software} y aplicaciones \textit{cloud-native} está intrínsecamente ligada a estos conceptos. Una aplicación \textit{cloud-native} está diseñada específicamente para aprovechar las ventajas de la nube, como la escalabilidad y la alta disponibilidad, y suele estar compuesta por microservicios empaquetados en contenedores y orquestados mediante plataformas como Kubernetes~\citep{gannon2017cloud}. Este paradigma arquitectónico permite una entrega continua y una resiliencia superior frente a modelos monolíticos tradicionales~\citep{oyeniran2024comprehensive}. Para el \GRID, la adopción de estas tecnologías representa una evolución natural de su infraestructura. El grupo ya cuenta con una base sólida de virtualización tradicional mediante el hipervisor XCP-ng. La incorporación de \VBC\ no busca reemplazar esta infraestructura, sino complementarla, ofreciendo un portafolio de servicios más diversificado. Los contenedores pueden coexistir con las máquinas virtuales, permitiendo asignar la tecnología más adecuada según la carga de trabajo. Además, en la implementación de un \textit{cluster} de Kubernetes se puede utilizar \VM\ como nodos, que sirve de base para el posterior despliegue de servicios.

\section{GitOps}
\noindent
GitOps, por otro lado, es un paradigma operativo que utiliza repositorios Git como fuente de verdad para la definición de la infraestructura y las aplicaciones, automatizando los despliegues para que el estado del sistema siempre refleje el estado declarado en el repositorio~\citep{kormanik2023exploring}. En conjunto, estos conceptos delinean un panorama tecnológico moderno y robusto sobre el cual se diseña la solución arquitectónica para el \GRID. Proporcionan el vocabulario técnico y las bases estructurales necesarias para entender la evaluación, selección e integración de las tecnologías de \VBC\ en el contexto específico del grupo de investigación, con sus particulares restricciones de recursos y sus objetivos misionales de docencia, investigación y extensión.


