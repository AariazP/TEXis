\section{Nichos de mercado}
\noindent
La identificación de los nichos de mercado permite comprender el posicionamiento y la orientación estratégica de cada tecnología de \VBC. En este apartado se analizan las audiencias objetivo y los contextos en los que las diferentes tecnologías encuentran mayor adopción y aplicabilidad. Cada una de estas herramientas responde a necesidades particulares dentro del ecosistema de la computación y la orquestación de contenedores, lo que facilita distinguir sus ventajas competitivas y su relevancia en función de los requerimientos de los proveedores de servicios, organizaciones y desarrolladores.
\subsection{Containerd}
\noindent
Containerd está dirigido a proveedores de servicios en la nube y plataformas de orquestación como Kubernetes, donde se requiere una solución de gestión de contenedores ligera y compatible con OCI \citep{Vano2023}. Su arquitectura modular lo convierte en una opción preferida para grandes infraestructuras \citep{Zhou2021}.

\subsection{CRI-O}
\noindent
CRI-O está diseñado específicamente para su integración con Kubernetes, sirviendo como un motor de contenedores ligero y compatible con OCI para esta plataforma \citep{CNCF2019}. Es una solución ideal para proveedores de servicios en la nube y organizaciones que utilizan Kubernetes como su plataforma de orquestación principal \citep{151962df5f7e4b9faba0629540c11439}.

\subsection{Docker}
\noindent
Docker se posiciona principalmente en el nicho de mercado de desarrolladores de software, empresas tecnológicas y proveedores de servicios en la nube que buscan una solución para la creación, implementación y gestión de aplicaciones en contenedores \citep{Hill2025}. Su capacidad de automatizar despliegues y la portabilidad entre entornos lo convierte en una opción ideal para DevOps y desarrollo ágil \citep{Mag2025}.

\subsection{Firecracker}
\noindent
Firecracker está orientado a proveedores de servicios en la nube y plataformas de cómputo en la nube que requieren micro VMs eficientes y seguras \citep{Jain}. Es una solución ideal para plataformas \textit{serverless} y entornos multi-tenant \citep{246288}.

\subsection{Google gVisor}
\noindent
Google gVisor está dirigido a proveedores de servicios en la nube y organizaciones que priorizan la seguridad en sus entornos de contenedores \citep{LopezFalcon2024}. Su arquitectura de \textit{sandbox} proporciona un aislamiento fuerte, lo que lo convierte en una opción atractiva para aplicaciones sensibles \citep{gvisor2025}.

\subsection{Hyper-V Containers}
\noindent
Hyper-V Containers están orientados a empresas que utilizan infraestructuras basadas en Windows, ofreciendo una solución de contenedorización segura y eficiente para aplicaciones basadas en Windows \citep{Smith2016}. Su integración con el ecosistema de Microsoft lo hace ideal para empresas con infraestructuras híbridas \citep{Clark2024}.

\subsection{Kata Containers}
\noindent
Kata Containers se centra en entornos donde se requiere un alto nivel de seguridad y aislamiento, como proveedores de servicios en la nube y empresas que manejan información confidencial \citep{Viktorsson2020}. Su capacidad para combinar la eficiencia de los contenedores con el aislamiento de máquinas virtuales es su principal ventaja \citep{10.1145/1272996.1273025}.

\subsection{Linux VServer}
\noindent
Linux VServer está orientado a administradores de sistemas y proveedores de servicios que requieren una solución de virtualización ligera basada en contenedores para la administración de servidores seguros y eficientes \citep{10.1145/1272996.1273025}. Es una opción adecuada para entornos de servidor dedicados y alojamientos compartidos \citep{LinuxVirt2017}.

\subsection{LXC (Linux Containers)}
\noindent
LXC es popular en entornos de virtualización ligera y servidores, donde se requiere un control granular sobre los entornos de contenedores \citep{Silva2024}. Su uso está orientado a proveedores de alojamiento web, desarrolladores de software y administradores de sistemas que necesitan un control preciso del entorno del sistema operativo \citep{Simon2023}.

\subsection{LXD}
\noindent
LXD se enfoca en nichos de mercado que requieren entornos de virtualización basados en contenedores que imiten máquinas virtuales, como proveedores de servicios en la nube, plataformas de pruebas y entornos de desarrollo \citep{Silva2024}. Su capacidad para ofrecer entornos de sistema completo lo hace ideal para desarrolladores y administradores de sistemas \citep{Kaiser2022}.

\subsection{OpenVZ}
\noindent
OpenVZ se centra en proveedores de alojamiento web y servicios VPS, donde se requiere una solución de virtualización ligera basada en contenedores que permita un control granular sobre los recursos del sistema y la administración de múltiples instancias \citep{OpenVZ2015}.

\subsection{Podman}
\noindent
Podman está orientado a entornos empresariales y desarrolladores que requieren una solución de contenerización sin \textit{daemon}, compatible con OCI y con enfoque en la seguridad \citep{Surendhar2024}. Su naturaleza sin \textit{daemon} y su capacidad para ejecutar contenedores de forma aislada permiten su adopción en entornos donde la seguridad y la conformidad son prioridades \citep{Trevor2022}.

\subsection{Rkt}
\noindent
Rkt fue diseñado para satisfacer las necesidades de proveedores de servicios en la nube y organizaciones que buscan una alternativa a Docker con un enfoque en la seguridad y compatibilidad OCI \citep{Lingayat2018}. Aunque su desarrollo ha sido discontinuado, sigue siendo relevante en entornos donde la compatibilidad y la seguridad son críticas \citep{Watada2019}.

\subsection{runC}
\noindent
runC está orientado a proveedores de servicios en la nube, plataformas de orquestación como Kubernetes y desarrolladores de software que buscan una solución de contenedorización ligera y compatible con OCI \citep{Perez2005}. Su adopción en proyectos de gran escala se debe a su eficiencia y cumplimiento de estándares de contenedores \citep{151962df5f7e4b9faba0629540c11439}.

\subsection{Sarus}
\noindent
Sarus está dirigido a entornos de HPC y computación científica, donde los usuarios necesitan ejecutar contenedores de forma segura en sistemas de alto rendimiento \citep{Sarus2021}. Su compatibilidad con estándares de contenedores y su enfoque en la seguridad lo hacen ideal para centros de investigación y universidades \citep{B2020}.

\subsection{Singularity}
\noindent
Singularity se centra en entornos de computación científica y HPC, donde se requiere portabilidad de aplicaciones sin necesidad de privilegios de root \citep{10.1145/3332186.3332192}. Es ampliamente adoptado en universidades, centros de investigación y laboratorios que ejecutan aplicaciones de alto rendimiento \citep{Kurtzer2017}.

\subsection{Udocker}
\noindent
Udocker se especializa en nichos de mercado académicos y de investigación, donde los usuarios necesitan ejecutar contenedores sin privilegios en sistemas que no permiten la instalación de software de nivel de sistema \citep{Campos2017}. Su facilidad para funcionar en entornos HPC (Computación de Alto Rendimiento) sin requerir permisos de root lo hace adecuado para instituciones de investigación \citep{Gomes2018}.

\subsection{Wasm (WebAssembly)}
\noindent
Wasm se centra en el nicho de desarrollo web y aplicaciones de alto rendimiento en el navegador \citep{Haas2017}. Su capacidad para ejecutar código en múltiples plataformas, incluidas aplicaciones de escritorio y móviles, lo convierte en una opción atractiva para empresas de desarrollo de software que buscan optimización multiplataforma \citep{Jangda2019}.
\clearpage
\section{Comparativa de licencias}
\noindent
La tabla~\ref{tab:licencias-vbc} presenta un panorama comparativo de las tecnologías de \VBC\ presentadas antes, detallando aspectos clave como el tipo de licencia, los términos de uso y los costos asociados. Este análisis permite identificar no solo las diferencias en cuanto a modelos de distribución y sostenibilidad económica, sino también las implicaciones legales y técnicas que pueden influir en la selección de una u otra herramienta en contextos de investigación o implementación empresarial. Asimismo, se evidencia la coexistencia de soluciones de código abierto, ampliamente utilizadas en entornos académicos y científicos, junto con alternativas propietarias que implican costos más elevados, lo cual resalta la necesidad de evaluar cuidadosamente la relación entre funcionalidad, libertad de uso y viabilidad financiera.
\input{tablas-images/cp3/licencias.tex}

\section{Interfaz de uso}
\noindent
En la tabla~\ref{tab:interfaz-vbc} se describe la interfaz de uso por cada tecnología. Como se puede apreciar, la gran mayoría de tecnologías se utilizan a través de una \CLI. Esto, en muchos casos, puede implicar un aumento en la curva de aprendizaje, pero también facilita la gestión y automatización de las tecnologías una vez que se ha comprendido el uso de su interfaz.
\begin{table}[H]
    \centering
    \includegraphics[width=\textwidth] {tablas-images/cp3/interfaz-uso.png}
    \caption{Interfaz de uso de cada VBC}\label{tab:tabla-interfaz-uso}
\end{table}

\section{Integración con la nube}
\noindent
En el cuadro~\ref{tab:integracion-cloud-vbc} se describe la integración a las distintas plataformas \textit{cloud} que tiene cada tecnología. Se evidencia que muchas tecnologías tienen integración directa con 3 de los proveedores cloud más famosos: \AWS, \GCP\ y Azure. Algunas tecnologías, por otro lado, solo soportan implementación de nubes privadas, como LXD.\@
\begin{table}[H]
\centering
\scriptsize
\setlength{\tabcolsep}{3pt}
\renewcommand{\arraystretch}{1.1}
\begin{tabularx}{\textwidth}{|p{0.2\textwidth}|X|}
\hline
\textbf{Tecnología} & \textbf{Integración con Proveedores de Cloud} \\
\hline
Containerd & Integración fuerte con Kubernetes, que a su vez se integra con proveedores de nube como \AWS\ (EKS), \GCP\ (GKE), Azure (AKS) y otros. \\
\hline
CRI-O & Integración directa con Kubernetes, lo que le permite ser utilizado en proveedores de nube como \AWS\ (EKS), \GCP\ (GKE), Azure (AKS), y otros servicios de orquestación de contenedores. \\
\hline
Docker & Integración con \AWS\ (ECR, ECS), \GCP\ (GCR, GKE), Azure (ACR, AKS), y otros proveedores a través de herramientas como Docker Compose, Docker Swarm y Docker Desktop. \\
\hline
Google gVisor & Integración con Google Cloud, especialmente en Google Kubernetes Engine (GKE), para agregar una capa adicional de seguridad a los contenedores. \\
\hline
Hyper-V containers & Integración exclusiva con Microsoft Azure, especialmente con Azure Kubernetes Service (AKS) y otras soluciones basadas en Hyper-V. \\
\hline
Kata Containers & Soporta proveedores de nube pública como \AWS\, Google Cloud, y Azure a través de Kubernetes, proporcionando aislamiento similar a máquinas virtuales en entornos de contenedores. \\
\hline
Linux VServer & Utilizado principalmente en proveedores de hosting dedicados y servidores privados, sin integración directa con proveedores de nube pública como \AWS\, \GCP\ o Azure. \\
\hline
LXC & Se puede integrar en plataformas de nube privada y algunas soluciones híbridas. Se usa en servidores de nube como OpenStack, pero no tiene una integración directa con plataformas públicas principales. \\
\hline
LXD & Puede integrarse con plataformas de nube privada, como OpenStack, para ofrecer contenedores ligeros que emulan máquinas virtuales. No tiene integración directa con los proveedores de nube pública principales, pero puede ser utilizado en soluciones personalizadas. \\
\hline
OpenVZ & Tradicionalmente usado en proveedores de hosting como OVH, aunque su uso ha disminuido frente a soluciones más modernas. La integración con nubes públicas es limitada y generalmente personalizada. \\
\hline
Podman & Compatible con \AWS\ (ECR), \GCP\ (GCR), Azure (ACR), aunque su integración con orquestadores como Kubernetes es más reciente y menos prevalente que Docker. \\
\hline
Rkt & Aunque estaba integrado con Kubernetes y otras plataformas, su descontinuación limita la integración con proveedores de nube. En el pasado, soportaba plataformas como \AWS\ y \GCP\ (Google Cloud). \\
\hline
runC & Integración con Kubernetes, que se usa ampliamente en proveedores de nube como \AWS\ (EKS), \GCP\ (GKE), y Azure (AKS) para la orquestación de contenedores. \\
\hline
Singularity & Utilizado principalmente en entornos de computación científica y HPC. Puede integrarse con proveedores como \AWS\ (HPC, Batch) y \GCP\ (Compute Engine) para tareas específica de alto rendimiento. \\
\hline
Udocker & Generalmente se usa en entornos sin privilegios de root y en plataformas como \HPC. No tiene una integración directa con proveedores de nube a gran escala. \\
\hline
Wasm (WebAssembly) & Integración principalmente con servicios de computación en la nube como \AWS\ Lambda, \GCP\ (Cloud Functions), y Azure Functions, ya que permite la ejecución de código en la nube sin dependencia del sistema operativo subyacente. \\
\hline
\end{tabularx}
\caption{Integración cloud de cada VBC}\label{tab:integracion-cloud-vbc}
\end{table}

\clearpage
\section{Cuadrante Gartner}
\noindent
La tabla~\ref{tab:cuadrante-gartner} sintetiza la clasificación de tecnologías de \VBC\ en un cuadrante de Gartner adaptado, considerando los ejes de visión (X) y capacidad de ejecución (Y). Esta representación permite identificar el posicionamiento relativo de cada tecnología en el mercado, agrupándolas en cuatro cuadrantes estratégicos: líderes, retadores, visionarios y jugadores de nicho. Con ello se facilita el análisis comparativo entre soluciones consolidadas, alternativas emergentes con alto potencial, y herramientas especializadas con menor alcance, aportando una perspectiva integral para la toma de decisiones en proyectos de investigación y aplicaciones prácticas.
Los calculos tanto de visión como de ejecución se realizaron a partir de las métricas obtenidas en el análisis \DAR\ realizado. Se tuvieron en cuenta las características inherentes a cada tecnología, como su madurez, adopción en la industria, innovación, soporte comunitario y facilidad de integración, posteriormente se realizó un promedio ponderado de cada una de estas características para obtener un valor numérico que representara la visión y la ejecución de cada tecnología. Este promedio también se aproximó al entero superior más cercano para facilitar la interpretación en el cuadrante.
\begin{table}[H]
    \centering
    \includegraphics[width=\textwidth] {tablas-images/cp3/medicion-gartner.png}
    \caption{Tabla de medición para el cuadrante gartner}\label{tab:tabla-medicion-gartner}
\end{table}
\noindent
A partir de estas valoraciones se definieron los cuadrantes, estableciendo umbrales numéricos en cada eje para su clasificación. De este modo, se consideran Líderes aquellas tecnologías con Visión $\geq$ 7 y Ejecución $\geq$ 7, caracterizadas por estar consolidadas, combinar madurez técnica con una fuerte proyección estratégica, como es el caso de Docker y Containerd. Los Retadores corresponden a soluciones con Visión $<$ 7 y Ejecución $\geq$ 7, es decir, tecnologías robustas y estables en su uso, pero con menor capacidad de innovación o proyección a futuro, como Podman y CRI-O. En el cuadrante de Visionarios se ubican aquellas con Visión $\geq$ 7 y Ejecución $<$ 7, las cuales presentan un gran potencial disruptivo, aunque aún con limitaciones de adopción masiva o falta de consolidación, como Wasm, gVisor o Firecracker. Finalmente, los Jugadores de Nicho, definidos por Visión $<$ 7 y Ejecución $<$ 7, incluyen tecnologías con baja adopción, soporte reducido o enfoques muy específicos que restringen su aplicabilidad general, como LXC, Udocker y OpenVZ.
\begin{figure}[H]
    \centering
    \includegraphics[scale=0.1] {tablas-images/cp3/cuadrante-gartner.png}
    \caption{Cuadrante de Gartner de cada VBC}\label{fig:tabla-cuadrante-gartner}
\end{figure}

\section{Entornos de ejecución}
\noindent
A continuación, en la tabla~\ref{tab:entornos-ejecucion-vbc} se describen los ambientes de ejecución de diversas tecnologías de \VBC, resaltando los sistemas operativos compatibles y los contextos de uso más comunes. Este análisis evidencia la versatilidad de las herramientas, que abarcan desde entornos de desarrollo y producción en sistemas tradicionales (Linux, Windows y macOS), hasta aplicaciones en computación de alto rendimiento (\HPC), plataformas en la nube y navegadores web. Asimismo, se observa la coexistencia de soluciones ampliamente adoptadas, como Docker o Containerd, junto con propuestas especializadas en seguridad, portabilidad o rendimiento, lo que permite identificar sus ventajas diferenciales en función de los requerimientos técnicos y organizacionales.
\begin{table}[H]
    \centering
    \includegraphics[width=\textwidth] {tablas-images/cp3/entorno-ejecucion.png}
    \caption{Entornos de ejecución de cada VBC}\label{tab:tabla-entorno-ejecucion}
\end{table}

\section{Evaluación DOFA}
\noindent
La tabla~\ref{tab:matriz-dofa} presenta un análisis integral de las tecnologías de virtualización basadas en contenedores (\VBC), considerando tanto los factores internos —fortalezas y debilidades— como los externos —oportunidades y amenazas— que influyen en su adopción y desarrollo. Este enfoque permite identificar los beneficios clave de la contenerización, como la portabilidad, la eficiencia en el uso de recursos y la escalabilidad, al tiempo que visibiliza los retos asociados a la seguridad, la gestión de infraestructuras complejas y la dependencia de plataformas propietarias.
 
\begin{table}[H]
    \centering
    \includegraphics[width=\textwidth] {tablas-images/cp3/matriz-dofa.png}
    \caption{Tabla de matriz DOFA para el cuadrante gartner}\label{tab:tabla-matriz-dofa}
\end{table}
\clearpage
\section{Documentación y soporte}
\noindent
La tabla~\ref{tab:documentacion-tecnologias} presenta un compendio de enlaces a la documentación oficial de distintas tecnologías de \VBC\, lo cual constituye un recurso para investigadores, desarrolladores y administradores de sistemas. La disponibilidad de documentación confiable y actualizada resulta determinante para una implementación adecuada, resolver problemas técnicos y aprovechar al máximo las funcionalidades de cada herramienta. De esta forma, el cuadro permite acceder de manera centralizada a las fuentes oficiales de soporte, facilitando la consulta y el aprendizaje en entornos académicos y profesionales
\begin{table}[H]
\centering
\scriptsize
\setlength{\tabcolsep}{3pt}
\renewcommand{\arraystretch}{1.1}
\begin{tabular}{|>{\centering\arraybackslash}p{0.5cm}|>{\raggedright\arraybackslash}p{3.5cm}|>{\centering\arraybackslash}p{2.5cm}|}
\hline
\multicolumn{2}{|c|}{\textbf{Tecnología}} & \textbf{Documentación} \\
\hline
1 & Docker & \href{https://docs.docker.com/}{link} \\
\hline
2 & Podman & \href{https://podman.io/docs}{link} \\
\hline
3 & Udocker & \href{https://github.com/indigo-dc/udocker}{link} \\
\hline
4 & Wasm & \href{https://webassembly.org/docs/faq/}{link} \\
\hline
5 & LXC & \href{https://linuxcontainers.org/incus/docs/main/}{link} \\
\hline
6 & Containerd & \href{https://containerd.io/docs/}{link} \\
\hline
7 & LXD & \href{https://linuxcontainers.org/incus/docs/main/}{link} \\
\hline
8 & Rkt & \href{https://github.com/rkt/rkt}{link} \\
\hline
9 & Singularity & \href{https://docs.sylabs.io/guides/4.3/user-guide/}{link} \\
\hline
10 & runC & \href{https://github.com/opencontainers/runc}{link} \\
\hline
11 & CRI-O & \href{https://github.com/cri-o/cri-o}{link} \\
\hline
12 & Hyper-V containers & \href{https://docs.microsoft.com/en-us/virtualization/windowscontainers/}{link} \\
\hline
13 & OpenVZ & \href{https://openvz.org/}{link} \\
\hline
14 & Linux VServer & \href{http://linux-vserver.org/Documentation}{link} \\
\hline
15 & Google gVisor & \href{https://gvisor.dev/docs/}{link} \\
\hline
16 & Kata Containers & \href{https://katacontainers.io/docs/}{link} \\
\hline
17 & Firecracker & \href{https://firecracker-microvm.github.io/}{link} \\
\hline
18 & Sarus & \href{https://github.com/eth-cscs/sarus}{link} \\
\hline
\end{tabular}
\caption{Enlaces a la documentación de tecnologías de contenerización}\label{tab:documentacion-tecnologias}
\end{table}