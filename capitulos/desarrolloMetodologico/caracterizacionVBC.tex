\section{Nichos de mercado}

\subsection{Docker}
Docker se posiciona principalmente en el nicho de mercado de desarrolladores de software, empresas tecnológicas y proveedores de servicios en la nube que buscan una solución para la creación, implementación y gestión de aplicaciones en contenedores \citep{Hill2025}. Su capacidad de automatizar despliegues y garantizar la portabilidad entre entornos lo convierte en una opción ideal para DevOps y desarrollo ágil \citep{Mag2025}.

\subsection{Podman}
Podman está orientado a entornos empresariales y desarrolladores que requieren una solución de contenerización sin \textit{daemon}, compatible con OCI y con enfoque en la seguridad \citep{Surendhar2024}. Su naturaleza sin \textit{daemon} y su capacidad para ejecutar contenedores de forma aislada permiten su adopción en entornos donde la seguridad y la conformidad son prioridades \citep{Trevor2022}.

\subsection{Udocker}
Udocker se especializa en nichos de mercado académicos y de investigación, donde los usuarios necesitan ejecutar contenedores sin privilegios en sistemas que no permiten la instalación de software de nivel de sistema \citep{Campos2017}. Su facilidad para funcionar en entornos HPC (Computación de Alto Rendimiento) sin requerir permisos de root lo hace adecuado para instituciones de investigación \citep{Gomes2018}.

\subsection{Wasm (WebAssembly)}
Wasm se centra en el nicho de desarrollo web y aplicaciones de alto rendimiento en el navegador \citep{Haas2017}. Su capacidad para ejecutar código de forma eficiente en múltiples plataformas, incluidas aplicaciones de escritorio y móviles, lo convierte en una opción atractiva para empresas de desarrollo de software que buscan optimización multiplataforma \citep{Jangda2019}.

\subsection{LXC (Linux Containers)}
LXC es popular en entornos de virtualización ligera y servidores, donde se requiere un control granular sobre los entornos de contenedores \citep{Silva2024}. Su uso está orientado a proveedores de alojamiento web, desarrolladores de software y administradores de sistemas que necesitan un control preciso del entorno del sistema operativo \citep{Simon2023}.

\subsection{Containerd}
Containerd está dirigido a proveedores de servicios en la nube y plataformas de orquestación como Kubernetes, donde se requiere una solución de gestión de contenedores ligera y compatible con OCI \citep{Vano2023}. Su arquitectura modular lo convierte en una opción preferida para grandes infraestructuras \citep{Zhou2021}.

\subsection{LXD}
LXD se enfoca en nichos de mercado que requieren entornos de virtualización basados en contenedores que imiten máquinas virtuales, como proveedores de servicios en la nube, plataformas de pruebas y entornos de desarrollo \citep{Silva2024}. Su capacidad para ofrecer entornos de sistema completo lo hace ideal para desarrolladores y administradores de sistemas \citep{Kaiser2022}.

\subsection{Rkt}
Rkt fue diseñado para satisfacer las necesidades de proveedores de servicios en la nube y organizaciones que buscan una alternativa a Docker con un enfoque en la seguridad y compatibilidad OCI \citep{Lingayat2018}. Aunque su desarrollo ha sido discontinuado, sigue siendo relevante en entornos donde la compatibilidad y la seguridad son críticas \citep{Watada2019}.

\subsection{Singularity}
Singularity se centra en entornos de computación científica y HPC, donde se requiere portabilidad de aplicaciones sin necesidad de privilegios de root \citep{10.1145/3332186.3332192}. Es ampliamente adoptado en universidades, centros de investigación y laboratorios que ejecutan aplicaciones de alto rendimiento \citep{Kurtzer2017}.

\subsection{runC}
runC está orientado a proveedores de servicios en la nube, plataformas de orquestación como Kubernetes y desarrolladores de software que buscan una solución de contenedorización ligera y compatible con OCI \citep{Perez2005}. Su adopción en proyectos de gran escala se debe a su eficiencia y cumplimiento de estándares de contenedores \citep{151962df5f7e4b9faba0629540c11439}.

\subsection{CRI-O}
CRI-O está diseñado específicamente para su integración con Kubernetes, sirviendo como un motor de contenedores ligero y compatible con OCI para esta plataforma \citep{CNCF2019}. Es una solución ideal para proveedores de servicios en la nube y organizaciones que utilizan Kubernetes como su plataforma de orquestación principal \citep{151962df5f7e4b9faba0629540c11439}.

\subsection{Hyper-V Containers}
Hyper-V Containers están orientados a empresas que utilizan infraestructuras basadas en Windows, ofreciendo una solución de contenedorización segura y eficiente para aplicaciones basadas en Windows \citep{Smith2016}. Su integración con el ecosistema de Microsoft lo hace ideal para empresas con infraestructuras híbridas \citep{Clark2024}.

\subsection{OpenVZ}
OpenVZ se centra en proveedores de alojamiento web y servicios VPS, donde se requiere una solución de virtualización ligera basada en contenedores que permita un control granular sobre los recursos del sistema y la administración de múltiples instancias \citep{OpenVZ2015}.

\subsection{Linux VServer}
Linux VServer está orientado a administradores de sistemas y proveedores de servicios que requieren una solución de virtualización ligera basada en contenedores para la administración de servidores seguros y eficientes \citep{10.1145/1272996.1273025}. Es una opción adecuada para entornos de servidor dedicados y alojamientos compartidos \citep{LinuxVirt2017}.

\subsection{Google gVisor}
Google gVisor está dirigido a proveedores de servicios en la nube y organizaciones que priorizan la seguridad en sus entornos de contenedores \citep{LopezFalcon2024}. Su arquitectura de \textit{sandbox} proporciona un aislamiento fuerte, lo que lo convierte en una opción atractiva para aplicaciones sensibles \citep{gvisor2025}.

\subsection{Kata Containers}
Kata Containers se centra en entornos donde se requiere un alto nivel de seguridad y aislamiento, como proveedores de servicios en la nube y empresas que manejan información confidencial \citep{Viktorsson2020}. Su capacidad para combinar la eficiencia de los contenedores con el aislamiento de máquinas virtuales es su principal ventaja \citep{10.1145/1272996.1273025}.

\subsection{Firecracker}
Firecracker está orientado a proveedores de servicios en la nube y plataformas de cómputo en la nube que requieren micro VMs eficientes y seguras \citep{Jain}. Es una solución ideal para plataformas \textit{serverless} y entornos multi-tenant \citep{246288}.

\subsection{Sarus}
Sarus está dirigido a entornos de HPC y computación científica, donde los usuarios necesitan ejecutar contenedores de forma segura en sistemas de alto rendimiento \citep{Sarus2021}. Su compatibilidad con estándares de contenedores y su enfoque en la seguridad lo hacen ideal para centros de investigación y universidades \citep{B2020}.

\begin{table}[H]
\centering
\scriptsize
\setlength{\tabcolsep}{3pt}
\renewcommand{\arraystretch}{1.1}
\begin{tabular}{|>{\centering\arraybackslash}m{0.18\textwidth}| 
                >{\centering\arraybackslash}m{0.25\textwidth}| 
                >{\centering\arraybackslash}m{0.20\textwidth}| 
                >{\centering\arraybackslash}m{0.25\textwidth}|}
\hline
\textbf{Tecnologías} & \textbf{Licencias} & \textbf{Términos de uso} & \textbf{Costo} \\
\hline
Docker & Apache 2.0 & \href{https://www.docker.com/legal/docker-terms-service/}{link} & \$11-\$24 \\
\hline
Podman & Apache 2.0 & \href{https://github.com/containers/podman/blob/main/LICENSE}{link} & Gratis \\
\hline
Udocker & Apache 2.0 & \href{https://github.com/indigo-dc/udocker/blob/master/LICENSE}{link} & Gratis \\
\hline
Wasm & Apache 2.0 & \href{https://github.com/WebAssembly/design/blob/main/LICENSE}{link} & Gratis \\
\hline
LXC & GNU LGPLv2.1+ & \href{https://linuxcontainers.org/lxc/introduction/}{link} & Gratis \\
\hline
Containerd & Apache 2.0 & \href{https://github.com/containerd/containerd/blob/main/LICENSE}{link} & Gratis \\
\hline
LXD & AGPL-3.0 & \href{https://github.com/canonical/lxd}{link} & Gratis \\
\hline
Rkt & Apache 2.0 & \href{https://github.com/rkt/rkt/blob/master/LICENSE}{link} & Descontinuado \\
\hline
Singularity & BSD 3-Clause & \href{https://github.com/sylabs/singularity/blob/main/LICENSE.md}{link} & CE: Gratis, PRO: \$30/año \\
\hline
runC & Apache 2.0 & \href{https://github.com/opencontainers/runc/blob/main/LICENSE}{link} & Gratis \\
\hline
CRI-O & Apache 2.0 & \href{https://github.com/cri-o/cri-o/blob/main/LICENSE}{link} & Gratis \\
\hline
Hyper-V containers & Windows Propietaria & \href{https://learn.microsoft.com/es-es/virtualization/windowscontainers/images-eula}{link} & \$1,176 USD \\
\hline
OpenVZ & GPL v2 & \href{https://openvz.org/}{link} & Gratis \\
\hline
Linux VServer & GPL v2 & \href{http://linux-vserver.org/}{link} & Gratis \\
\hline
Google gVisor & Apache 2.0 & \href{https://github.com/google/gvisor}{link} & Gratis \\
\hline
Kata Containers & Apache 2.0 & \href{https://github.com/kata-containers/kata-containers/blob/main/LICENSE}{link} & Gratis \\
\hline
Firecracker & Apache 2.0 & \href{https://github.com/firecracker-microvm/firecracker}{link} & Gratis \\
\hline
Sarus & BSD 3-Clause & \href{https://github.com/eth-cscs/sarus}{link} & Gratis \\
\hline
\end{tabular}
\caption{Comparativa de tecnologías de contenerización, licencias, términos de uso y costos}
\end{table}

\begin{table}[H]
\centering
\scriptsize
\setlength{\tabcolsep}{3pt}
\renewcommand{\arraystretch}{1.1}
\begin{tabularx}{\textwidth}{|p{0.2\textwidth}|p{0.75\textwidth}|}
\hline
\textbf{Tecnología} & \textbf{Interfaz de Uso} \\
\hline
Docker & \CLI\ principalmente, con Docker Desktop para interfaz gráfica. \\
\hline
Podman & \CLI\ similar a Docker, sin daemon. Opcional Podman Desktop. \\
\hline
Udocker & \CLI\ específica para ejecutar contenedores sin privilegios root. \\
\hline
Wasm (WebAssembly) & Ejecución través de navegadores web, \API\@s de JavaScript. \\
\hline
LXC & \CLI\ mediante comando lxc, sin interfaz gráfica oficial. \\
\hline
Containerd & \CLI\ con herramientas como ctr, backend para otras herramientas. \\
\hline
LXD & \CLI\ mediante lxd/lxc, con interfaz web LXD Web \UI\. \\
\hline
Rkt & \CLI\ mediante comandos como rkt run (descontinuado). \\
\hline
Singularity & \CLI\ mediante comandos singularity para gestión de contenedores. \\
\hline
runC & \CLI\ mediante comandos runc, runtime bajo Docker y Kubernetes. \\
\hline
CRI-O & \CLI\, interactúa con Kubernetes, sin interfaz gráfica dedicada. \\
\hline
Hyper-V containers & \CLI\ (PowerShell) o Hyper-V Manager para VMs. \\
\hline
OpenVZ & \CLI\ mediante comandos vzctl, con interfaces gráficas de terceros. \\
\hline
Linux VServer & \CLI\ mediante comandos vserver para gestión. \\
\hline
Google gVisor & \CLI\ mediante comandos estándar de Docker con seguridad adicional. \\
\hline
Kata Containers & \CLI\ mediante kata-runtime, integración con Kubernetes. \\
\hline
Firecracker & \CLI\ mediante \API\ RESTful y herramientas firecracker. \\
\hline
Sarus & \CLI\ mediante comando sarus para entornos \HPC\. \\
\hline
\end{tabularx}
\caption{Interfaz de uso de cada VBC}\label{tab:interfaz-vbc}
\end{table}

\begin{table}[H]
\centering
\scriptsize
\setlength{\tabcolsep}{3pt}
\renewcommand{\arraystretch}{1.1}
\begin{tabularx}{\textwidth}{|p{0.2\textwidth}|X|}
\hline
\textbf{Tecnología} & \textbf{Integración con Proveedores de Cloud} \\
\hline
Containerd & Integración fuerte con Kubernetes, que a su vez se integra con proveedores de nube como \AWS\ (EKS), \GCP\ (GKE), Azure (AKS) y otros. \\
\hline
CRI-O & Integración directa con Kubernetes, lo que le permite ser utilizado en proveedores de nube como \AWS\ (EKS), \GCP\ (GKE), Azure (AKS), y otros servicios de orquestación de contenedores. \\
\hline
Docker & Integración con \AWS\ (ECR, ECS), \GCP\ (GCR, GKE), Azure (ACR, AKS), y otros proveedores a través de herramientas como Docker Compose, Docker Swarm y Docker Desktop. \\
\hline
Google gVisor & Integración con Google Cloud, especialmente en Google Kubernetes Engine (GKE), para agregar una capa adicional de seguridad a los contenedores. \\
\hline
Hyper-V containers & Integración exclusiva con Microsoft Azure, especialmente con Azure Kubernetes Service (AKS) y otras soluciones basadas en Hyper-V. \\
\hline
Kata Containers & Soporta proveedores de nube pública como \AWS\, Google Cloud, y Azure a través de Kubernetes, proporcionando aislamiento similar a máquinas virtuales en entornos de contenedores. \\
\hline
Linux VServer & Utilizado principalmente en proveedores de hosting dedicados y servidores privados, sin integración directa con proveedores de nube pública como \AWS\, \GCP\ o Azure. \\
\hline
LXC & Se puede integrar en plataformas de nube privada y algunas soluciones híbridas. Se usa en servidores de nube como OpenStack, pero no tiene una integración directa con plataformas públicas principales. \\
\hline
LXD & Puede integrarse con plataformas de nube privada, como OpenStack, para ofrecer contenedores ligeros que emulan máquinas virtuales. No tiene integración directa con los proveedores de nube pública principales, pero puede ser utilizado en soluciones personalizadas. \\
\hline
OpenVZ & Tradicionalmente usado en proveedores de hosting como OVH, aunque su uso ha disminuido frente a soluciones más modernas. La integración con nubes públicas es limitada y generalmente personalizada. \\
\hline
Podman & Compatible con \AWS\ (ECR), \GCP\ (GCR), Azure (ACR), aunque su integración con orquestadores como Kubernetes es más reciente y menos prevalente que Docker. \\
\hline
Rkt & Aunque estaba integrado con Kubernetes y otras plataformas, su descontinuación limita la integración con proveedores de nube. En el pasado, soportaba plataformas como \AWS\ y \GCP\ (Google Cloud). \\
\hline
runC & Integración con Kubernetes, que se usa ampliamente en proveedores de nube como \AWS\ (EKS), \GCP\ (GKE), y Azure (AKS) para la orquestación de contenedores. \\
\hline
Singularity & Utilizado principalmente en entornos de computación científica y HPC. Puede integrarse con proveedores como \AWS\ (HPC, Batch) y \GCP\ (Compute Engine) para tareas específica de alto rendimiento. \\
\hline
Udocker & Generalmente se usa en entornos sin privilegios de root y en plataformas como \HPC. No tiene una integración directa con proveedores de nube a gran escala. \\
\hline
Wasm (WebAssembly) & Integración principalmente con servicios de computación en la nube como \AWS\ Lambda, \GCP\ (Cloud Functions), y Azure Functions, ya que permite la ejecución de código en la nube sin dependencia del sistema operativo subyacente. \\
\hline
\end{tabularx}
\caption{Integración cloud de cada VBC}\label{tab:integracion-cloud-vbc}
\end{table}
\begin{table}[H]
    \centering
    \includegraphics[width=\textwidth] {tablas-images/cp3/medicion-gartner.png}
    \caption{Tabla de medición para el cuadrante gartner}\label{tab:tabla-medicion-gartner}
\end{table}
\begin{figure}[H]
    \centering
    \includegraphics[scale=0.1] {tablas-images/cp3/cuadrante-gartner.png}
    \caption{Cuadrante de Gartner de cada VBC}\label{fig:tabla-cuadrante-gartner}
\end{figure}
\begin{table}[H]
    \centering
    \includegraphics[width=\textwidth] {tablas-images/cp3/entorno-ejecucion.png}
    \caption{Entornos de ejecución de cada VBC}\label{tab:tabla-entorno-ejecucion}
\end{table}
\begin{table}[H]
    \centering
    \includegraphics[width=\textwidth] {tablas-images/cp3/matriz-dofa.png}
    \caption{Tabla de matriz DOFA para el cuadrante gartner}\label{tab:tabla-matriz-dofa}
\end{table}


\begin{table}[H]
\centering
\rowcolors{2}{gray!10}{white}
\begin{tabularx}{\textwidth}{>{\raggedright\arraybackslash}X >{\raggedright\arraybackslash}X}
\rowcolor{gray!30}
\textbf{Tecnología} & \textbf{Enlace a la Documentación} \\

Docker & \href{https://docs.docker.com/}{link} \\
Podman & \href{https://podman.io/docs}{link} \\
Udocker & \href{https://github.com/indigo-dc/udocker}{link} \\
Wasm (WebAssembly) & \href{https://webassembly.org/docs/faq/}{link} \\
LXC & \href{https://linuxcontainers.org/incus/docs/main/}{link} \\
Containerd & \href{https://containerd.io/docs/}{link} \\
LXD & \href{https://linuxcontainers.org/incus/docs/main/}{link} \\
Rkt & \href{https://github.com/rkt/rkt}{link} \\
Singularity & \href{https://docs.sylabs.io/guides/4.3/user-guide/}{link} \\
runC & \href{https://github.com/opencontainers/runc}{link} \\
CRI-O & \href{https://github.com/cri-o/cri-o}{link} \\
Hyper-V containers & \href{https://docs.microsoft.com/en-us/virtualization/windowscontainers/}{link} \\
OpenVZ & \href{https://openvz.org/}{link} \\
Linux VServer & \href{http://linux-vserver.org/Documentation}{link} \\
Google gVisor & \href{https://gvisor.dev/docs/}{link} \\
Kata Containers & \href{https://katacontainers.io/docs/}{link} \\
Firecracker & \href{https://firecracker-microvm.github.io/}{link} \\
Sarus & \href{https://github.com/eth-cscs/sarus}{link} \\

\end{tabularx}
\caption{Enlaces a la documentación de tecnologías de contenerización}
\end{table}