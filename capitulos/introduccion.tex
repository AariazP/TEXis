\ChapterImageStar[cap:introduccion]{Introducción}{./images/fondo.png}\label{cap:introduccion}
\mbox{}\\
La computación en la nube \textit{(Cloud Computing)} es uno de los conceptos con más crecimiento en la industria de la tecnología\citep{Jayaweera2024}. Las organizaciones han identificado en esta forma de computación una manera de aprovisionamiento de recursos informáticos rápida y según la demanda. Entre sus principales beneficios se incluyen la flexibilidad, la escalabilidad y la eficiencia en costos\citep{Ahmadi2024}. La adopción de estos recursos ha transformado el desarrollo de soluciones tecnológicas, lo cual ha posibilitado que la planificación, el análisis, el diseño, el desarrollo, las pruebas y el mantenimiento se realicen completamente en la nube. Esto ha dado origen a aplicaciones nativas de este entorno, conocidas como \textit{cloud native apps}.\\
Las \textit{cloud native apps} permiten a las organizaciones implementar soluciones complejas con un rendimiento mejorado, distribuyendo sus cargas de trabajo en múltiples entornos de nube y optimizando el retorno de inversión\citep{Alonso2023}. Con el aumento en el uso de estas aplicaciones nativas, ha aumentado también la demanda por por infraestructura que las soporte. Para mitigar los costos que implica el crecimiento de estos equipos físicos se busca la posibilidad de consolidar cada vez más los recursos de \TI. La virtualización es útil debido a que permite una consolidación de recursos según las necesidades organizacionales. Anteriormente el despliegue de aplicaciones se realizaba directamente sobre el sistema de la máquina física; actualmente, la gran mayoría se ejecuta sobre sistemas virtualizados\citep{Jain2016}. Las máquinas virtuales, o de sistema completo, han sido hasta ahora el estándar de facto para la segmentación de infraestructura de \TI;\@ sin embargo, la virtualización ligera, también conocida como Virtualización Basada en Contenedores (\textbf{\VBC})\footnote{Las siglas utilizadas en este documento se explican en el capítulo~\ref{cap:siglas}.}, se ha ido posicionando como una alternativa moderna a las máquinas virtuales.\\ \\
En este contexto, desde la aparición de Docker en 2013, la virtualización ligera ha transformado el desarrollo de software, fortaleciendo prácticas como DevOps, donde la escalabilidad y la replicabilidad son fundamentales\citep{Docker2021}. Docker ha experimentado un notable crecimiento en su adopción, debido a su capacidad para ejecutar aplicaciones en el mismo entorno en el que fueron construidas, sin importar el lugar donde se implementen. El crecimiento de Docker se ve evidenciado en el uso de \textit{Docker images} por parte de los desarrolladores. En 2023 se registraron 130 mil millones de descargas, cifra que aumentó a 242 mil millones en 2024\citep{Docker2024}. A partir del auge de Docker, surgieron nuevas tecnologías de contenerización, la aparición de estas puede percibirse inicialmente como una ventaja para organizaciones, desarrolladores y demás actores de \TI\;\@ sin embargo, la proliferación de estas herramientas puede representar un reto al momento de elegir la idónea en una arquitectura de solución.\\
Este trabajo aborda la situación ya expuesta, cuyo objetivo principal es proponer una arquitectura de solución basada en contenedores para el Grupo de Investigación en Redes, Información y Distribución (\GRID) de la Universidad del Quindío. Inicialmente, se realiza una valoración de necesidades de la organización cliente, destacando sus objetivos misionales enfocados en el apoyo a la docencia, la investigación y la extensión. El desafío consiste en el aprovechamiento de la infraestructura actual del \GRID\ aportando al cumplimiento de sus objetivos misionales. Lo anterior, haciendo uso de los aportes del presente trabajo. Posteriormente, se profundiza en una revisión del estado del arte mediante un estudio de mapeo sistemático \textit{(Systematic Mapping Study --- \SMS\ )}, con el objetivo de comprender las tecnologías de \VBC\ y los dominios de \TI\ en los que se desarrollan. Paso seguido, se realiza un análisis \DAR\ \textit{(Decision Analysis and Resolution)} basado en el modelo de \CMMI\, el cual permite definir la tecnología de contenedores adecuada
en la implementación de una solución. A partir de este análisis, se desarrolla la arquitectura de solución con base en las necesidades del grupo de investigación.