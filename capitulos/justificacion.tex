\chapter*{Justificación}
Actualmente el Grupo de Investigación en Redes, Información y Distribución (GRID) presenta diversas necesidades y oportunidades con relación a los servicios tecnológicos que ofrece a la Universidad del Quindío, en apoyo a sus objetivos misionales de docencia, investigación y extensión. En este contexto, el GRID busca identificar tecnologías emergentes que permitan potenciar su capacidad de brindar servicios tecnológicos avanzados para su propio beneficio y de la comunidad académica de su influencia. Con relación a lo anterior, la virtualización basada en procesos se presenta como una oportunidad para potenciar la gestión de recursos y servicios de tecnología informática (TI). Aunque el GRID cuenta con una infraestructura basada en máquinas virtuales y gestionadas mediante un hipervisor tipo I, aún se requiere de instancias computacionales más livianas para ser usadas en su oferta de servicios computacionales hacia la comunidad académica, representada por los estudiantes de Ingeniería de Sistemas y Computación de la Universidad del Quindío.
Como menciona Sepúlveda-Rodríguez, Chavarro-Porras, Sanabria-Ordoñez, Castro y Matthews (2022), las tecnologías de virtualización han proliferado en los últimos años y constituyen la base subyacente de infraestructuras modernas como el cloud computing. La VBC representan una opción de virtualización que requieren menos recursos computacionales para su operación (Xavier et al., 2013) y que en paralelo con las máquinas virtuales existentes en el GRID podrían constituir una oferta de servicios de TI con mayor diversificación, escalabilidad, flexibilidad y mantenibilidad para satisfacer los requerimientos del contexto académico del grupo de investigación. 