\ChapterImageStar[cap:justificacion]{Justificación}{./images/fondo.png}\label{cap:justificacion}
Actualmente, el Grupo de Investigación en Redes, Información y Distribución 
(\GRID) enfrenta diversas necesidades y oportunidades en relación con los servicios 
tecnológicos que ofrece a la Universidad del Quindío, en apoyo a sus objetivos 
misionales de docencia, investigación y extensión. En este contexto, el \GRID\ 
orienta sus esfuerzos hacia la identificación de tecnologías emergentes que fortalezcan 
su capacidad de ofrecer servicios tecnológicos avanzados, tanto para beneficio propio 
como para la comunidad académica de su área de influencia. En este marco, la virtualización basada en procesos se vislumbra como una alternativa 
estratégica para la gestión de recursos y servicios de tecnología informática 
(\TI). Si bien el \GRID\ dispone de una infraestructura sustentada en máquinas virtuales, 
gestionadas mediante un hipervisor tipo I, persiste la necesidad de contar con instancias 
computacionales más livianas que permitan ampliar la oferta de servicios hacia la comunidad 
académica, particularmente a los estudiantes del programa de Ingeniería de Sistemas y 
Computación de la Universidad del Quindío.\\
Como señalan diversos autores, las tecnologías de virtualización han experimentado un 
crecimiento acelerado en los últimos años, consolidándose como la base fundamental de 
infraestructuras modernas, entre ellas el cloud computing~\citep{Sepulveda-Rodriguez2022}. 
En este sentido, la virtualización basada en contenedores (\VBC) se presenta como una opción 
al requerir recursos computacionales más ligeros para su operación~\citep{Xavier2013}. 
Su incorporación, en complemento a las máquinas virtuales ya desplegadas en el \GRID, 
posibilita el diseño de un portafolio de servicios de \TI\ más diversificado, escalable, 
flexible y de fácil mantenimiento. De este modo, la adopción de tecnologías de virtualización basadas en contenedores no solo 
responde a las necesidades actuales del grupo de investigación, sino que también abre la 
puerta a nuevas oportunidades de innovación y transferencia de conocimiento dentro del 
contexto académico. Con ello, el \GRID\ podría consolidarse como un referente institucional 
en el aprovechamiento de tecnologías de virtualización, incrementando su capacidad para 
atender demandas crecientes de infraestructura computacional y apoyando con mayor solidez 
la misión universitaria.