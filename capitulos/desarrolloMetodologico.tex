\chapter*{Desarrollo Metodológico}

\section*{1. Caracterización del GRID}
El Grupo de Investigación en Redes, Información y Distribución (GRID) de la Universidad del Quindío se dedica a la educación, investigación y extensión, siendo los objetivos misionales de la Universidad del Quindío. 
Desde el grupo de investigación se busca ofrecer servicios tecnológicos avanzados a la comunidad académica, especialmente a los estudiantes de Ingeniería de Sistemas y Computación.
Es por esto que en este apartado se caracterizó el GRID, identificando sus necesidades y oportunidades con relación a las tecnologías de virtualización basadas en contenedores (VBC).

\section*{2. Mapeo SMS}
El mapeo sistemático de estudios (SMS) se realizó para identificar y analizar las tecnologías de virtualización basadas en contenedores (VBC) más relevantes 
y utilizadas en la actualidad. Este proceso incluyó la revisión de literatura académica, artículos técnicos y estudios de caso, con el fin de establecer un 
panorama claro sobre las opciones disponibles y sus características principales.

\section*{3. Identificación y caracterización de las tecnologías VBC}
En esta sección se llevó a cabo un análisis detallado de las tecnologías de virtualización basadas en contenedores (VBC) identificadas en el mapeo SMS. Se examinaron sus características, ventajas y desventajas, así como su aplicabilidad en el contexto del GRID.

\section*{4. Benchmarking de tecnologías VBC}
El benchmarking se realizó para comparar las tecnologías VBC en función de criterios como uso de CPU, throughput de red, I/O. Esta evaluación permitió identificar las soluciones más adecuadas para las necesidades del GRID.

\section*{5. Análisis DAR}
El análisis de riesgos y oportunidades (DAR) se llevó a cabo para evaluar el impacto potencial de la implementación de tecnologías VBC en el GRID. Se identificaron los principales riesgos asociados y se propusieron estrategias para mitigarlos.

\section*{6. Diseño de la solución arquitectónica}

\section*{7. Implementación de la solución}

\section*{8. Validación de la solución}

