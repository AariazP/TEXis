\ChapterImageStar[cap:cumplimiento-objetivos]{Cumplimiento de objetivos}{./images/fondo.png}\label{cap:cumplimiento-objetivos}
\mbox{}\\
\section{Cumplimiento de objetivos}
\noindent
En este capítulo se presenta un análisis del cumplimiento de los objetivos planteados al inicio del proyecto. Se evalúa cómo cada objetivo específico ha sido abordado y alcanzado a lo largo del desarrollo del trabajo.

\begin{table}[H]
\centering
\caption{Cumplimiento del objetivo específico 1}
\label{tab:cumplimiento-objetivo-1}
\begin{tabular}{|p{6cm}|p{9cm}|}
\hline
\multicolumn{2}{|c|}{\textbf{Objetivo específico 1:}} \\
\hline
\multicolumn{2}{|p{15cm}|}{\textit{Reconocer necesidades, problemas y oportunidades con relación a la gestión de \textcolor{blue}{RSI} en los \textcolor{blue}{ETAI}}} \\
\hline
\textbf{Detalle de cumplimiento} & \textbf{Observaciones} \\
\hline
Ver capítulo \textcolor{blue}{9} Desarrollo Metodológico, Etapa 2: Análisis. Se realizaron procesos cuali-cuantitativos para obtener información del grupo objetivo. & 
\textbf{Paso 1:} Reconocimiento de las instituciones pertenecientes al grupo objetivo (ver sección \textcolor{blue}{9.2.1}).

- \textbf{Formato 1}, Identificación de instituciones (apéndice \textcolor{blue}{A.1}). Identificación de la \textit{Universidad del Quindío} (sección \textcolor{blue}{9.2.1.1}) e identificación de la \textit{Universidad Tecnológica de Pereira} (sección \textcolor{blue}{9.2.1.2}).

\textbf{Paso 2:} Insumos para el diseño del sistema de gobierno y gestión.

- \textbf{Encuesta 1}, formulario (apéndice \textcolor{blue}{A.2}) y respuestas (apéndice \textcolor{blue}{A.3}).

\textbf{Paso 3:} Caracterización de los \textit{ETAI}.

- \textbf{Encuesta 2}, formulario (apéndice \textcolor{blue}{A.4}) y respuestas (apéndice \textcolor{blue}{A.5}). \\
\hline
\end{tabular}
\end{table}
