\ChapterImagePrelim[cap:siglas]{Siglas y Abreviaturas}{./images/fondo.png}\label{cap:siglas}
\mbox{}\\
A continuación, se presentan las siglas y abreviaturas utilizadas en este documento, junto con su significado completo para facilitar la comprensión.
\begin{description}
  \item[API] Interfaz de Programación de Aplicaciones
  \item[AWS] Amazon Web Services
  \item[CLI] Interfaz de Línea de Comandos (\textit{Command Line Interface})
  \item[CMMI] Capability Maturity Model Integration
  \item[CNCF] Cloud Native Computing Foundation
  \item[CPD] Centro de Procesamiento de Datos
  \item[CPU] Unidad Central de Procesamiento (\textit{Central Processing Unit})
  \item[CS] Ciencia de la Computación (\textit{Computer Science})
  \item[DAR] Decision Analysis and Resolution
  \item[GRID] Grupo de Investigación en Redes, Información y Distribución
  \item[HPC] Computación de Alto Rendimiento (\textit{High Performance Computing})
  \item[IEC] Comisión Electrotécnica Internacional (\textit{International Electrotechnical Commission})
  \item[ISO] Organización Internacional de Normalización (\textit{International Organization for Standardization})
  \item[OCI] Iniciativa de Contenedores Abiertos (\textit{Open Container Initiative})
  \item[OS] Sistema Operativo (\textit{Operating System})
  \item[PMBOK] Project Management Body of Knowledge
  \item[PMI] Instituto de Gestión de Proyectos (\textit{Project Management Institute})
  \item[PMV] Producto Mínimo Viable
  \item[SMS] Systematic Mapping Study
  \item[TI] Tecnologías de la Información
  \item[TOGAF] The Open Group Architecture Framework
  \item[UI] Interfaz de Usuario (\textit{User Interface})
  \item[VBC] Virtualización Basada en Contenedores
  \item[VM] Máquina Virtual (\textit{Virtual Machine})
\end{description}